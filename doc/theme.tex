\documentclass[a4paper, 10pt, oneside, notitlepage, onecolumn]{jsarticle}
\usepackage{ascmac}
\usepackage[osf]{mathpazo}
\pagestyle{myheadings}

\author{}

\begin{document}
\begin{flushright}
    2010年10月7日
\end{flushright}
\begin{center}
    4J後期実験テーマ概要\\
\hspace{0.1em}
    \textsc{\LARGE 関数型言語の設計と実装}
\end{center}
\begin{flushright}
    L班 07317 小堀 育男\\
        07322 酒本 典明
\end{flushright}
\section{目的}
    教育目的で作られたコンパイラ――
    特にML系の言語ではMinCamlが有名であるが、
    それは、ML系の言語としては、代数的データ型やパターンマッチングなどの
    重要な機能を欠いており、その機能を補うのが目的である。

    また、MinCamlはSPARCとPowerPCをターゲットとしているが、
    我々は一般に普及しているx86をターゲットとする。

\section{概要}
\subsection{nibKameの仕様}
    言語仕様は以下のとおりとする。
    \begin{itemize}
	    \item 型推論をする
	    \item 以下の組み込み型を用意する。
	    \begin{itemize}
	        \item 整数型、浮動小数点数型
	        \item 文字型
	        \item Bool型
	        \item Array型、List型
	    \end{itemize}
	    \item 無名関数を使えるようにする。
	    \item \texttt{let}式、\texttt{match}式、
	        \texttt{fun}式を使えるようにする。
    \end{itemize}

\subsection{コンパイラの仕様}
    コンパイラの仕様は以下のとおりとする。
    \begin{itemize}
        \item 対象言語はnibKameとする。
        \item ターゲットはx86アセンブリとする。
        \item 主な記述言語はOCamlとする。
    \end{itemize}

\end{document}
