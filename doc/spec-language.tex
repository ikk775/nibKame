\documentclass[a4paper,titlepage,disablejfam]{jsbook}

\usepackage[dvipdfmx]{graphicx}

% 定理類似環境:
\newtheorem{theo}{定理}
\newtheorem{defi}[theo]{定義}
\newtheorem{lemm}[theo]{補題}
\newtheorem{prop}[theo]{命題}
\renewcommand{\proofname}{\bf 証明}
\pagestyle{myheadings}

% 人名
\newcommand{\sakamoto}{酒本 典明}
\newcommand{\kobori}{小堀 育男}

% 見出し:
\renewcommand{\lstlistlistingname}{プログラム一覧}
\newsavebox{\articleauthor}
\newcommand{\responsibility}[1]{\nopagebreak[2]\begin{flushright}[文責: #1]\end{flushright}}
\newcommand{\defresponsible}[2]{\newenvironment{#1}[3][\relax]{%
\ifx##1\relax#2[##2\quad(##3)]{##2}%
\else#2[##1\quad(##3)]{##2}\fi%
\sbox{\articleauthor}{##3}}%
{\responsibility{\usebox{\articleauthor}}}}
\defresponsible{resbonsiblechapter}{\chapter}
\defresponsible{resbonsiblesection}{\section}

% 略語:
\newcommand{\algorithmW}{algorithm $\mathscr{W}$}
\newcommand{\algorithmU}{algorithm $\mathscr{U}$}

% 記号:
\DeclareSymbolFont{symbolsC}{U}{txsyc}{m}{n}
\DeclareMathSymbol{\strictif}{\mathrel}{symbolsC}{74}
\DeclareMathSymbol{\boxright}{\mathrel}{symbolsC}{128}
\newcommand{\R}{\mathbb{R}}
\newcommand{\Z}{\mathbb{Z}}
\newcommand{\N}{\mathbb{N}}
\newcommand{\msubset}{\subseteq}
\newcommand{\mnsubset}{\nsubseteq}
\newcommand{\mpsubset}{\subset}
\newcommand{\mnpsubset}{\nsubset}
\newcommand{\msupset}{\supseteq}
\newcommand{\mnsupset}{\nsupseteq}
\newcommand{\mpsupset}{\supset}
\newcommand{\mnpsupset}{\nsupset}
\newcommand{\limpl}{\supset}
\newcommand{\gtrdotrel}{\mathrel{\gtrdot}}
\newcommand{\Rrel}{\mathrel{R}}
\newcommand{\curlyveeord}{\mathord{\curlyvee}}
\newcommand{\curlywedgeord}{\mathord{\curlywedge}}

\newcommand{\powerset}{\mathfrak{P}}
\newcommand{\domain}{\mathop{\mathfrak{Dom}}}
\newcommand{\codomain}{\mathop{\mathfrak{Cod}}}

\newcommand{\mathnkop}[1]{\mathop{\mathsf{#1}}}
\newcommand{\mathnkenv}[1]{\mathcal{#1}}
\newcommand{\mathnkval}[1]{\mathsf{#1}}
\newcommand{\mathnkvariant}[1]{\mathsf{#1}}
\newcommand{\mathset}[1]{\mathit{#1}}
\newcommand{\mathfunc}[1]{\mathit{#1}}
\newcommand{\newmonadicopdot}[2]{\newcommand{#1}[2]{#2##1.\,##2}}
\newmonadicopdot{\foralldot}{\forall}
\newmonadicopdot{\existsdot}{\exists}
\newmonadicopdot{\lambdadot}{\lambda}
\newcommand{\fixdot}[2]{\mathnkop{fix}#1\;#2}
\newcommand{\fundot}[2]{\mathnkop{fun}#1\rightarrow#2}
\newcommand{\semanticS}[1]{\mathcal{S}\left\llbracket#1\right\rrbracket}
\newcommand{\semanticM}[1]{\mathcal{M}\left\llbracket#1\right\rrbracket}
\newcommand{\mypair}[2]{\left(#1,#2\right)}

\newcommand{\lambdaIf}{\mathop{\mathrm{if}}}
\newcommand{\lambdaThen}{\mathop{\mathrm{then}}}
\newcommand{\lambdaElse}{\mathop{\mathrm{else}}}
\newcommand{\lambdaLet}{\mathop{\mathrm{let}}}
\newcommand{\lambdaIn}{\mathop{\mathrm{in}}}

\newcommand{\semanticSVal}{\mathop{\pi_\mathrm{val}}}
\newcommand{\semanticSVEnv}{\mathop{\pi_\rho}}

\newcommand{\ltrue}{\top}
\newcommand{\lfalse}{\bot}
\newcommand{\defeq}{\triangleq} % 定義時に使う等号

\newcommand{\envExpr}{\mathnkenv{E}}
\newcommand{\envType}{\mathnkenv{T}}
\newcommand{\envVariant}{\mathnkenv{V}}
\newcommand{\envPattern}{{\mathnkenv{E}_p}}
\newcommand{\substExpr}{{\Sigma_\mathnkenv{E}}}
\newcommand{\substType}{{\Sigma_\mathnkenv{T}}}
\newcommand{\substVariant}{{\Sigma_\mathnkenv{V}}}
\newcommand{\applysubst}[2]{\mathfrak{S}\ #1\ #2}
\newcommand{\removeassoc}[2]{\mathfrak{Rm}\ #1\ #2}
\newcommand{\patternandenv}[2]{#1\Rsh#2}
\newcommand{\valueandenv}[2]{\mypair{#1}{#2}}
\newcommand{\domainrestrict}[2]{#1\mathbin{\upharpoonright}#2}
\newcommand{\freevars}[1]{\mathit{FV}(#1)}
\newcommand{\freetypevars}[1]{\mathit{FTV}(#1)}
\newcommand{\boundvars}[1]{\mathit{BV}(#1)}
\newcommand{\clauseor}{\mathbin{|}}
\newcommand{\patternor}{\mathbin{|}}
\newcommand{\patternand}{\mathbin{@}}
\newcommand{\patternany}{\,\underline{\hspace{0.7em}}\,}

% 日本語特有の記号
\newcommand{\jpdash}{―\nobreak\hspace{-0.5zw}\nobreak―\nobreak\hspace{-0.5zw}\nobreak―}

\newcommand{\rulename}[1]{\text{\bfseries\scshape #1}}
\newcommand{\typename}[1]{\text{\sffamily\bfseries #1}}
\newcommand{\widevec}[1]{\overrightarrow{#1}}
\newcommand{\varrange}[1]{\boldsymbol{\widetilde{#1}}} %\widetriangle \widering \wideparen \widetilde \widehat のいづれかが良いと思う
\newcommand{\refEq}[1]{式(\ref{#1})}
\newcommand{\refTbl}[1]{表\ref{#1}}
\newcommand{\refFig}[1]{図\ref{#1}}
\newcommand{\refCh}[1]{第\ref{#1}章}
\newcommand{\refSc}[1]{第\ref{#1}節}
\newcommand{\refSsc}[1]{第\ref{#1}項}
\newcommand{\refSssc}[1]{第\ref{#1}目}

% 罫線に太線を追加
\makeatletter
\def\@arrayrule{\@addtopreamble{%
\vrule \@width \arrayrulewidth}}
\def\Hline{\noalign{\hrule height 3\arrayrulewidth}}
%\def\LTHline{\noalign{\hrule height 3\arrayrulewidth}}
\def\Vline{\vrule width 3\arrayrulewidth}
%\def\LTHline{\LT@Hline}
\def\LT@Hline{%
  \noalign{\ifnum0=`}\fi
    \penalty\@M
    \futurelet\@let@token\LT@@Hline}
\def\LT@@Hline{%
  \ifx\@let@token\Hline
    \global\let\@gtempa\@gobble
    \gdef\LT@sep{\penalty-\@medpenalty\vskip\doublerulesep}%
  \else
    \global\let\@gtempa\@empty
    \gdef\LT@sep{\penalty-\@lowpenalty\vskip-\arrayrulewidth}%
  \fi
  \ifnum0=`{\fi}%
  \multispan\LT@cols
     \unskip\leaders\hrule\@height3\arrayrulewidth\hfill\cr
  \noalign{\LT@sep}%
  \multispan\LT@cols
     \unskip\leaders\hrule\@height3\arrayrulewidth\hfill\cr
  \noalign{\penalty\@M}%
  \@gtempa}
\def\LT@makecaption#1#2#3{%
  \LT@mcol\LT@cols c{\hbox to\z@%
  {\hss\parbox[t]\LTcapwidth{%
    \sbox\@tempboxa{#1{#2\hskip1zw\relax}#3}%
    \ifdim\wd\@tempboxa>\hsize
      #1{#2\hskip1zw\relax}#3%
    \else%
      \hbox to\hsize{\hfil\box\@tempboxa\hfil}%
    \fi%
    \endgraf\vskip\baselineskip}%
  \hss}}}%
\def\LT@array[#1]#2{%
  \refstepcounter{table}\stepcounter{LT@tables}%
  \if l#1%
    \LTleft\z@ \LTright\fill
  \else\if r#1%
    \LTleft\fill \LTright\z@
  \else\if c#1%
    \LTleft\fill \LTright\fill
  \fi\fi\fi
  \let\LT@mcol\multicolumn
  \let\LT@@tabarray\@tabarray
  \let\LT@@hl\hline
  \def\@tabarray{%
    \let\hline\LT@@hl
    \LT@@tabarray}%
  \let\\\LT@tabularcr\let\tabularnewline\\%
  \def\newpage{\noalign{\break}}%
  \def\pagebreak{\noalign{\ifnum`}=0\fi\@testopt{\LT@no@pgbk-}4}%
  \def\nopagebreak{\noalign{\ifnum`}=0\fi\@testopt\LT@no@pgbk4}%
  \let\hline\LT@hline \let\kill\LT@kill\let\caption\LT@caption
  \let\Hline\LT@Hline
% \@tempdima\ht\strutbox%                                       変更1
  \iftdir\@tempdima\ht\tstrutbox\else\@tempdima\ht\strutbox\fi%  <-
  \let\@endpbox\LT@endpbox
  \ifx\extrarowheight\@undefined
    \let\@acol\@tabacol
    \let\@classz\@tabclassz \let\@classiv\@tabclassiv
    \def\@startpbox{\vtop\LT@startpbox}%
    \let\@@startpbox\@startpbox
    \let\@@endpbox\@endpbox
    \let\LT@LL@FM@cr\@tabularcr
  \else
    \advance\@tempdima\extrarowheight
    \col@sep\tabcolsep
    \let\@startpbox\LT@startpbox\let\LT@LL@FM@cr\@arraycr
  \fi
  \setbox\@arstrutbox\hbox{\vrule
    \@height \arraystretch \@tempdima
%   \@depth \arraystretch \dp \strutbox%          変更2
    \iftdir\@depth \arraystretch \dp \tstrutbox%   <-
    \else\@depth \arraystretch \dp \strutbox\fi%   <-
    \@width \z@}%
  \let\@sharp##\let\protect\relax
   \begingroup
    \@mkpream{#2}%
    \xdef\LT@bchunk{%
       \global\advance\c@LT@chunks\@ne
       \global\LT@rows\z@\setbox\z@\vbox\bgroup
       \LT@setprevdepth
       \tabskip\LTleft \noexpand\halign to\hsize\bgroup
      \tabskip\z@ \@arstrut \@preamble \tabskip\LTright \cr}%
  \endgroup
  \expandafter\LT@nofcols\LT@bchunk&\LT@nofcols
  \LT@make@row
% \m@th\let\par\@empty%                               変更3
  \iftdir\m@th\let\par\@@par%                          <-
  \else\m@th\let\par\@empty\fi%                        <-
  \everycr{}\lineskip\z@\baselineskip\z@
  \LT@bchunk}

\makeatother

% listings用設定
\lstdefinelanguage{nibkame}{%
   morekeywords={%
        let,letrec,in,if,match,with,type,fun,+,-,*,/,+.,-.,*.,\.,(),~-,;,
        and,or,not,unit,Nil%
   },%
%   morekeywords={%
%        ignore
%   },%
%   morekeywords={%
%        print_int,print_float,print_char,print_string
%   },%
%   morekeywords={%
%        hd,tl,null,map,length
%   },%
%   morekeywords={%
%        array-create,array-set,array-ref,array-from-list,array-from-list-with-length
%   },%
   sensitive,% ???
   alsodigit=->,%
   morecomment=[l];,%
   morecomment=[s]{\#|}{|\#},%
   morestring=[b]",%"
   literate=%
       {[|}{[\hskip -1pt$|$}2%
       {|]}{$|$\hskip -1pt]}2%
%      {[]}{\ensuremath{[\hskip -0.1em]}}2%
       {->}{\ensuremath{\rightarrow}~}2%
       {::}{\ensuremath{:\hskip -0.1em:}~}2%
}[keywords,comments,strings]

\lstdefinelanguage[Objective display]{Caml}[Objective]{Caml}{%
    morestring=[d]',%'
%    classoffset=2,
%    morekeywords={int,float,char,string},keywordstyle=\color{red},
%    classoffset=3,
%    morekeywords={list,array},keywordstyle=\color{blue},
%    classoffset=0,
    literate=%
        {[|}{[\hskip -1pt$|$}2%
        {|]}{$|$\hskip -1pt]}2%
%       {[]}{\ensuremath{[\hskip -0.1em]}}2%
        {->}{\ensuremath{\rightarrow}~}2%
        {::}{\ensuremath{:\hskip -0.1em:}~}2%
}
\lstset{%
  %language=[Objective display]Caml,%
  language=nibkame,%
  basicstyle={\normalfont\normalsize\sffamily},%
  commentstyle={\small\ttfamily\itshape\bfseries\upshape},%
  classoffset=1,%
  keywordstyle={\bfseries},%
  %frame=,%
  %framesep=0pt,%
  showstringspaces=false,%
  %numbers=left,%
  %numberstyle={\scriptsize},%
  %stepnumber=1,%
  tabsize=8,%
  lineskip=-0.5ex,%
%
  breaklines=true,%
  linewidth=\the\textwidth,
%  columns=[l]flexible%
  columns=flexible%
}


\begin{document}
%\title{電子情報工学科実験報告書 \\ 関数型言語の設計と実装}
%\title{\scshape The Programming Language Nibkame\\ Language Specification}
\title{nibkame言語仕様}
\author{
小堀 育男 \\
酒本 典明
}
%\date{}

\frontmatter

\maketitle

\tableofcontents
%\listoffigures
%\listoftables
%\lstlistoflistings

\newpage

\mainmatter % 序論も本文の内

\chapter{序論}\label{ch:intro}
\begin{abstract}
 本実験課題選定の理由と目的について論じる.また,実装の概要と報告書の構成を述べる.
\end{abstract}

\section{目的}
関数型言語の有用性はHughes\cite{hughes1989functional}などにより主張され
ていた.現在では,関数型言語に由来する機能が,C++やJava, C\# など
の言語に取り込まれている.またScala\footnote{\url{http://www.scala-lang.org/}}
やF\#{}\footnote{\url{http://msdn.microsoft.com/ja-jp/fsharp}}
など既存の手続き型言語との連携を強く意識した言語も登場している.

関数型言語に特徴的な機能として第一級の関数やクロージャなどがある.これら
は計算機の命令セットの持つ機能からかけ離れている.こういった機能を持った
言語を設計し,実装することで機能に対する理解を深めることを目標とする.

そのために,我々は静的に強く型付けされる関数型言語nibkameを設計しコンパ
イラを実装した.これは学生実験における参照実装として開発された
MinCaml\cite{住井英二郎:2008-04-24}を基にして開発を行なった.

MinCamlおよびMinCamlコンパイラは簡潔な実装(約2000行ほど)で性能の良いコー
ドを生成することに特徴のある関数型言語である.構文はObjective
Caml\footnote{\url{http://caml.inria.fr/ocaml/}}のサブセットであり,算術演算や
タプル構造,破壊的代入の可能な配列,高階関数,再帰と末尾呼び出し,型推論
などの機能が実現されており,レイトレーシングなどの複雑なプログラムを記述
することができる.MinCamlはプログラム言語処理系の教材として用いられるこ
とを主眼としており,その機能選定は実装を簡潔に保つことを重視して行われた.

簡潔さを保つために,多くのアプリケーションプログラムを記述するときに必須
でない機能は省略された.その中にはMinCamlコンパイラの記述に多く用いられ
ている代数的データ型やパターンマッチングなどが存在する.
nibkameではそういった機能と,その他にも生産性に高い影響を与える機能を追
加することでより実用的な言語を開発することを目指した.
MinCamlコンパイラはターゲットとする計算機としてSPARCとPowerPCをサポート
している.しかし実行する計算機の確保の容易さから,nibkameコンパイラはター
ゲットをIA-32アーキテクチャとした.

詳しくは\refCh{ch:lang_design}で述べるが,これらの目標を達成するために次
に挙げる機能の追加を行った.
\begin{itemize}
 \item トップレベル環境における複数の式
 \item 多相関数
 \item 代数的データ型
 \item パターンマッチング
 \item IA-32アーキテクチャ向け機械語の生成
\end{itemize}
また,これを達成するために,複数の定義をひとまとめに扱うモジュール機構と
3番地形式の中間言語の2番地形式への変換,メモリ管理機構の強化が必要となっ
た.


\section{構成} % ここは内容にあわせて書き換えてください.あと
			  % subsectionはオーバースペックな気が.
\begin{comment}
\refCh{ch:preparation}ではコンパイラの動作を理解するために必要となる理論
や規則について概説する.本報告書で用いる論理学や集合の記法についても説明
する.
\refCh{ch:lang_design}では,nibkame言語に搭載された機能のうち関数型言語
に特有なものについて説明し,最終的に実装された言語機能と構文を示す.
\refCh{ch:impl}ではコンパイラのモジュール構造と各フェーズの対応を示し,
コンパイラの実装について説明する.
\refCh{ch:sample-program}では実際のnibkameプログラムの例を示し,コンパイラの動作
と得られる目的プログラムについて説明する.
\refCh{ch:conclude}では実装した機能についてまとめ,これからの課題を議論する.
\end{comment}

%\mainmatter
%\chapter{原理}\label{ch:原理}

\chapter{記法}\label{ch:notation}
\begin{abstract}
本報告書で用いる理論・記法について概説する. 
本章で説かれる内容は本報告書で用いられる範囲に限定し,
かつ,直感的な解りやすさを最優先としたため,
論理的に不正確な箇所が多々あるがお赦し願いたい.
\end{abstract}

\section{全体的な記法について}\label{sc:about-notation}
以下,演算子の結合力を表す際に,
\begin{equation}\label{eq:bonding-strength}
\times \gtrdotrel +
\end{equation}
と表すこととする.
これは,$A\times B + C$という式があったときに$(A\times B)+ C$と解釈することを意味する.

等号系の記号の使い分けは\refTbl{tbl:equal-usage}のようにする.

\begin{table}[hbt]
    \caption{等号系記号の使い分け}\label{tbl:equal-usage}
    \begin{center}
    \begin{tabular}{cp{40zw}}
        \Hline
        \multicolumn{1}{c}{記号} & \multicolumn{1}{c}{目的} \\
        \hline
        $=$ & 通常の等値,即ち数値・真偽値・集合などの等しさを真偽値で表す,または,等しいことを言明する. \\
        $\equiv$ & $=$よりも広い意味で同じであることを表す. \\
        $\defeq$ & 通常の式において左辺の記号が右辺で定義されることを示す.\\
        $\Coloneqq$ & BNFにおいて,左辺の記号が右辺で定義されることを示す.\refSc{sc:about-BNF-form} \\
        \Hline
    \end{tabular}
    \end{center}
\end{table}

変数$e$の動く領域を$\varrange{e}$と表すこととする\footnote{一般的な方法ではない}.

自然数$\N$は0を含むこととする.


\section{集合について}\label{sc:about-set}
元として含むことを表す記号として$\in$,$\ni$を用いる.  部分集合を表す記号として$\msubset$,$\msupset$を,真部分集合を表す記号として$\mpsubset$,$\mpsupset$を用いる. 空集合の記号としては$\emptyset$を用いる.

集合$x$の羃集合は$\powerset x$若しくは$\powerset(x)$と書く
\footnote{$2^x$や$\boldsymbol{2}^x$という表記も一般的.}.

順序対若しくは単純に対は,$(x,y)$と表すことにする\footnote{標準的には$\left\langle x,y\right\rangle$である.}.
順序対は
\begin{equation}
(x_0,x_1) = (y_0,y_1) \iff x_0 = y_0 \land x_1 = y_1
\end{equation}
という式によって特徴付けられる.
組は$(x_0,x_1,\cdots,x_n)$と表すことにする\footnote{これも標準的には$\left\langle x_0,x_1,\cdots,x_n\right\rangle$である.}.
順序対と同様に
\begin{equation}
(x_0,x_1,\cdots,x_n) = (y_0,y_1,\cdots,x_n) \iff \foralldot{i\in\{0,1,\cdots,n\}}{x_i = y_i}
\end{equation}
となる.
$(x_0,x_1,\cdots,x_n)$は$(x_i)_{i\in\{0,1,\cdots,n\}}$とも書く.

積集合とはいくつかの集合の要素で組を作ったときの集合のことであり,
\begin{equation}
X_0\times X_1\times\cdots\times X_n\triangleq \{(x_0,x_1,\cdots,x_n)|\foralldot{i\in\{0,1,\cdots,n\}}{x_i\in X_i}\}
\end{equation}
と定義される.
$X_0\times X_1\times\cdots\times X_n$は$\prod_{i\in\{0,1,\cdots,n\}} X_i$とも書く.

標準射影$\pi_i:\left(\prod_{j\in\{0,1,\cdots,n\}} X_j\right)\rightarrow X_i$とは,組または対からその構成要素を取り出す演算であり,
\begin{equation}
\pi_i(X_0\times X_1\times\cdots\times X_n)\triangleq X_i
\end{equation}
である.

順序対の集合$X$\jpdash 例えば,$X=\{(x,y)|x\in\R,y\in\Z\}$など\jpdash は二項関係とも呼ばれ,
その始域$\domain X$,終域$\codomain X$は夫々,
\begin{align}
    \domain X &\triangleq \{\pi_0 x|x\in X\}\\
    \codomain X &\triangleq \{\pi_1 x|x\in X\}
\end{align}
と定義される.
二項関係$R$に対して,始域を集合$C$に限定した二項関係$\{x|x\in R, \pi_0 x\in C\}$を
$\domainrestrict{R}{C}$と書く.
二項関係への追加$R\uplus\mypair{x}{y}$は,
$R\uplus\mypair{x}{y}\triangleq \domainrestrict{R}{\left(\domain R\setminus \{x\}\right)}$
と定義される.

$x_0,x_1,\cdots,x_n$ ($n\in\N$)という並びをを$\vec{x}$と略記する.



\section{論理について}\label{sc:about-logic}
真・偽を表すの記号として夫々,$\ltrue$,$\lfalse$を用いる.
論理和・論理積・否定・含意の記号として夫々,$\lor$,$\land$,$\lnot$,$\limpl$を用いる.
変数$x$がその領域$\varrange{x}$を動くとき,
述語$P(x)$が全ての$x$で真になることを$\foralldot{x}{P(x)}$,
述語$P(x)$がある$x$で真になることを$\existsdot{x}{P(x)}$と表す.
$\foralldot{x}{\foralldot{y}{\cdots}}$は$\foralldot{x,y,\cdots}{}$と略記する.
$\exists$も同様.

$\forall$,$\exists$,$\lnot$,$\limpl$は右結合性,
例えば,$\lnot\lnot A\equiv \lnot(\lnot A)$,$A\limpl B\limpl C \equiv A\limpl (B\limpl C)$とする.
夫々の演算子の優先順位は$\lnot\gtrdotrel\land,\lor\gtrdotrel\limpl$となる.

\begin{equation}\label{eq:def-mp}
\infer[\rulename{M.P.}]{\beta}{\alpha & \alpha \limpl \beta}
\end{equation}
は,$\alpha$と$\alpha \limpl \beta$という仮定から,\rulename{M.P.}という推論規則によって$\beta$が導き出されることを表す記法である.
\ref{eq:def-mp}によって定義される推論規則は,推論規則内の$\alpha$,$\beta$に式を代入することによって
\begin{equation}\label{eq:def-mp-instance-1}
\infer[\rulename{M.P.}]{A\land B}{A & A \limpl A\land B}
\end{equation}
や,
\begin{equation}\label{eq:def-mp-instance-2}
\infer[\rulename{M.P.}]{\foralldot{x}{Q(a, x) \limpl P(x)}}{\existsdot{x}{P(x)} & \existsdot{x}{P(x)} \limpl \foralldot{x}{Q(a, x) \limpl P(x)}}
\end{equation}
も表す.
この例における$\alpha$,$\beta$のように,超文脈的に扱われる変数のことをメタ変数(meta-variable)と呼ぶ.


$\Gamma_1,\Gamma_2,\cdots\vdash A_1, A_2,\cdots$は,
主に,$\Gamma_1, \Gamma_2, \cdots$から$A_1, A_2, \cdots$が推論できることを表す記法として用いる.


\section{バッカス・ナウア記法について}\label{sc:about-BNF-form}
バッカス・ナウア記法(Backus-Naur Form, BNF)
は文脈自由文法を定義するのに用いられるメタ言語であって,
プログラミング言語の構文記述やデータ構造の記述など,広汎に利用される.
計算機科学でも式として頻繁に用いられるので解説する.

$\Coloneqq$の左辺にある名前は右辺にある式で定義される.
例えば,
\begin{equation}\label{eq:bnf-example-1} 
e\Coloneqq \mathsf{abc}
\end{equation}
というBNFに対しては
\begin{equation}
\varrange{e} = \{\mathsf{abc}\}
\end{equation}
となる.

右辺において$\mid$は「または」の意味を表し,$\mid$で区切られたいずれかの式に当てはまる.例えば,
\begin{equation}\label{eq:bnf-example-2} 
e\Coloneqq \mathsf{a} \mid \mathsf{b} \mid \mathsf{c}
\quad\text{または}\quad
\begin{array}{rcl}
    e    &\Coloneqq& \mathsf{a} \\
            &\mid&   \mathsf{b} \\
            &\mid&   \mathsf{c} 
\end{array}
\end{equation}
というBNFに対しては
\begin{equation}
\varrange{e} = \{\mathrm{a}, \mathrm{b}, \mathrm{c}\}
\end{equation}
となる.

右辺において左辺の記号が使われた場合は再帰となる
\begin{equation}\label{eq:bnf-example-3} 
\begin{array}{rcl}
    e    &\Coloneqq& \mathsf{a} \\
            &\mid&   e\mathsf{'} 
\end{array}
\end{equation}
というBNFに対しては
\begin{equation}
\varrange{e} = \{\mathrm{a}, \mathrm{a'}, \mathrm{a''}, \mathrm{a'''}, \cdots \}
\end{equation}
となる.


\section{ラムダ式について}\label{sc:about-lambda-expression}
ラムダ式(lambda expression,$\lambda$-expression) とは,一言で言ってしまうと関数そのものを抽象化したものである.

例えば以下の式を考えよう.
\begin{align}
f(x) &= (x + 1)^2\\
f(x + 1) &= {\left((x + 1) + 1\right)}^2 = x^2 + 4x + 4\\
(x + 1) &= x + 1
\end{align}
これは$(x + 1)$に対して関数$f$を適用すると$x^2 + 4x + 4$となるが,
関数を適用しないとそれは$x + 1$そのものであることを表している.

では逆に,$f$に引数を適用しないとどうなるのであろうか.
\begin{align}
f(x) &= x^2 + 2x + 1\\
f    &= \;?
\end{align}
この場合の$f$そのものを表すのにラムダ式を用いることが出来,そうすると
\begin{align}
f &= \lambdadot{x}{(x^2 + 2x + 1)}
\end{align}
となって関数そのものを陽に扱うことが出来る.
以下,必要の無い限り括弧は省略し,$\lambdadot{x}{(x^2 + 2x + 1)}$は$\lambdadot{x}{x^2 + 2x + 1}$と書く.

ラムダ式を用いて記述した関数に引数を適用させると,
\begin{align}
f(y) &= (\lambdadot{x}{x^2 + 2x + 1})(y) = y^2 + 2y + 2 \\
f(2) &= (\lambdadot{x}{x^2 + 2x + 1})(2) = 2^2 + 2\cdot 2 + 1 = 9
\end{align}
となる.

次に複数引数の関数を考えよう.
\begin{align}
g(x, y) &= x + y
\end{align}
この関数を表すためには,
$\lambdadot{x,y}{}$という記法を導入するのが良いと思われる.
従って,
\begin{align}
\label{eq:lambda-multiple}
g &= \lambdadot{x,y}{x + y}\\
\label{eq:lambda-multiple-inst}
g(a, b) &= a + b\\
g(1, 2) &= (\lambdadot{x,y}{x + y})(1, 2) = 1 + 2 = 3
\end{align}
と表せる.

ここで,天下り的ではあるが,$\lambdadot{x}{\lambdadot{y}{x+y}}$を考える.
すると,
\begin{gather}
\begin{aligned}
\label{eq:lambda-curried}
h &= \lambdadot{x}{\lambdadot{y}{x+y}} \\
\end{aligned} \\
\begin{aligned}
\label{eq:lambda-curried-partial}
h(a) &= (\lambdadot{x}{\lambdadot{y}{x+y}})(a)\\
     &= {\lambdadot{y}{a+y}}
\end{aligned} \\
\begin{aligned}
\label{eq:lambda-curried-inst}
\left(h(a)\right)(b) &= \left((\lambdadot{x}{\lambdadot{y}{x+y}})(a)\right)(b) \\
        &= (\lambdadot{y}{a+y})(b)\\
        &= a + b
\end{aligned} \\
\begin{aligned}
\left(h(1)\right)(2) &= \left((\lambdadot{x}{\lambdadot{y}{x+y}})(1)\right)(2) \\
        &= (\lambdadot{y}{1+y})(2) \\
        &= 1 + 2 = 3
\end{aligned}
\end{gather}
このようになる.
以下,$f(a)(b)(c)\cdots$は
$\delimitershortfall-1pt\left({\cdots\left({\left({\left({f(a)}\right)(b)}\right)(c)}\right)\cdots}\right)$
を表すものとし,
必要の無い限り括弧を省略する.

ここで,
\refEq{eq:lambda-multiple-inst}と\refEq{eq:lambda-curried-inst}
を見比べると$g(a, b) = h(a)(b)$であることがわかる.
つまり,
\begin{equation}
(\lambdadot{x,y}{\cdots})(a,b) \equiv (\lambdadot{x}{\lambdadot{y}{\cdots}})(a)(b)
\end{equation}
といえる.

これは簡単に拡張が出来て,任意の多変数関数$g$に対してある関数$h$があって
\begin{equation}
g(a_1, a_2, \cdots, a_n)= h(a_1)(a_2)\cdots(a_n)
\end{equation}
となる.
この操作$\mathrm{curry}: g\mapsto h$をカリー化(currying)と呼ぶ.
纏めると,カリー化を行うことにより全ての多変数関数は一変数関数で書くことが出来ることとなる.

このことより,
$\lambdadot{x}{\lambdadot{y}{\lambdadot{z}{\cdots}}}$の略記として
$\lambdadot{x,y,z,\cdots}{}$または
$\lambdadot{xyz\cdots}{}$と書き,
$f(a)(b)(c)\cdots$を
$f(a,b,c\cdots)$と書くことがある.

ここで,二変数の関数に一つだけ引数を与えた\refEq{eq:lambda-curried-partial}のような場合を考える.
\begin{align}
f (x)(y) &= x + y \quad \left(f = \lambdadot{xy}{x + y}\right) \\
\mathit{add3}(x)   &= 3 + x \quad \left(\mathit{add3} = \lambdadot{x}{3 + x}\right) 
\end{align}
なる$f$,$\mathit{add3}$を考えたときに
\begin{align}
\begin{aligned}
f(3)            &= (\lambdadot{x}{\lambdadot{y}{x+y}})(3) \\
                &= \lambdadot{y}{3 + y} \\
                &= \lambdadot{x}{3 + x} \\
                &= \mathit{add3}
\end{aligned}
\end{align}
となって,二つの引数を持つ関数$f$に引数を一つ適用した関数$f(3)$と
常識的な方法で定義した引数に3を加える関数$\mathit{add3}$が等しくなる.
このように$n$の引数を取る関数に対して$m$ ($<n$)の引数を適用することを部分適用(partial application)と呼ぶ.

以下,関数の引数を表す括弧は省略し,
$f(x)(y)(z)\cdots$は$f\ x\ y\ z\ \cdots$または$f x y z \cdots$と書く.


% \section{ラムダ計算}\label{sc:about-lambda-calculus}

% \section{型付ラムダ計算}\label{sc:about-typed-lambda-calculus}

\section{型について}\label{sc:about-type}

ある式$e$が型$t$を持つことを
\begin{equation}
e: t
\end{equation}
と書く.

例えば,真偽値をとる変数$b$があったとき,
その型は真偽値型(\textsf{bool})であると定めることが出来る.

関数$f$と型$t$の式$e:t$があって,$e$に$f$を適用した$f\ e$の型が$s$であるとき,
$f$の型は$t\rightarrow s$であるとする.



\chapter{対象言語の設計}\label{ch:lang_design}

\begin{abstract}
この章では,
まずnibkame言語のコア部分の仕様を定義し,
後に,その拡張としてnibkame言語本体とライブラリの仕様について述べる.
そして,最終的に実装された言語機能と構文を示す.
\end{abstract}

\section{���̍\��}\label{sc:expr-syntax}
nibkame�R�A����ɂ����鎮�̒�`�͈ȉ��̗l�ɂȂ�D
\begin{subequations}
\begin{equation}\label{eq:def-expr} 
\begin{array}{rcll}
e&\Coloneqq& c                          & \text{�萔 (\refSsc{ssc:expr-constant})} \\
 &\mid&  s:\tau                     & \text{�O���萔 (\refSsc{ssc:expr-external})} \\
 &\mid&  v                          & \text{�ϐ� (\refSsc{ssc:expr-variable})} \\
 &\mid&  g                          & \text{�^�\�z�q (\refSsc{ssc:expr-constructor})} \\
 &\mid&  \fundot{x}{e}           & \text{�֐� (\refSsc{ssc:expr-function})} \\
 &\mid&  e, \cdots, e               & \text{�g (\refSsc{ssc:expr-tuple})} \\
 &\mid&  \mathnkop{ref}e            & \text{�Q�Ƃ̐��� (\refSsc{ssc:expr-reference})} \\
 &\mid&  e!                         & \text{�Q�� (\refSsc{ssc:expr-reference})} \\
 &\mid&  e\coloneqq e                & \text{��� (\refSsc{ssc:expr-reference})} \\
 &\mid&  e\ e                       & \text{�K�p (\refSsc{ssc:expr-apply})} \\
 &\mid&  \mathnkop{let}x=e\mathnkop{in}e & \text{���� (\refSsc{ssc:expr-let})} \\
 &\mid&  \fixdot{f}{\fundot{x}{e}} & \text{�s���_���Z�q (\refSsc{ssc:expr-fix})} \\
% &\mid&  \mathnkop{if}e\mathnkop{then}e\mathnkop{else}e & \text{���� (\refSsc{ssc:expr-if})} \\
 &\mid&  \mathnkop{match}e\mathnkop{with}\mathit{clss} & \text{�p�^�[���}�b�`���O (\refSsc{ssc:expr-match})} \\
 &\mid&  \mathnkop{match}\clauseor e\mathnkop{with}\mathit{clss} & \text{����D�ŏ���$\clauseor$�͖��������D} \\
\mathset{Expr}&\triangleq& \varrange{e}            & \text{��} \\
\mathset{ExprVar}&\triangleq& \varrange{x}            &  \\
\mathset{ExprConst}&\triangleq& \varrange{c}            &  \\
\end{array}
\end{equation}
\begin{equation}\label{eq:def-clause}
\begin{array}{rcll}
\mathit{cls} &\Coloneqq& p\rightarrow e & \\
             &\mid& p \mathnkop{when} e \rightarrow e & \text{�K�[�h�t} \\
\mathit{clss} &\Coloneqq& \mathit{cls} & \\
              &\mid& \mathit{clss}\clauseor\mathit{cls} & \\
\mathset{Clause}&\triangleq& \varrange{\mathit{cls}}            & \text{��} \\
\end{array}
\end{equation}
\begin{equation}\label{eq:def-pattern}
\begin{array}{rcll}
p&\Coloneqq&\patternany             & \text{���C���h�J�[�h} \\
 &\mid&  c                          & \text{�萔} \\
 &\mid&  x                          & \text{�ϐ�} \\
 &\mid&  g                          & \text{�^�\�z�q} \\
 &\mid&  g\ p                       & \text{�^�\�z�q�K�p} \\
 &\mid&  p,\cdots,p                 & \text{�g} \\
 &\mid&  p \patternand p            & \text{�A��} \\
 &\mid&  p \patternor p             & \text{�I��} \\
 &\mid&  \lnot p                    & \text{�ے�} \\
\mathset{Pattern}&\triangleq& \varrange{p}            & \text{�p�^�[��} \\
\mathset{Constructor}&\triangleq& \varrange{g}            & \text{�^�\�z�q} \\
\mathset{Var}&\triangleq&\mathset{ExprVar}\cup\mathset{Constructor}
\end{array}
\end{equation}
\end{subequations}
�����ŁC$\mathset{ExprVar}\cap\mathset{Constructor}=\emptyset$�ł���D

\subsection{�萔}\label{ssc:expr-constant}
�萔�Ƃ�
\begin{itemize}
    \item \lstinline|()|
    \item \lstinline|Nil|
    \item \lstinline|true|
    \item \lstinline|false|
    \item ����
    \item ���������_��
    \item ����
\end{itemize}
�̌܂‚ł���D

���̂����C\lstinline|()|��C����Ō����Ƃ����void�̂悤�Ȃ��̂ŁC
�֐������������Ȃ��Ⴕ���͕Ԃ��Ȃ����Ƃ𖾎�����̂ɗp����D
\lstinline|Nil|��C����Ō����Ƃ����NULL�̂悤�Ȃ��̂ŁC
���X�g�̖����ł��邱�Ƃ������D

���ɂ‚��Ă����‚�����������\refTbl{tbl:int-float}�̂悤�ɂȂ�D

\begin{table}[h]
    \caption{�����ƕ��������_��}\label{tbl:int-float}
    \begin{center}
    \begin{tabular}{crl}
    \Hline
    �����̗�&\lstinline|0|,
             \lstinline|1|,
             \lstinline|100|,
             \lstinline|-1| \\
    \hline
    ���������_���̗� &\lstinline|0.0|,
             \lstinline|1.0|,
             \lstinline|-1.0|,
             \lstinline|1.0e10|,
             \lstinline|1.0e-10|\\
    \Hline
    \end{tabular}
    \end{center}
\end{table}

nibkame�R�A����ɂ����ėp�ӂ���Ă���g�ݍ��݂̕����^��
char�^�ł��邪�C���̌^��%C����Ɠ��l��
8�r�b�g�̕������������ł���D
%char�^�ł���C�v�Z�@��ł̕\���͏����n�Ɉˑ�����D
�����̃R�[�h��ł̕\����\lstinline|'c'|�̂悤�ɂȂ�D

\subsection{�O���萔}\label{ssc:expr-external}
�O���萔�́Cnibkame�R�A����\�[�X�R�[�h��Ŋ��ɒ�`����Ă�����̂Ƃ��Ďg�p���邱�Ƃ̏o����
�萔�ł���C��̓I�ɂ͊O����C�����^�C���̊֐�����Z�E���Z�Ȃǂ̃v���~�e�B�u�֐��ł���D

\subsection{�g}\label{ssc:expr-tuple}
�g�Ƃ͂����‚��̒l�̕��т̂��Ƃł���C
�W���ɂ�����g�Ƃقړ������̂ł���D

\subsection{�ϐ�}\label{ssc:expr-variable}
�ϐ��͈�ʂ̃v���O���~���O����ɂ�����T�O�Ɠ����ł���D
�������Cnibkame�ɂ�����ϐ��͐������ɍs���鑩���̌�͈�؂̑�����s�����Ƃ��o���Ȃ��D
���̂��߁C�Q�Ƃ��p�����Ă��Ȃ��ꍇ�ɂ́C��������]������Ɠ������ʂƂȂ�D

\subsection{����}\label{ssc:expr-let}
�ϐ��������s���̂�$\mathnkop{let}$���ł���D
nibkame�ɂ�����ϐ��̃X�R�[�v�̓��L�V�J���X�R�[�v(lexical scope)�ł���D

�Ⴆ�΁C
\begin{equation}
\begin{aligned}
&\mathnkop{let} r = \\
&\quad \mathnkop{let} x = 5 \mathnkop{in} \\
&\quad \mathnkop{let} a = x \mathnkop{in} \\
&\quad \mathnkop{let} x = 3 \mathnkop{in} \\
&\quad \mathnkop{let} b = x \mathnkop{in} \\
&\quad\quad a, b, x
\end{aligned}
\end{equation}
�Ƃ����R�[�h�ł�
$r$��$5, 3, 3$�ƂȂ�D

\subsection{�Q��}\label{ssc:expr-reference}
nibkame�ɂ����Ă͈�x���������ϐ��̒��g��ύX���邱�Ƃ��o���Ȃ��D
�������C���ۂɕ���p�̈�ؖ����R�[�h�������͎̂�ԂȂ̂ŁC
�Q�ƂƂ����T�O�𓱓����C
�ϐ��Ɋi�[����Ă���u�f�[�^���i�[����Ă���ꏊ�ւ̎Q�Ɓv�͕ω����Ȃ����C
�u�f�[�^���i�[����Ă���ꏊ�ւ̎Q�Ɓv�����ǂ��āC�i�[����Ă���f�[�^�𓾂���C�ύX�����肷�邱�Ƃ��o����悤�ɂȂ��Ă���D

�Ⴆ�΁C
\begin{equation}
\begin{aligned}
&\mathnkop{let} p = \mathnkop{ref} 3 \\
&\mathnkop{let} q = \\
&\quad \mathnkop{let} x = p \coloneqq 7 \mathnkop{in} \\
&\quad\quad p!
\end{aligned}
\end{equation}
�Ƃ����R�[�h�͏ォ�珇�Ɏ��s����C
�ŏI�I��\lstinline|q|��\lstinline|7|�C\lstinline|p|��\lstinline|7|���i�[����Ă���ꏊ�ւ̎Q�Ƃ���������Ă��邱�ƂȂ�D

\subsection{�^�\�z�q}\label{ssc:expr-constructor}
�^�\�z�q�Ƃ͂�����㐔�I�f�[�^�\���������̂ł���D
�W���Ō����Ƃ���̒��a�Ɠ������D

\subsection{�p�^�[���}�b�`���O}\label{ssc:expr-match}
�p�^�[���}�b�`���O�́C����������s������C
�^�\�z�q�ɂ���č��ꂽ�㐔�I�f�[�^�^�̒l����C
���g�����o���ĕϐ��ɑ���������ł���D

$\mathnkop{match}e\mathnkop{with}\mathit{clss}$
�Ƃ������̏ꍇ���ƁC
��$p\mathnkop{with}e_g\rightarrow e_c$��擪���珇��
�p�^�[��$p$��$e$�̒l�ɓ��Ă͂߂邱�Ƃ��ł��邩�ǂ����ƁC�K�[�h$e_g$�̒l��$\mathnkval{true}$�ǂ����𒲂ׁC
������������ꍇ�͎�$e_c$��]�����C���̒l��match���̒l�ƂȂ�D
�������C�K�[�h�̂Ȃ���$p\rightarrow e_c$��$p\mathnkop{with}\mathnkval{true}\rightarrow e_c$�Ɠ����ł���D

\paragraph{�萔}
�萔�p�^�[���͒萔�ɂ̂݃}�b�`����D���Ƃ��΁C
\begin{equation}
\begin{aligned}
&\mathnkop{match}\text{\lstinline|'b'|}\mathnkop{with} \\
&\quad \clauseor \text{\lstinline|'a'|}\rightarrow 1 \\
&\quad \clauseor \text{\lstinline|'b'|}\rightarrow 2 \\
&\quad \clauseor \text{\lstinline|'c'|}\rightarrow 3 \\
&\quad \clauseor \text{\lstinline|'d'|}\rightarrow 4
\end{aligned}
\end{equation}
�Ƃ����R�[�h�ɂ�����match���̒l��$2$�ƂȂ�D

\paragraph{���C���h�J�[�h}
���C���h�J�[�h�p�^�[���͌^�\�z�q�������S�ĂɃ}�b�`����D���Ƃ��΁C
\begin{equation}
\begin{aligned}
&\mathnkop{match}e\mathnkop{with} \\
&\quad \clauseor \patternany \rightarrow 1
\end{aligned}
\end{equation}
�Ƃ����R�[�h�ɂ�����match���̒l�́C
�ϐ�$e$�̒l���Ȃ�ł��낤�Ə��$1$�ł���D

�܂��C�p�^�[���̕]���������C
\begin{equation}
\begin{aligned}
&\mathnkop{match}\text{\lstinline|'d'|}\mathnkop{with} \\
&\quad \clauseor \text{\lstinline|'a'|}\rightarrow 1 \\
&\quad \clauseor \text{\lstinline|'b'|}\rightarrow 2 \\
&\quad \clauseor \patternany \rightarrow 3 \\
&\quad \clauseor \text{\lstinline|'d'|}\rightarrow 4
\end{aligned}
\end{equation}
�Ƃ����R�[�h�ɂ�����match���̒l�́C
���C���h�J�[�h�p�^�[���Ƀ}�b�`���邽��$3$�ƂȂ�D

\paragraph{�ϐ�}
�ϐ��p�^�[���̓��C���h�J�[�h�p�^�[���Ƃقړ����ł���C�^�\�z�q�������S�ĂɃ}�b�`����D
���C���h�J�[�h�p�^�[���Ƃ̈Ⴂ�́C
�ϐ��p�^�[��$x$�ɒl$e$�����Ă͂܂����Ƃ��C
�ϐ�$x$�ɒl$e$���������ꂽ��Ԃ�
�߂̃K�[�h$e_g$�Ǝ�$e_c$���]������邱�Ƃł���D

���Ƃ��΁C
\begin{equation}
\begin{aligned}
&\mathnkop{match}\text{\lstinline|'a'|}\mathnkop{with} \\
&\quad \clauseor x\rightarrow 1,x \\
\end{aligned}
\end{equation}
�Ƃ����R�[�h�ɂ�����match���̒l�́C
$1,\text{\lstinline|a|}$�ƂȂ�D

\paragraph{�^�\�z�q�E�^�\�z�q�K�p}
�^�\�z�q�p�^�[���ƌ^�\�z�q�K�p�p�^�[���͑㐔�I�f�[�^�^�ɑ΂��ēK�p�����D
�^�\�z�q�̌^�ɂ���Ăǂ��炪�g���Ȃ��Ă͂����Ȃ��������܂�C
���������Ȃ��^�\�z�q�ɑ΂��Ă͌^�\�z�q�p�^�[���C
���������^�\�z�q�ɑ΂��Ă͌^�\�z�q�K�p�p�^�[�����g����D

���Ƃ��΁C
\begin{equation}
\begin{aligned}
\mathnkvariant{Left} : \typename{int}\rightarrow \typename{intEither}\\
\mathnkvariant{Right} : \typename{int}\rightarrow \typename{intEither}
\end{aligned}
\end{equation}
�Ƃ����^�\�z�q���������ꍇ�C
\begin{equation}
\begin{aligned}
&\mathnkop{match}\mathnkvariant{Left}\ 2\mathnkop{with} \\
&\quad \clauseor \mathnkvariant{Left}\ x\rightarrow 1,x \\
&\quad \clauseor \mathnkvariant{Left}\ 2\rightarrow 2,x \\
&\quad \clauseor \mathnkvariant{Right}\ x\rightarrow 3,x
\end{aligned}
\end{equation}
�Ƃ����R�[�h�ɂ�����match���̒l��$2,2$�ƂȂ�D

�܂��C
\begin{equation}
\begin{aligned}
\mathnkvariant{MyTrue} : \typename{mybool}\\
\mathnkvariant{MyFalse} : \typename{mybool}
\end{aligned}
\end{equation}
�Ƃ����^�\�z�q���������ꍇ�C
\begin{equation}
\begin{aligned}
&\mathnkop{match}\mathnkvariant{MyTrue}\mathnkop{with} \\
&\quad \clauseor \mathnkvariant{MyTrue}\rightarrow 1 \\
&\quad \clauseor \mathnkvariant{MyFalse}\rightarrow 0 \\
\end{aligned}
\end{equation}
�Ƃ����R�[�h�ɂ�����match���̒l��$1$�ƂȂ�D

\paragraph{�g}
�g�p�^�[���͑g�ɑ΂��ēK�p�����D
�g�̐��͈�v���Ă���K�v������D

���Ƃ��΁C
\begin{equation}
\begin{aligned}
&\mathnkop{match}1,2,\mathnkval{false}\mathnkop{with} \\
&\quad \clauseor x,y,\mathnkval{true}\rightarrow x,y \\
&\quad \clauseor x,y,\mathnkval{false}\rightarrow y,x \\
\end{aligned}
\end{equation}
�Ƃ����R�[�h�ɂ�����match���̒l��$2,1$�ƂȂ�D

\paragraph{�A��}
�A���p�^�[��$p_1\patternand \cdots\patternand p_n$��
�S�Ẵp�^�[��$p_i$���}�b�`�����Ƃ��̂݃}�b�`����D
�������C�e�p�^�[���ɏo������ϐ��p�^�[���̖��O�́C���̃p�^�[���ɏo�����Ă͂����Ȃ��D

���Ƃ��΁C
\begin{equation}
\begin{aligned}
&\mathnkop{match}1,(2,\mathnkval{false})\mathnkop{with} \\
&\quad \clauseor x,((1, z)\patternand(y,(b\patternand\mathnkval{false})))\rightarrow x,b,z \\
&\quad \clauseor x,((2, c)\patternand(y,\mathnkval{false}))\rightarrow x,y,c \\
\end{aligned}
\end{equation}
�Ƃ����R�[�h�ɂ�����match���̒l��$1,2,\mathnkval{false}$�ƂȂ�D

\paragraph{�I��}
�I���p�^�[��$p_1\patternor \cdots\patternor p_n$��
�S�Ẵp�^�[��$p_i$���}�b�`�����Ƃ��̂݃}�b�`����D
�������C�e�p�^�[���ɏo������ϐ��p�^�[���̖��O�ƌ^�̏W��
�͑S�ē����łȂ���΂����Ȃ��D

���Ƃ��΁C
\begin{equation}
\begin{aligned}
&\mathnkop{match}1,(2,\mathnkval{false})\mathnkop{with} \\
&\quad \clauseor x,((1, z)\patternor(\patternany,(z\patternand\mathnkval{true})))\rightarrow z,x \\
&\quad \clauseor x,((1, z)\patternor(\patternany,(z\patternand\mathnkval{false})))\rightarrow x,z \\
\end{aligned}
\end{equation}
�Ƃ����R�[�h�ɂ�����match���̒l��$1,\mathnkval{false}$�ƂȂ�D

\paragraph{�ے�}
�ے�p�^�[��$\lnot p$�̓p�^�[��$p$�����̈ʒu�Ń}�b�`���Ȃ��������ɁC�}�b�`����D
�������C�p�^�[��$p$�̒��ɕϐ��p�^�[�����܂�ł͂Ȃ�Ȃ��D

���Ƃ��΁C
\begin{equation}
\begin{aligned}
&\mathnkop{match}1\mathnkop{with} \\
&\quad \clauseor \lnot 1\rightarrow \mathnkval{true} \\
&\quad \clauseor \lnot 2\rightarrow \mathnkval{false} \\
\end{aligned}
\end{equation}
�Ƃ����R�[�h�ɂ�����match���̒l��$1,\mathnkval{false}$�ƂȂ�D
�܂��C�ے�p�^�[���͘A���E�I���p�^�[���ƂƂ��Ɏg�p����ƕ֗��ł���C
\begin{equation}
\begin{aligned}
&\mathnkop{match}1,(2,\mathnkval{false})\mathnkop{with} \\
&\quad \clauseor x,((\lnot 1, z)\patternand(y,(b\patternand\mathnkval{false})))\rightarrow x,b,z \\
&\quad \clauseor x,((2, c)\patternand(y,\mathnkval{false}))\rightarrow x,y,c \\
\end{aligned}
\end{equation}
�Ƃ����R�[�h�ɂ�����match���̒l��$1,\mathnkval{false},2$�ƂȂ�D

\subsection{�֐�}\label{ssc:expr-function}
�֐��͂���l���Ƃ��Ă���l��Ԃ��l�ł���D
nibkame�ɂ����Ċ֐��͑�ꋉ�̃I�u�W�F�N�g(first-class object)�Ȃ̂ŁC
�ϐ��ɑ���������C�֐��̕Ԃ�l�Ƃ��ĕԂ����肷�邱�Ƃ��ł���D

\subsection{�s���_���Z�q}\label{ssc:expr-fix}
nibkame�R�A����ɂ����āC
$\mathnkop{let} x = e_1 \mathnkop{in} e_2$�Ƃ��������̎��ɂ�����
�ϐ�$x$�̒�`$e_1$�̒���$x$���g�p���邱�Ƃ��ł��Ȃ��D
����āC�ċA�֐����`���邽�߂ɕs���_���Z�q(�s���_�R���r�l�[�^�Ƃ�)��p����D
�s���_���Z�q$\mathnkop{fix}$�͈ȉ��̓��������D
\begin{equation}
\mathnkop{fix}f= f(\mathnkop{fix}f)
\end{equation}

\subsection{�K�p}\label{ssc:expr-apply}
�֐��ɒl��^���l�𓾂邱�Ƃł���D
nibkame�ɂ����ĕ]���헪(evaluation strategy)��
�l�n��(call-by-value)�ł���D
���̂��߁C�֐��ɓn���������͎��O�ɕ]������Ă���n����邱�ƂɂȂ�D


\section{型}\label{sc:type}
nibkameでは型システムとして,Hindley-Milner型システムを採用した.

Hindley-Milner型システムとはHaskellやMLのような静的型付け関数プログラミング言語によく用いられる型システムであり,
「すべての式に一意な最汎(most general)型,即ち主型(principal type)を付けることができる」という重要な性質を持つ.

代表的なアルゴリズムは「\algorithmW」\cite{Milner1978348}\cite{Damas:1982:PTF:582153.582176}として知られているものである.
このアルゴリズムはHindley-MilnerアルゴリズムともDamas-Milnerアルゴリズムとも呼ばれる.
型変数への代入による型の同一化(unification)を基本操作としている.

このアルゴリズムの起源は,Haskell CurryとRobert Feysが1958年に単純型付ラムダ計算の型推論のために考案したものであり,
1969年にJ. Roger Hindleyがそれを拡張し,また,それが常に最汎な型に推論することを証明した.
1978年にRobin MilnerはHindleyとは独立に\algorithmW\cite{Milner1978348}を導き出し,
1982年,Luis Damasが最終的にMilnerのアルゴリズムの完全性を証明した\cite{Damas:1982:PTF:582153.582176}.

今現在,
Hindley-Milner型システムを採用しているnibkameでは型は以下の様に定義される.
\begin{equation}\label{eq:type} 
\begin{array}{rcll}
    \tau    &\Coloneqq& \iota                       & \text{型定数 (\refSsc{ssc:type-constant})} \\
            &\mid&  \alpha                          & \text{型変数 (\refSsc{ssc:type-variable})} \\
            &\mid&  \tau \rightarrow \tau           & \text{関数型 (\refSsc{ssc:type-function})} \\
            &\mid&  \tau \times \cdots \times \tau  & \text{直積型 (\refSsc{ssc:type-tuple})} \\
            &\mid&  \tau\ \textsf{ref}              & \text{参照型 (\refSsc{ssc:type-reference})} \\
            &\mid&  \tau\ \tau                      & \text{型適用 (\refSsc{ssc:type-apply})} \\
    \mathset{Type}&\triangleq& \varrange{\tau}            & \text{型} \\
    \sigma  &\Coloneqq &\tau                            & \text{型} \\
            &\mid&  \foralldot{\vec{\alpha}}{\tau}      & \text{全称量化}\\
    \mathset{TypeScheme}&\triangleq& \varrange{\sigma}            & \text{型スキーム}\\
    \mathset{TypeVar}&\triangleq& \varrange{\alpha}            & 
\end{array}
\end{equation}

\subsection{型定数}\label{ssc:type-constant}
nibkameにおいて型定数は次の様に定義される.
\begin{equation}\label{eq:type-constant} 
\begin{array}{rcll}
    \iota   &\Coloneqq& \typename{int}                  & \text{整数型} \\
            &\mid&  \typename{float}                & \text{実数型} \\
            &\mid&  \typename{char}                 & \text{char型} \\
            &\mid&  \typename{list}                 & \text{リスト型} \\
            &\mid&  \typename{array}                & \text{配列型} \\
            &\mid&  \mathit{variant}                & \text{ヴァリアント型} \\
    \mathset{TypeConstant}&\triangleq& \varrange{\iota}            & \text{型定数}\\
    \mathset{TypeVariant}&\triangleq& \varrange{\mathit{variant}}  & \text{ヴァリアント型}
\end{array}
\end{equation}

以上のうち,\typename{int}と\typename{float}は数値を表す型であり,
内部的には夫々マシンにおけるもっとも自然な整数を表す表現\jpdash 例えばレジスタによる整数表現\jpdash と,
もっとも自然な浮動小数点数を表す表現\jpdash 例えばIEEEの浮動小数点数\jpdash が用いられる.
\typename{char}は1バイトによって表現できる符号なし数
\jpdash 典型的には$0$から$2^8-1$\jpdash 
を表す型である.

$\typename{list}$と$\typename{array}$,ヴァリアント型が型定数となっているのに違和感を覚える向きもあるだろうが理由は
\refSsc{ssc:type-apply}で詳説する.

\subsection{型変数}\label{ssc:type-variable}
型変数とは型を動く変数である.

\subsection{関数型}\label{ssc:type-function}
関数型とはその名の通り関数が持つ型であり,
$s$型をとって$t$型を返す関数の型は$s\rightarrow t$となる.

\subsection{直積型}\label{ssc:type-tuple}
直積型は組の型であり,組の夫々の要素の型の組となっている.

\subsection{参照型}\label{ssc:type-reference}
参照が持つ型であり,例えば,\lstinline|int|型への参照であれば,その型は
\lstinline|int ref|となる.

\subsection{型適用}\label{ssc:type-apply}
ML系言語の型には,
$\typename{int}\ \typename{list}$や,
$(\typename{int}\times\typename{char}\ \typename{array})\ \typename{list}$
等の型が存在する
\footnote{
    Haskell,Mirandaではそれぞれ
    $[\typename{Int}]$,
    $[(\typename{Int}, \typename{Array}\ \typename{Char})]$
    となる.
    }
.
これは夫々整数型のリスト,整数と(char型の配列)の組のリストという意味である.

上記の例よりわかるとおり,\typename{list}や\typename{array}などは型を引数として取る型である.
それを実現するために,それらを型を取って型を返す関数の様に扱う.
但し,ML系関数の記法と合わせるために,関数と引数の並びが通常と逆になる.

\subsection{ヴァリアント型}\label{ssc:type-variant}
代数的データ型ともいう.これはユーザが新しく定義できる型である.ヴァリア
ントは定義のたびにまったく独立した新しい型となる.
ヴァリアント型の値は型構築子によって生成される.

たとえば,
\begin{lstlisting}
type season = Spring | Summer | Fall | Winter
\end{lstlisting}
とすれば
\lstinline|Spring|,\lstinline|Summer|,\lstinline|Fall|,\lstinline|Winter|
を型構築子とするヴァリアント型\lstinline|season|が定義される.

また各型構築子は値引数を取ることができ,ヴァリアント型は型引数を取ること
ができる.これを用いて
\begin{lstlisting}
type 'a tree = Node of 'a tree * 'a * 'a tree | Leaf of 'a
\end{lstlisting}
のように二分木などを定義することができる.

\subsection{実例}\label{ssc:example}
1と3.14,'c'はそれぞれ定数型
\typename{int},\typename{float},\typename{char}である.

整数を1つ取り,後続の整数を返す関数add1を考える.するとこの型は
\lstinline|int -> int|となる.

任意の要素が2つの組を引数に取り,1つめの値を返す関数fstを考える.引数に
は任意の型を取れるから,\lstinline|(int, char) -> int|,
\lstinline|(float array, int) -> float array|などが考えられる.もっとも
一般的な型は\lstinline|('a, 'b) -> 'a|である.


\subsection{型変数への代入}\label{ssc:type-order}
$\tau$に含まれるいくつかの型変数に対して置換$\substType$を適用することで,
$\Sigma\tau=\tau'$となるとき,
$\tau\succ\tau'$とかく.
この$\succ$は半順序となる.
$\tau\succ\tau'$かつ$\tau'\succ\tau$のとき型$\tau$,$\tau'$は$\alpha$等値といい,
$\tau\mathrel{\overset{\alpha}{=}}\tau'$と表す.





\section{意味論}\label{sc:semantics}
まず,変数や型変数,型構築子の型の環境は次の様に定義される.
\begin{equation}\label{eq:env} 
\begin{array}{rcll}
    \envExpr  &\subset& \mathset{Var}\times\mathset{TypeScheme} & \text{変数の型環境} \\
    \envType  &\subset& \mathset{TypeVar}\times\mathset{TypeScheme} & \text{型変数の型環境} \\
    \envVariant  &\subset& \mathset{TypeVariant}\times\powerset(\mathset{Constructor}\times\mathset{TypeScheme}) & \text{ヴァリアントの型環境} \\
\end{array}
\end{equation}
%置換の集合は次のように定義される.
%\begin{equation}\label{eq:subst} 
%\begin{array}{rcll}
%    \substExpr  &\subset& \mathset{Var}\times\mathset{Expr} & \text{変数の置換} \\
%    \substType  &\subset& \mathset{TypeVar}\times\mathset{Type} & \text{型変数の置換} \\
%\end{array}
%\end{equation}
%置換の集合については特別に$\{(v,e),\cdots\}$を$[e/v,\cdots]$とも書く.
式$e$,型$\tau$の中に現れる自由変数の集合は夫々$\freevars{e}$,$\freetypevars{\tau}$と書く.

%式に対する置換の適用$\applysubst{\Sigma}{x}$は次のように定義される.
%\begin{equation}
%\begin{aligned}
%    \applysubst{\substExpr}{c} &= c \\
%    \applysubst{\substExpr}{s:\tau} &= s:\tau \\
%    \applysubst{\substExpr}{x} &=\begin{cases}
%        e & \text{if $(v,e)\in\substExpr$.} \\
%        x & \text{otherwise.} \end{cases} \\
%    \applysubst{\substExpr}{\fundot{x}{e}} &= \applysubst{\substExpr'}{e} &
%            \text{which $\substExpr' = \domainrestrict{\substExpr}{(\domain\substExpr\setminus\{x\})}$.}\\
%\end{aligned}
%\end{equation}

\subsection{�^�t��}\label{sc:typing}
�萔�̌^��Ԃ��֐�$\mathfunc{ConstType}:\mathset{ExprConst}\rightarrow\mathset{TypeConst}$�͎��̗l�ɒ�`�����D
\begin{equation}\label{eq:func-const-type} 
    \mathfunc{ConstType}(c) = \begin{cases}
        \typename{unit} & \text{if $c$ is $\mathnkop{()}$.} \\
        \typename{bool} & \text{if $c$ is $\mathnkop{true}$ or $\mathnkop{false}$.} \\
        \typename{int} & \text{if $c$ is integer.} \\
        \typename{float} & \text{if $c$ is float.} \\
        \typename{char} & \text{if $c$ is char.} \\
        \alpha\ \typename{list} & \text{if $c$ is $\mathnkop{Nil}$. And $\alpha$ is fresh.}
        \end{cases}
\end{equation}

�p�^�[���ɂ�鑩���̌^�‹�$\patternandenv{p:\tau}{\envPattern}$�͈ȉ��̗l�ɒ�`�����D
\begin{subequations}
\begin{equation}
    \infer[\rulename{P-All}]
        {\envExpr\vdash \patternandenv{\_:\tau}{\emptyset}}
        {}\\
\end{equation}
\begin{equation}
    \infer[\rulename{P-Not}]
        {\envExpr\vdash \patternandenv{\lnot p:\tau}{\emptyset}}
        {\envExpr\vdash \patternandenv{p:\tau}{\emptyset}}\\
\end{equation}
\begin{equation}
    \infer[\rulename{P-Const}]
        {\envExpr\vdash \patternandenv{c:\tau}{\emptyset}}
        {\mathfunc{ConstType}(c) \succ \tau} \\
\end{equation}
\begin{equation}
    \infer[\rulename{P-Var}]
        {\envExpr\vdash \patternandenv{x:\tau}{\{x:\tau\}}}
        {x\in\mathset{ExprVar}}\\
\end{equation}
\begin{equation}
    \infer[\rulename{P-Constructor}]
        {\envExpr\vdash \patternandenv{g:\tau}{\emptyset}}
        {g\in\mathset{Constructor} & (g,\tau)\in\envExpr & \tau_1\rightarrow\tau_2\not\succ\tau}\\
\end{equation}
\begin{equation}
    \infer[\rulename{P-Apply}]
        {\envExpr\vdash \patternandenv{g\ p:\tau}{\envPattern}}
        {g\in\mathset{Constructor} & (g,\tau'\rightarrow\tau )\in\envExpr & \envExpr\vdash \patternandenv{p:\tau'}{\envPattern} }\\
\end{equation}
\begin{equation}
    \infer[\rulename{P-Tuple}]
        {\envExpr\vdash \patternandenv{e_0,\cdots,e_n:\tau_0\times\cdots\times\tau_n}{\bigcup_i \envPattern_i}}
        {\envExpr\vdash\patternandenv{p_0:\tau_0}{\envPattern_0} &
        \cdots &
        \envExpr\vdash\patternandenv{p_n:\tau_n}{\envPattern_n} &
        \foralldot{i,j}{i\not= j\limpl (\domain\envPattern_i \cap \domain\envPattern_j = \emptyset)}}\\
\end{equation}
\begin{equation}
    \infer[\rulename{P-And}]
        {\envExpr\vdash \patternandenv{p_1 \patternand p_2:\tau}{\envPattern_1\cap\envPattern_2}}
        {\envExpr\vdash \patternandenv{p_1:\tau}{\envPattern_1} &
        \envExpr\vdash \patternandenv{p_2:\tau}{\envPattern_2} &
        \domain\envPattern_1 \cap \domain\envPattern_2 = \emptyset
        }\\
\end{equation}
\begin{equation}
    \infer[\rulename{P-Or}]
        {\envExpr\vdash \patternandenv{p_1 \patternor p_2:\tau}{\envPattern}}
        {\envExpr\vdash \patternandenv{p_1:\tau}{\envPattern} &
        \envExpr\vdash \patternandenv{p_2:\tau}{\envPattern}}
\end{equation}
\end{subequations}
���̎��Œ^�͈ȉ��̗l�ɒ�`�����D
�A���C�f��̖�������
%$i,n\in\N$�C
$c\in\mathset{ExprConst}$, 
$x,v\in\mathset{ExprVar}$�C
$e\in\mathset{Expr}$
�Ƃ���D
\begin{subequations}
\begin{equation}
    \infer[\rulename{E-Const}]
        {\envExpr\vdash c:\tau}
        {\mathfunc{ConstType}(c) \succ \tau} \\
\end{equation}
\begin{equation}
    \infer[\rulename{E-Var}]
        {\envExpr\vdash v:\tau}
        {(v,\tau')\in\envExpr & \tau' \succ \tau}\\
\end{equation}
\begin{equation}
    \infer[\rulename{E-Fun}]
        {\envExpr\vdash \fundot{x}{e}:\tau_1\rightarrow\tau_2}
        {\envExpr\uplus(x,\tau_1)\vdash e:\tau_2}\\
\end{equation}
\begin{equation}
    \infer[\rulename{E-Tuple}]
        {\envExpr\vdash e_0,\cdots,e_n:\tau_0\times\cdots\times\tau_n}
        {\envExpr\vdash e_0:\tau_0 &
        \cdots &
        \envExpr\vdash e_n:\tau_n}\\
\end{equation}
\begin{equation}
    \infer[\rulename{E-MakeRef}]
        {\envExpr\vdash \mathnkop{ref}e:\tau\ \mathsf{ref}}
        {\envExpr\vdash e:\tau}\\
\end{equation}
\begin{equation}
    \infer[\rulename{E-Ref}]
        {\envExpr\vdash e!:\tau}
        {\envExpr\vdash e:\tau\ \mathsf{ref}}\\
\end{equation}
\begin{equation}
    \infer[\rulename{E-SetRef}]
        {\envExpr\vdash e_1\coloneqq e_2:\typename{unit}}
        {\envExpr\vdash e_1:\tau\ \mathsf{ref} &
        \envExpr\vdash e_2:\tau}\\
\end{equation}
\begin{equation}
    \infer[\rulename{E-Apply}]
        {\envExpr\vdash e_1\ e_2:\tau}
        {\envExpr\vdash e_1:\tau'\rightarrow\tau &
        \envExpr\vdash e_2:\tau'} \\
\end{equation}
\begin{equation}
    \infer[\rulename{E-Let}]
        {\envExpr\vdash \mathnkop{let}x=e_1\mathnkop{in}e_2:\tau}
        {\envExpr\vdash e_1:\tau' &
        \envExpr\uplus(x,\tau')\vdash e_2:\tau}\\
\end{equation}
%\begin{equation}
%    \infer[\rulename{E-If}]
%        {\envExpr\vdash \mathnkop{if}e_1\mathnkop{then}e_2\mathnkop{else}e_3:\tau}
%        {\envExpr\vdash e_1:\typename{bool} &
%        \envExpr\vdash e_2:\tau &
%        \envExpr\vdash e_3:\tau}\\
%\end{equation}
\begin{equation}
    \infer[\rulename{E-Fix}]
        {\envExpr\vdash \fixdot{f}e:\tau}
        {\envExpr\uplus(f,\tau)\vdash e:\tau}\\
\end{equation}
\begin{equation}
    \infer[\rulename{E-Match}]
        {\envExpr\vdash \mathnkop{match}e\mathnkop{with}\mathit{clss}:\tau}
        {\envExpr\vdash e:\tau' &
        \envExpr\vdash \mathit{clss}:\tau'\rightarrow\tau}
\end{equation}
\end{subequations}
�߂̌^�͈ȉ��̗l�ɒ�`�����D
\begin{subequations}
\begin{equation}
    \infer[\rulename{C-Always}]
        {\envExpr\vdash p\rightarrow e:\tau'\rightarrow\tau}
        {\envExpr\vdash \patternandenv{p:\tau'}{\envPattern} &
        \envExpr\uplus\envPattern\vdash e:\tau}\\
\end{equation}
\begin{equation}
    \infer[\rulename{C-Guarded}]
        {\envExpr\vdash p\mathnkop{with}e_g\rightarrow e:\tau'\rightarrow\tau}
        {\envExpr\vdash \patternandenv{p:\tau'}{\envPattern} &
        \envExpr\uplus\envPattern\vdash e_g:\typename{bool} &
        \envExpr\uplus\envPattern\vdash e:\tau}\\
\end{equation}
\begin{equation}
    \infer[\rulename{C-Seq}]
        {\envExpr\vdash \mathit{cls}_0\clauseor\cdots\clauseor\mathit{cls}_n:\tau}
        {\envExpr\vdash cls_0:\tau &
        \cdots &
        \envExpr\vdash cls_n:\tau}
\end{equation}
\end{subequations}





\section{nibkame追加構文}\label{sc;additional-syntax}
以上で定義したnibkameコア言語に
if文などの糖衣構文(syntax sugar)を追加し,nibkame言語とする.

構文上の変換規則$\mathcal{A}$は以下のように定義される.
\begin{equation}
\begin{aligned}
    \semanticA{\mathnkop{if}e_c\mathnkop{then}e_\top\mathnkop{else}e_\bot} &=
        \mathnkop{match}\semanticA{e_c}\mathnkop{with}
            \mathnkval{true}\rightarrow \semanticA{e_\top}
            \clauseor \mathnkval{false}\rightarrow \semanticA{e_\bot} \\
    \semanticA{e} &= e \qquad\text{if $e$ is not matched in the definitions above.}
\end{aligned}
\end{equation}


\section{�W�����C�u����}
nibkame����ŕW���Ŏg�p�ł���֐��̎d�l�ɂ‚��ďq�ׂ�D

�����ŏq�ׂ�֐��̓\�[�X�R�[�h�̓ǂݍ��݈ȑO�����`����Ă���C�����`
�����֐��Ɠ����悤�Ɏg�p���邱�Ƃ��ł���D

\subsection{�Z�p���Z}
�g�p�ł���Z�p���Z�q�Ƃ��̌^�����Ɏ����D���ʂň͂܂ꂽ���ʎq�͒��u���Z�q
�ł��邱�Ƃ������D���ꂼ��̗D�揇�ʂ͕��i�̌v�Z�Ɠ����ł���D
\begin{lstlisting}
 (+) : int -> int -> int
 (-) : int -> int -> int
 ( * ) : int -> int -> int
 (/) : int -> int -> int
 (+.) : float -> float -> float
 (-.) : float -> float -> float
 ( *. ) : float -> float -> float
 (/.) : float -> float -> float
\end{lstlisting}

\subsection{��r���Z�q}
�g�p�ł����r���Z�q�Ƃ��̌^�����Ɏ����D���ꂼ��̉��Z�q�͎Z�p���Z�q���
�D�揇�ʂ͒Ⴂ�D
\begin{lstlisting}
 (=) : 'a -> 'a -> bool
 (<>) : 'a -> 'a -> bool
 (<=) : 'a -> 'a -> bool
 (<) : 'a -> 'a -> bool
 (>) : 'a -> 'a -> bool
 (>=) : 'a -> 'a -> bool
\end{lstlisting}
������r�͐����ƕ��������_���C�����^�݂̂ɒ�`����C�^�v����z��ɑ΂���
�͖���`�ł���D

\subsection{����p�������֐�}
�������v�Z�ɗp���Ȃ����Ƃ𖾎��I�Ɏ������߁C���̊֐�����`����Ă���D
\begin{lstlisting}
 ignore : 'a -> unit
\end{lstlisting}

\subsection{�W���o�͊֐�}
���̊֐��́C���ꂼ��̈������\����͊�֓��͉”\�Ȍ`���ň󎚂���֐��ł�
��D
\begin{lstlisting}
 print_int : int -> unit
 print_float : float -> unit
 print_char : char -> unit
 print_string : char array -> unit
\end{lstlisting}

\subsection{���X�g�����֐�}
���X�g�^�̒l����������֐������Ɏ����D
\begin{lstlisting}
 hd : 'a list -> 'a
 tl : 'a list -> 'a list
 null : 'a list -> bool
 map : ('a -> 'b) -> 'a list -> 'b list
 length : 'a list -> int
\end{lstlisting}
\lstinline|hd|��\lstinline|tl|�́C���X�g����擪�̗v�f���㑱�̃�
�X�g�����o���֐��ł���D��̃��X�g��K�p�����ꍇ�̓���͖���`�ł���D
\lstinline|null|�̓��X�g�̖����ł��邩���肷��֐��ł���D
\lstinline|map|�́C��2�����̃��X�g�̊e�v�f�ɑ�1�����̊֐���K�p���C����
�����X�g�ɂ��ĕԂ��֐��ł���D
\lstinline|length|�̓��X�g�̒��������߂�֐��ł���D

\subsection{�z�񏈗��֐�}
�z��𑀍삷��֐������Ɏ����D�z��̍X�V�͔j��I�ł��邱�Ƃɒ��ӂ���K�v
������D
\begin{lstlisting}
 array-create : 'a -> int -> 'a array
 array-set : 'a array -> 'a -> int -> unit
 array-ref : 'a array -> int -> 'a
 array-from-list : 'a list -> 'a array
 array-from-list-with-length : int -> 'a list -> 'a array
\end{lstlisting}
\lstinline|array-create|�͑�1�����������l�Ƃ����C��������2���������̔z��
��V���ɐ�������֐��ł���D
\lstinline|array-set|�͔z��̎w�肵���v�f��j��I�ɍX�V����֐��ł���D
\lstinline|array-ref|�͔z��̎w�肵���v�f�����o���֐��ł���D
\lstinline|array-from-list|�̓��X�g�Ɠ��������œ����v�f�����”z��𐶐�
����֐��ł���D
\lstinline|array-from-list-with-length|�̓��X�g�̓��C�w�肳�ꂽ�����̐�
���v�f�����”z��𐶐�����D���X�g��蒷���l���w�肳�ꂽ�Ƃ��C����ȍ~��% �{���ɁH
�l�͖���`�ł���D




%\chapter*{謝辞}
\addcontentsline{toc}{chapter}{謝辞}
\section*{謝辞}
樋口先生にはプロジェクト計画についてご意見をいただきました.大墳先生には発表
資料の作成や発表についてご指導いただきました.牛田先生には予稿などの書法につ
いて助言をいただきました.深謝の意を表します.

\bibliographystyle{jplain}
\bibliography{reference}

\backmatter
\appendix

\end{document}
