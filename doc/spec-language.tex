\documentclass[a4paper,titlepage,disablejfam]{jsbook}

\usepackage[dvipdfmx]{graphicx}

% 定理類似環境:
\newtheorem{theo}{定理}
\newtheorem{defi}[theo]{定義}
\newtheorem{lemm}[theo]{補題}
\newtheorem{prop}[theo]{命題}
\renewcommand{\proofname}{\bf 証明}
\pagestyle{myheadings}

% 人名
\newcommand{\sakamoto}{酒本 典明}
\newcommand{\kobori}{小堀 育男}

% 見出し:
\renewcommand{\lstlistlistingname}{プログラム一覧}
\newsavebox{\articleauthor}
\newcommand{\responsibility}[1]{\nopagebreak[2]\begin{flushright}[文責: #1]\end{flushright}}
\newcommand{\defresponsible}[2]{\newenvironment{#1}[3][\relax]{%
\ifx##1\relax#2[##2 (##3)]{##2}%
\else#2[##1 (##3)]{##2}\fi%
\sbox{\articleauthor}{##3}}%
{\responsibility{\usebox{\articleauthor}}}}
\defresponsible{resbonsiblechapter}{\chapter}
\defresponsible{resbonsiblesection}{\section}

% 略語:
\newcommand{\algorithmW}{algorithm $\mathscr{W}$}
\newcommand{\algorithmU}{algorithm $\mathscr{U}$}

% 記号:
\DeclareSymbolFont{symbolsC}{U}{txsyc}{m}{n}
\DeclareMathSymbol{\strictif}{\mathrel}{symbolsC}{74}
\DeclareMathSymbol{\boxright}{\mathrel}{symbolsC}{128}
\newcommand{\R}{\mathbb{R}}
\newcommand{\Z}{\mathbb{Z}}
\newcommand{\N}{\mathbb{N}}
\newcommand{\msubset}{\subseteq}
\newcommand{\mnsubset}{\nsubseteq}
\newcommand{\mpsubset}{\subset}
\newcommand{\mnpsubset}{\nsubset}
\newcommand{\msupset}{\supseteq}
\newcommand{\mnsupset}{\nsupseteq}
\newcommand{\mpsupset}{\supset}
\newcommand{\mnpsupset}{\nsupset}
\newcommand{\limpl}{\supset}
\newcommand{\gtrdotrel}{\mathrel{\gtrdot}}
\newcommand{\Rrel}{\mathrel{R}}
\newcommand{\curlyveeord}{\mathord{\curlyvee}}
\newcommand{\curlywedgeord}{\mathord{\curlywedge}}

\newcommand{\powerset}{\mathfrak{P}}
\newcommand{\domain}{\mathop{\mathfrak{Dom}}}
\newcommand{\codomain}{\mathop{\mathfrak{Cod}}}

\newcommand{\mathnkop}[1]{\mathop{\mathsf{#1}}}
\newcommand{\mathnkenv}[1]{\mathcal{#1}}
\newcommand{\mathnkval}[1]{\mathsf{#1}}
\newcommand{\mathset}[1]{\mathit{#1}}
\newcommand{\mathfunc}[1]{\mathit{#1}}
\newcommand{\newmonadicopdot}[2]{\newcommand{#1}[2]{#2##1.\,##2}}
\newmonadicopdot{\foralldot}{\forall}
\newmonadicopdot{\existsdot}{\exists}
\newmonadicopdot{\lambdadot}{\lambda}
\newcommand{\fixdot}[2]{\mathnkop{fix}#1\;#2}
\newcommand{\fundot}[2]{\mathnkop{fun}#1\rightarrow#2}
\newcommand{\semanticS}[1]{\mathcal{S}\left\llbracket#1\right\rrbracket}
\newcommand{\semanticM}[1]{\mathcal{M}\left\llbracket#1\right\rrbracket}
\newcommand{\mypair}[2]{\left(#1,#2\right)}

\newcommand{\lambdaIf}{\mathop{\mathrm{if}}}
\newcommand{\lambdaThen}{\mathop{\mathrm{then}}}
\newcommand{\lambdaElse}{\mathop{\mathrm{else}}}
\newcommand{\lambdaLet}{\mathop{\mathrm{let}}}
\newcommand{\lambdaIn}{\mathop{\mathrm{in}}}

\newcommand{\semanticSVal}{\mathop{\pi_\mathrm{val}}}
\newcommand{\semanticSVEnv}{\mathop{\pi_\rho}}

\newcommand{\ltrue}{\top}
\newcommand{\lfalse}{\bot}
\newcommand{\defeq}{\triangleq} % 定義時に使う等号

\newcommand{\envExpr}{\mathnkenv{E}}
\newcommand{\envType}{\mathnkenv{T}}
\newcommand{\envVariant}{\mathnkenv{V}}
\newcommand{\envPattern}{{\mathnkenv{E}_p}}
\newcommand{\substExpr}{{\Sigma_\mathnkenv{E}}}
\newcommand{\substType}{{\Sigma_\mathnkenv{T}}}
\newcommand{\substVariant}{{\Sigma_\mathnkenv{V}}}
\newcommand{\applysubst}[2]{\mathfrak{S}\ #1\ #2}
\newcommand{\removeassoc}[2]{\mathfrak{Rm}\ #1\ #2}
\newcommand{\patternandenv}[2]{#1\Rsh#2}
\newcommand{\valueandenv}[2]{\mypair{#1}{#2}}
\newcommand{\domainrestrict}[2]{#1\mathbin{\upharpoonright}#2}
\newcommand{\freevars}[1]{\mathit{FV}(#1)}
\newcommand{\freetypevars}[1]{\mathit{FTV}(#1)}
\newcommand{\boundvars}[1]{\mathit{BV}(#1)}
\newcommand{\clauseor}{\mathbin{|}}
\newcommand{\patternor}{\mathbin{|}}
\newcommand{\patternand}{\mathbin{@}}
\newcommand{\patternany}{\_}

% 日本語特有の記号
\newcommand{\jpdash}{―\nobreak\hspace{-0.5zw}\nobreak―\nobreak\hspace{-0.5zw}\nobreak―}

\newcommand{\rulename}[1]{\text{\bfseries\scshape #1}}
\newcommand{\typename}[1]{\text{\sffamily\bfseries #1}}
\newcommand{\widevec}[1]{\overrightarrow{#1}}
\newcommand{\varrange}[1]{\boldsymbol{\widetilde{#1}}} %\widetriangle \widering \wideparen \widetilde \widehat のいづれかが良いと思う
\newcommand{\refEq}[1]{式\ref{#1}}
\newcommand{\refTbl}[1]{表\ref{#1}}
\newcommand{\refFig}[1]{図\ref{#1}}
\newcommand{\refCh}[1]{\ref{#1}章}
\newcommand{\refSc}[1]{\ref{#1}節}
\newcommand{\refSsc}[1]{\ref{#1}項}
\newcommand{\refSssc}[1]{\ref{#1}目}

% 罫線に太線を追加
\makeatletter
\def\@arrayrule{\@addtopreamble{%
\vrule \@width \arrayrulewidth}}
\def\Hline{\noalign{\hrule height 3\arrayrulewidth}}
%\def\LTHline{\noalign{\hrule height 3\arrayrulewidth}}
\def\Vline{\vrule width 3\arrayrulewidth}
%\def\LTHline{\LT@Hline}
\def\LT@Hline{%
  \noalign{\ifnum0=`}\fi
    \penalty\@M
    \futurelet\@let@token\LT@@Hline}
\def\LT@@Hline{%
  \ifx\@let@token\Hline
    \global\let\@gtempa\@gobble
    \gdef\LT@sep{\penalty-\@medpenalty\vskip\doublerulesep}%
  \else
    \global\let\@gtempa\@empty
    \gdef\LT@sep{\penalty-\@lowpenalty\vskip-\arrayrulewidth}%
  \fi
  \ifnum0=`{\fi}%
  \multispan\LT@cols
     \unskip\leaders\hrule\@height3\arrayrulewidth\hfill\cr
  \noalign{\LT@sep}%
  \multispan\LT@cols
     \unskip\leaders\hrule\@height3\arrayrulewidth\hfill\cr
  \noalign{\penalty\@M}%
  \@gtempa}
\def\LT@makecaption#1#2#3{%
  \LT@mcol\LT@cols c{\hbox to\z@%
  {\hss\parbox[t]\LTcapwidth{%
    \sbox\@tempboxa{#1{#2\hskip1zw\relax}#3}%
    \ifdim\wd\@tempboxa>\hsize
      #1{#2\hskip1zw\relax}#3%
    \else%
      \hbox to\hsize{\hfil\box\@tempboxa\hfil}%
    \fi%
    \endgraf\vskip\baselineskip}%
  \hss}}}%
\def\LT@array[#1]#2{%
  \refstepcounter{table}\stepcounter{LT@tables}%
  \if l#1%
    \LTleft\z@ \LTright\fill
  \else\if r#1%
    \LTleft\fill \LTright\z@
  \else\if c#1%
    \LTleft\fill \LTright\fill
  \fi\fi\fi
  \let\LT@mcol\multicolumn
  \let\LT@@tabarray\@tabarray
  \let\LT@@hl\hline
  \def\@tabarray{%
    \let\hline\LT@@hl
    \LT@@tabarray}%
  \let\\\LT@tabularcr\let\tabularnewline\\%
  \def\newpage{\noalign{\break}}%
  \def\pagebreak{\noalign{\ifnum`}=0\fi\@testopt{\LT@no@pgbk-}4}%
  \def\nopagebreak{\noalign{\ifnum`}=0\fi\@testopt\LT@no@pgbk4}%
  \let\hline\LT@hline \let\kill\LT@kill\let\caption\LT@caption
  \let\Hline\LT@Hline
% \@tempdima\ht\strutbox%                                       変更1
  \iftdir\@tempdima\ht\tstrutbox\else\@tempdima\ht\strutbox\fi%  <-
  \let\@endpbox\LT@endpbox
  \ifx\extrarowheight\@undefined
    \let\@acol\@tabacol
    \let\@classz\@tabclassz \let\@classiv\@tabclassiv
    \def\@startpbox{\vtop\LT@startpbox}%
    \let\@@startpbox\@startpbox
    \let\@@endpbox\@endpbox
    \let\LT@LL@FM@cr\@tabularcr
  \else
    \advance\@tempdima\extrarowheight
    \col@sep\tabcolsep
    \let\@startpbox\LT@startpbox\let\LT@LL@FM@cr\@arraycr
  \fi
  \setbox\@arstrutbox\hbox{\vrule
    \@height \arraystretch \@tempdima
%   \@depth \arraystretch \dp \strutbox%          変更2
    \iftdir\@depth \arraystretch \dp \tstrutbox%   <-
    \else\@depth \arraystretch \dp \strutbox\fi%   <-
    \@width \z@}%
  \let\@sharp##\let\protect\relax
   \begingroup
    \@mkpream{#2}%
    \xdef\LT@bchunk{%
       \global\advance\c@LT@chunks\@ne
       \global\LT@rows\z@\setbox\z@\vbox\bgroup
       \LT@setprevdepth
       \tabskip\LTleft \noexpand\halign to\hsize\bgroup
      \tabskip\z@ \@arstrut \@preamble \tabskip\LTright \cr}%
  \endgroup
  \expandafter\LT@nofcols\LT@bchunk&\LT@nofcols
  \LT@make@row
% \m@th\let\par\@empty%                               変更3
  \iftdir\m@th\let\par\@@par%                          <-
  \else\m@th\let\par\@empty\fi%                        <-
  \everycr{}\lineskip\z@\baselineskip\z@
  \LT@bchunk}

\makeatother

% listings用設定
\lstdefinelanguage{nibkame}{%
   morekeywords={%
        let,letrec,if,match,with,type,fun,+,-,*,/,+.,-.,*.,\.,(),~-,;,
        and,or,not,unit,Nil%
   },%
%   morekeywords={%
%        ignore
%   },%
%   morekeywords={%
%        print_int,print_float,print_char,print_string
%   },%
%   morekeywords={%
%        hd,tl,null,map,length
%   },%
%   morekeywords={%
%        array-create,array-set,array-ref,array-from-list,array-from-list-with-length
%   },%
   sensitive,% ???
   alsodigit=->,%
   morecomment=[l];,%
   morecomment=[s]{\#|}{|\#},%
   morestring=[b]",%"
   literate=%
       {[|}{[\hskip -1pt$|$}2%
       {|]}{$|$\hskip -1pt]}2%
%      {[]}{\ensuremath{[\hskip -0.1em]}}2%
       {->}{\ensuremath{\rightarrow}~}2%
       {::}{\ensuremath{:\hskip -0.1em:}~}2%
}[keywords,comments,strings]

\lstdefinelanguage[Objective display]{Caml}[Objective]{Caml}{%
    morestring=[d]',%'
%    classoffset=2,
%    morekeywords={int,float,char,string},keywordstyle=\color{red},
%    classoffset=3,
%    morekeywords={list,array},keywordstyle=\color{blue},
%    classoffset=0,
    literate=%
        {[|}{[\hskip -1pt$|$}2%
        {|]}{$|$\hskip -1pt]}2%
%       {[]}{\ensuremath{[\hskip -0.1em]}}2%
        {->}{\ensuremath{\rightarrow}~}2%
        {::}{\ensuremath{:\hskip -0.1em:}~}2%
}
\lstset{%
  %language=[Objective display]Caml,%
  language=nibkame,%
  basicstyle={\normalfont\normalsize\sffamily},%
  commentstyle={\small\ttfamily\itshape\bfseries\upshape},%
  classoffset=1,%
  keywordstyle={\bfseries},%
  %frame=,%
  %framesep=0pt,%
  showstringspaces=false,%
  %numbers=left,%
  %numberstyle={\scriptsize},%
  %stepnumber=1,%
  tabsize=8,%
  lineskip=-0.5ex,%
%
  breaklines=true,%
  linewidth=\the\textwidth,
%  columns=[l]flexible%
  columns=flexible%
}


\begin{document}
%\title{電子情報工学科実験報告書 \\ 関数型言語の設計と実装}
%\title{\scshape The Programming Language Nibkame\\ Language Specification}
\title{nibkame言語仕様}
\author{
小堀 育男 \\
酒本 典明
}
%\date{}

\frontmatter

\maketitle

\tableofcontents
%\listoffigures
%\listoftables
%\lstlistoflistings

\newpage

\mainmatter % 序論も本文の内

\chapter{序論}\label{ch:intro}
\begin{abstract}
 本実験課題選定の理由と目的について論じる.また,実装の概要と報告書の構成を述べる.
\end{abstract}

\section{目的}
関数型言語の有用性はHughes\cite{hughes1989functional}などにより主張され
ていた.現在では,関数型言語に由来する機能が,C++やJava, C\# など
の言語に取り込まれている.またScala\footnote{\url{http://www.scala-lang.org/}}
やF\#{}\footnote{\url{http://msdn.microsoft.com/ja-jp/fsharp}}
など既存の手続き型言語との連携を強く意識した言語も登場している.

関数型言語に特徴的な機能として第一級の関数やクロージャなどがある.これら
は計算機の命令セットの持つ機能からかけ離れている.こういった機能を持った
言語を設計し,実装することで機能に対する理解を深めることを目標とする.

そのために,我々は静的に強く型付けされる関数型言語nibkameを設計しコンパ
イラを実装した.これは学生実験における参照実装として開発された
MinCaml\cite{住井英二郎:2008-04-24}を基にして開発を行なった.

MinCamlおよびMinCamlコンパイラは簡潔な実装(約2000行ほど)で性能の良いコー
ドを生成することに特徴のある関数型言語である.構文はObjective
Caml\footnote{\url{http://caml.inria.fr/ocaml/}}のサブセットであり,算術演算や
タプル構造,破壊的代入の可能な配列,高階関数,再帰と末尾呼び出し,型推論
などの機能が実現されており,レイトレーシングなどの複雑なプログラムを記述
することができる.MinCamlはプログラム言語処理系の教材として用いられるこ
とを主眼としており,その機能選定は実装を簡潔に保つことを重視して行われた.

簡潔さを保つために,多くのアプリケーションプログラムを記述するときに必須
でない機能は省略された.その中にはMinCamlコンパイラの記述に多く用いられ
ている代数的データ型やパターンマッチングなどが存在する.
nibkameではそういった機能と,その他にも生産性に高い影響を与える機能を追
加することでより実用的な言語を開発することを目指した.
MinCamlコンパイラはターゲットとする計算機としてSPARCとPowerPCをサポート
している.しかし実行する計算機の確保の容易さから,nibkameコンパイラはター
ゲットをIA-32アーキテクチャとした.

詳しくは\refCh{ch:lang_design}で述べるが,これらの目標を達成するために次
に挙げる機能の追加を行った.
\begin{itemize}
 \item トップレベル環境における複数の式
 \item 多相関数
 \item 代数的データ型
 \item パターンマッチング
 \item IA-32アーキテクチャ向け機械語の生成
\end{itemize}
また,これを達成するために,複数の定義をひとまとめに扱うモジュール機構と
3番地形式の中間言語の2番地形式への変換,メモリ管理機構の強化が必要となっ
た.


\section{構成} % ここは内容にあわせて書き換えてください.あと
			  % subsectionはオーバースペックな気が.
\begin{comment}
\refCh{ch:preparation}ではコンパイラの動作を理解するために必要となる理論
や規則について概説する.本報告書で用いる論理学や集合の記法についても説明
する.
\refCh{ch:lang_design}では,nibkame言語に搭載された機能のうち関数型言語
に特有なものについて説明し,最終的に実装された言語機能と構文を示す.
\refCh{ch:impl}ではコンパイラのモジュール構造と各フェーズの対応を示し,
コンパイラの実装について説明する.
\refCh{ch:sample-program}では実際のnibkameプログラムの例を示し,コンパイラの動作
と得られる目的プログラムについて説明する.
\refCh{ch:conclude}では実装した機能についてまとめ,これからの課題を議論する.
\end{comment}

%\mainmatter
%\chapter{原理}\label{ch:原理}

\chapter{記法}\label{ch:notation}
\begin{abstract}
本報告書で用いる理論・記法について概説する. 
本章で説かれる内容は本報告書で用いられる範囲に限定し,
かつ,直感的な解りやすさを最優先としたため,
論理的に不正確な箇所が多々あるがお赦し願いたい.
\end{abstract}

\section{全体的な記法について}\label{sc:about-notation}
以下,演算子の結合力を表す際に,
\begin{equation}\label{eq:bonding-strength}
\times \gtrdotrel +
\end{equation}
と表すこととする.
これは,$A\times B + C$という式があったときに$(A\times B)+ C$と解釈することを意味する.

等号系の記号の使い分けは\refTbl{tbl:equal-usage}のようにする.

\begin{table}[hbt]
    \caption{等号系記号の使い分け}\label{tbl:equal-usage}
    \begin{center}
    \begin{tabular}{cp{40zw}}
        \Hline
        \multicolumn{1}{c}{記号} & \multicolumn{1}{c}{目的} \\
        \hline
        $=$ & 通常の等値,即ち数値・真偽値・集合などの等しさを真偽値で表す,または,等しいことを言明する. \\
        $\equiv$ & $=$よりも広い意味で同じであることを表す. \\
        $\defeq$ & 通常の式において左辺の記号が右辺で定義されることを示す.\\
        $\Coloneqq$ & BNFにおいて,左辺の記号が右辺で定義されることを示す.\refSc{sc:about-BNF-form} \\
        \Hline
    \end{tabular}
    \end{center}
\end{table}

変数$e$の動く領域を$\varrange{e}$と表すこととする\footnote{一般的な方法ではない}.

自然数$\N$は0を含むこととする.


\section{集合について}\label{sc:about-set}
元として含むことを表す記号として$\in$,$\ni$を用いる.  部分集合を表す記号として$\msubset$,$\msupset$を,真部分集合を表す記号として$\mpsubset$,$\mpsupset$を用いる. 空集合の記号としては$\emptyset$を用いる.

集合$x$の羃集合は$\powerset x$若しくは$\powerset(x)$と書く
\footnote{$2^x$や$\boldsymbol{2}^x$という表記も一般的.}.

順序対若しくは単純に対は,$(x,y)$と表すことにする\footnote{標準的には$\left\langle x,y\right\rangle$である.}.
順序対は
\begin{equation}
(x_0,x_1) = (y_0,y_1) \iff x_0 = y_0 \land x_1 = y_1
\end{equation}
という式によって特徴付けられる.
組は$(x_0,x_1,\cdots,x_n)$と表すことにする\footnote{これも標準的には$\left\langle x_0,x_1,\cdots,x_n\right\rangle$である.}.
順序対と同様に
\begin{equation}
(x_0,x_1,\cdots,x_n) = (y_0,y_1,\cdots,x_n) \iff \foralldot{i\in\{0,1,\cdots,n\}}{x_i = y_i}
\end{equation}
となる.
$(x_0,x_1,\cdots,x_n)$は$(x_i)_{i\in\{0,1,\cdots,n\}}$とも書く.

積集合とはいくつかの集合の要素で組を作ったときの集合のことであり,
\begin{equation}
X_0\times X_1\times\cdots\times X_n\triangleq \{(x_0,x_1,\cdots,x_n)|\foralldot{i\in\{0,1,\cdots,n\}}{x_i\in X_i}\}
\end{equation}
と定義される.
$X_0\times X_1\times\cdots\times X_n$は$\prod_{i\in\{0,1,\cdots,n\}} X_i$とも書く.

標準射影$\pi_i:\left(\prod_{j\in\{0,1,\cdots,n\}} X_j\right)\rightarrow X_i$とは,組または対からその構成要素を取り出す演算であり,
\begin{equation}
\pi_i(X_0\times X_1\times\cdots\times X_n)\triangleq X_i
\end{equation}
である.

順序対の集合$X$\jpdash 例えば,$X=\{(x,y)|x\in\R,y\in\Z\}$など\jpdash は二項関係とも呼ばれ,
その始域$\domain X$,終域$\codomain X$は夫々,
\begin{align}
    \domain X &\triangleq \{\pi_0 x|x\in X\}\\
    \codomain X &\triangleq \{\pi_1 x|x\in X\}
\end{align}
と定義される.
二項関係$R$に対して,始域を集合$C$に限定した二項関係$\{x|x\in R, \pi_0 x\in C\}$を
$\domainrestrict{R}{C}$と書く.
二項関係への追加$R\uplus\mypair{x}{y}$は,
$R\uplus\mypair{x}{y}\triangleq \domainrestrict{R}{\left(\domain R\setminus \{x\}\right)}$
と定義される.

$x_0,x_1,\cdots,x_n$ ($n\in\N$)という並びをを$\vec{x}$と略記する.



\section{論理について}\label{sc:about-logic}
真・偽を表すの記号として夫々,$\ltrue$,$\lfalse$を用いる.
論理和・論理積・否定・含意の記号として夫々,$\lor$,$\land$,$\lnot$,$\limpl$を用いる.
変数$x$がその領域$\varrange{x}$を動くとき,
述語$P(x)$が全ての$x$で真になることを$\foralldot{x}{P(x)}$,
述語$P(x)$がある$x$で真になることを$\existsdot{x}{P(x)}$と表す.
$\foralldot{x}{\foralldot{y}{\cdots}}$は$\foralldot{x,y,\cdots}{}$と略記する.
$\exists$も同様.

$\forall$,$\exists$,$\lnot$,$\limpl$は右結合性,
例えば,$\lnot\lnot A\equiv \lnot(\lnot A)$,$A\limpl B\limpl C \equiv A\limpl (B\limpl C)$とする.
夫々の演算子の優先順位は$\lnot\gtrdotrel\land,\lor\gtrdotrel\limpl$となる.

\begin{equation}\label{eq:def-mp}
\infer[\rulename{M.P.}]{\beta}{\alpha & \alpha \limpl \beta}
\end{equation}
は,$\alpha$と$\alpha \limpl \beta$という仮定から,\rulename{M.P.}という推論規則によって$\beta$が導き出されることを表す記法である.
\ref{eq:def-mp}によって定義される推論規則は,推論規則内の$\alpha$,$\beta$に式を代入することによって
\begin{equation}\label{eq:def-mp-instance-1}
\infer[\rulename{M.P.}]{A\land B}{A & A \limpl A\land B}
\end{equation}
や,
\begin{equation}\label{eq:def-mp-instance-2}
\infer[\rulename{M.P.}]{\foralldot{x}{Q(a, x) \limpl P(x)}}{\existsdot{x}{P(x)} & \existsdot{x}{P(x)} \limpl \foralldot{x}{Q(a, x) \limpl P(x)}}
\end{equation}
も表す.
この例における$\alpha$,$\beta$のように,超文脈的に扱われる変数のことをメタ変数(meta-variable)と呼ぶ.


$\Gamma_1,\Gamma_2,\cdots\vdash A_1, A_2,\cdots$は,
主に,$\Gamma_1, \Gamma_2, \cdots$から$A_1, A_2, \cdots$が推論できることを表す記法として用いる.


\section{バッカス・ナウア記法について}\label{sc:about-BNF-form}
バッカス・ナウア記法(Backus-Naur Form, BNF)
は文脈自由文法を定義するのに用いられるメタ言語であって,
プログラミング言語の構文記述やデータ構造の記述など,広汎に利用される.
計算機科学でも式として頻繁に用いられるので解説する.

$\Coloneqq$の左辺にある名前は右辺にある式で定義される.
例えば,
\begin{equation}\label{eq:bnf-example-1} 
e\Coloneqq \mathsf{abc}
\end{equation}
というBNFに対しては
\begin{equation}
\varrange{e} = \{\mathsf{abc}\}
\end{equation}
となる.

右辺において$\mid$は「または」の意味を表し,$\mid$で区切られたいずれかの式に当てはまる.例えば,
\begin{equation}\label{eq:bnf-example-2} 
e\Coloneqq \mathsf{a} \mid \mathsf{b} \mid \mathsf{c}
\quad\text{または}\quad
\begin{array}{rcl}
    e    &\Coloneqq& \mathsf{a} \\
            &\mid&   \mathsf{b} \\
            &\mid&   \mathsf{c} 
\end{array}
\end{equation}
というBNFに対しては
\begin{equation}
\varrange{e} = \{\mathrm{a}, \mathrm{b}, \mathrm{c}\}
\end{equation}
となる.

右辺において左辺の記号が使われた場合は再帰となる
\begin{equation}\label{eq:bnf-example-3} 
\begin{array}{rcl}
    e    &\Coloneqq& \mathsf{a} \\
            &\mid&   e\mathsf{'} 
\end{array}
\end{equation}
というBNFに対しては
\begin{equation}
\varrange{e} = \{\mathrm{a}, \mathrm{a'}, \mathrm{a''}, \mathrm{a'''}, \cdots \}
\end{equation}
となる.


\section{ラムダ式について}\label{sc:about-lambda-expression}
ラムダ式(lambda expression,$\lambda$-expression) とは,一言で言ってしまうと関数そのものを抽象化したものである.

例えば以下の式を考えよう.
\begin{align}
f(x) &= (x + 1)^2\\
f(x + 1) &= {\left((x + 1) + 1\right)}^2 = x^2 + 4x + 4\\
(x + 1) &= x + 1
\end{align}
これは$(x + 1)$に対して関数$f$を適用すると$x^2 + 4x + 4$となるが,
関数を適用しないとそれは$x + 1$そのものであることを表している.

では逆に,$f$に引数を適用しないとどうなるのであろうか.
\begin{align}
f(x) &= x^2 + 2x + 1\\
f    &= \;?
\end{align}
この場合の$f$そのものを表すのにラムダ式を用いることが出来,そうすると
\begin{align}
f &= \lambdadot{x}{(x^2 + 2x + 1)}
\end{align}
となって関数そのものを陽に扱うことが出来る.
以下,必要の無い限り括弧は省略し,$\lambdadot{x}{(x^2 + 2x + 1)}$は$\lambdadot{x}{x^2 + 2x + 1}$と書く.

ラムダ式を用いて記述した関数に引数を適用させると,
\begin{align}
f(y) &= (\lambdadot{x}{x^2 + 2x + 1})(y) = y^2 + 2y + 2 \\
f(2) &= (\lambdadot{x}{x^2 + 2x + 1})(2) = 2^2 + 2\cdot 2 + 1 = 9
\end{align}
となる.

次に複数引数の関数を考えよう.
\begin{align}
g(x, y) &= x + y
\end{align}
この関数を表すためには,
$\lambdadot{x,y}{}$という記法を導入するのが良いと思われる.
従って,
\begin{align}
\label{eq:lambda-multiple}
g &= \lambdadot{x,y}{x + y}\\
\label{eq:lambda-multiple-inst}
g(a, b) &= a + b\\
g(1, 2) &= (\lambdadot{x,y}{x + y})(1, 2) = 1 + 2 = 3
\end{align}
と表せる.

ここで,天下り的ではあるが,$\lambdadot{x}{\lambdadot{y}{x+y}}$を考える.
すると,
\begin{gather}
\begin{aligned}
\label{eq:lambda-curried}
h &= \lambdadot{x}{\lambdadot{y}{x+y}} \\
\end{aligned} \\
\begin{aligned}
\label{eq:lambda-curried-partial}
h(a) &= (\lambdadot{x}{\lambdadot{y}{x+y}})(a)\\
     &= {\lambdadot{y}{a+y}}
\end{aligned} \\
\begin{aligned}
\label{eq:lambda-curried-inst}
\left(h(a)\right)(b) &= \left((\lambdadot{x}{\lambdadot{y}{x+y}})(a)\right)(b) \\
        &= (\lambdadot{y}{a+y})(b)\\
        &= a + b
\end{aligned} \\
\begin{aligned}
\left(h(1)\right)(2) &= \left((\lambdadot{x}{\lambdadot{y}{x+y}})(1)\right)(2) \\
        &= (\lambdadot{y}{1+y})(2) \\
        &= 1 + 2 = 3
\end{aligned}
\end{gather}
このようになる.
以下,$f(a)(b)(c)\cdots$は
$\delimitershortfall-1pt\left({\cdots\left({\left({\left({f(a)}\right)(b)}\right)(c)}\right)\cdots}\right)$
を表すものとし,
必要の無い限り括弧を省略する.

ここで,
\refEq{eq:lambda-multiple-inst}と\refEq{eq:lambda-curried-inst}
を見比べると$g(a, b) = h(a)(b)$であることがわかる.
つまり,
\begin{equation}
(\lambdadot{x,y}{\cdots})(a,b) \equiv (\lambdadot{x}{\lambdadot{y}{\cdots}})(a)(b)
\end{equation}
といえる.

これは簡単に拡張が出来て,任意の多変数関数$g$に対してある関数$h$があって
\begin{equation}
g(a_1, a_2, \cdots, a_n)= h(a_1)(a_2)\cdots(a_n)
\end{equation}
となる.
この操作$\mathrm{curry}: g\mapsto h$をカリー化(currying)と呼ぶ.
纏めると,カリー化を行うことにより全ての多変数関数は一変数関数で書くことが出来ることとなる.

このことより,
$\lambdadot{x}{\lambdadot{y}{\lambdadot{z}{\cdots}}}$の略記として
$\lambdadot{x,y,z,\cdots}{}$または
$\lambdadot{xyz\cdots}{}$と書き,
$f(a)(b)(c)\cdots$を
$f(a,b,c\cdots)$と書くことがある.

ここで,二変数の関数に一つだけ引数を与えた\refEq{eq:lambda-curried-partial}のような場合を考える.
\begin{align}
f (x)(y) &= x + y \quad \left(f = \lambdadot{xy}{x + y}\right) \\
\mathit{add3}(x)   &= 3 + x \quad \left(\mathit{add3} = \lambdadot{x}{3 + x}\right) 
\end{align}
なる$f$,$\mathit{add3}$を考えたときに
\begin{align}
\begin{aligned}
f(3)            &= (\lambdadot{x}{\lambdadot{y}{x+y}})(3) \\
                &= \lambdadot{y}{3 + y} \\
                &= \lambdadot{x}{3 + x} \\
                &= \mathit{add3}
\end{aligned}
\end{align}
となって,二つの引数を持つ関数$f$に引数を一つ適用した関数$f(3)$と
常識的な方法で定義した引数に3を加える関数$\mathit{add3}$が等しくなる.
このように$n$の引数を取る関数に対して$m$ ($<n$)の引数を適用することを部分適用(partial application)と呼ぶ.

以下,関数の引数を表す括弧は省略し,
$f(x)(y)(z)\cdots$は$f\ x\ y\ z\ \cdots$または$f x y z \cdots$と書く.


% \section{ラムダ計算}\label{sc:about-lambda-calculus}

% \section{型付ラムダ計算}\label{sc:about-typed-lambda-calculus}

\section{型について}\label{sc:about-type}

ある式$e$が型$t$を持つことを
\begin{equation}
e: t
\end{equation}
と書く.

例えば,真偽値をとる変数$b$があったとき,
その型は真偽値型(\textsf{bool})であると定めることが出来る.

関数$f$と型$t$の式$e:t$があって,$e$に$f$を適用した$f\ e$の型が$s$であるとき,
$f$の型は$t\rightarrow s$であるとする.



\chapter{対象言語の設計}\label{ch:lang_design}

\begin{abstract}
この章では,
まずnibkame言語のコア部分の仕様を定義し,
後に,その拡張としてnibkame言語本体とライブラリの仕様について述べる.
そして,最終的に実装された言語機能と構文を示す.
\end{abstract}

\section{式の構文}\label{sc:expr-syntax}
nibkameコア言語における式の定義は以下の様になる.
\begin{subequations}
\begin{equation}\label{eq:def-expr} 
\begin{array}{rcll}
e&\Coloneqq& c                          & \text{定数 (\refSsc{ssc:expr-constant})} \\
 &\mid&  s:\tau                     & \text{外部定数 (\refSsc{ssc:expr-external})} \\
 &\mid&  v                          & \text{変数 (\refSsc{ssc:expr-variable})} \\
 &\mid&  g                          & \text{型構築子 (\refSsc{ssc:expr-constructor})} \\
 &\mid&  \fundot{x}{e}           & \text{関数 (\refSsc{ssc:expr-function})} \\
 &\mid&  e, \cdots, e               & \text{組 (\refSsc{ssc:expr-tuple})} \\
 &\mid&  \mathnkop{ref}e            & \text{参照の生成 (\refSsc{ssc:expr-reference})} \\
 &\mid&  e!                         & \text{参照 (\refSsc{ssc:expr-reference})} \\
 &\mid&  e\coloneqq e                & \text{代入 (\refSsc{ssc:expr-reference})} \\
 &\mid&  e\ e                       & \text{適用 (\refSsc{ssc:expr-apply})} \\
 &\mid&  \mathnkop{let}x=e\mathnkop{in}e & \text{束縛 (\refSsc{ssc:expr-let})} \\
 &\mid&  \fixdot{f}{\fundot{x}{e}} & \text{不動点演算子 (\refSsc{ssc:expr-fix})} \\
% &\mid&  \mathnkop{if}e\mathnkop{then}e\mathnkop{else}e & \text{条件 (\refSsc{ssc:expr-if})} \\
 &\mid&  \mathnkop{match}e\mathnkop{with}\mathit{clss} & \text{パターンマッチング (\refSsc{ssc:expr-match})} \\
 &\mid&  \mathnkop{match}\clauseor e\mathnkop{with}\mathit{clss} & \text{同上.最初の$\clauseor$は無視される.} \\
\mathset{Expr}&\triangleq& \varrange{e}            & \text{式} \\
\mathset{ExprVar}&\triangleq& \varrange{x}            &  \\
\mathset{ExprConst}&\triangleq& \varrange{c}            &  \\
\end{array}
\end{equation}
\begin{equation}\label{eq:def-clause}
\begin{array}{rcll}
\mathit{cls} &\Coloneqq& p\rightarrow e & \\
             &\mid& p \mathnkop{when} e \rightarrow e & \text{ガード付} \\
\mathit{clss} &\Coloneqq& \mathit{cls} & \\
              &\mid& \mathit{clss}\clauseor\mathit{cls} & \\
\mathset{Clause}&\triangleq& \varrange{\mathit{cls}}            & \text{節} \\
\end{array}
\end{equation}
\begin{equation}\label{eq:def-pattern}
\begin{array}{rcll}
p&\Coloneqq&\patternany             & \text{ワイルドカード} \\
 &\mid&  c                          & \text{定数} \\
 &\mid&  x                          & \text{変数} \\
 &\mid&  g                          & \text{型構築子} \\
 &\mid&  g\ p                       & \text{型構築子適用} \\
 &\mid&  p,\cdots,p                 & \text{組} \\
 &\mid&  p \patternand p            & \text{連言} \\
 &\mid&  p \patternor p             & \text{選言} \\
 &\mid&  \lnot p                    & \text{否定} \\
\mathset{Pattern}&\triangleq& \varrange{p}            & \text{パターン} \\
\mathset{Constructor}&\triangleq& \varrange{g}            & \text{型構築子} \\
\mathset{Var}&\triangleq&\mathset{ExprVar}\cup\mathset{Constructor}
\end{array}
\end{equation}
\end{subequations}
ここで,$\mathset{ExprVar}\cap\mathset{Constructor}=\emptyset$である.

\subsection{定数}\label{ssc:expr-constant}
定数とは
\begin{itemize}
    \item \lstinline|()|
    \item \lstinline|Nil|
    \item \lstinline|true|
    \item \lstinline|false|
    \item 整数
    \item 浮動小数点数
    \item 文字
\end{itemize}
の五つである.

このうち,\lstinline|()|はC言語で言うところのvoidのようなもので,
関数が引数を取らない若しくは返さないことを明示するのに用いる.
\lstinline|Nil|はC言語で言うところのNULLのようなもので,
リストの末尾であることを示す.

数についていくつか例を挙げると\refTbl{tbl:int-float}のようになる.

\begin{table}[h]
    \caption{整数と浮動小数点数}\label{tbl:int-float}
    \begin{center}
    \begin{tabular}{crl}
    \Hline
    整数の例&\lstinline|0|,
             \lstinline|1|,
             \lstinline|100|,
             \lstinline|-1| \\
    \hline
    浮動小数点数の例 &\lstinline|0.0|,
             \lstinline|1.0|,
             \lstinline|-1.0|,
             \lstinline|1.0e10|,
             \lstinline|1.0e-10|\\
    \Hline
    \end{tabular}
    \end{center}
\end{table}

nibkameコア言語において用意されている組み込みの文字型は
char型であるが,この型は%C言語と同様に
8ビットの符号無し整数である.
%char型であり,計算機上での表現は処理系に依存する.
文字のコード上での表現は\lstinline|'c'|のようになる.

\subsection{外部定数}\label{ssc:expr-external}
外部定数は,nibkameコア言語ソースコード上で既に定義されているものとして使用することの出来る
定数であり,具体的には外部のCランタイムの関数や加算・減算などのプリミティブ関数である.

\subsection{組}\label{ssc:expr-tuple}
組とはいくつかの値の並びのことであり,
集合における組とほぼ同じものである.

\subsection{変数}\label{ssc:expr-variable}
変数は一般のプログラミング言語における概念と同じである.
しかし,nibkameにおける変数は生成時に行われる束縛の後は一切の代入を行うことが出来ない.
そのため,参照が用いられていない場合には,同じ式を評価すると同じ結果となる.

\subsection{束縛}\label{ssc:expr-let}
変数束縛を行うのが$\mathnkop{let}$式である.
nibkameにおける変数のスコープはレキシカルスコープ(lexical scope)である.

例えば,
\begin{equation}
\begin{aligned}
&\mathnkop{let} r = \\
&\quad \mathnkop{let} x = 5 \mathnkop{in} \\
&\quad \mathnkop{let} a = x \mathnkop{in} \\
&\quad \mathnkop{let} x = 3 \mathnkop{in} \\
&\quad \mathnkop{let} b = x \mathnkop{in} \\
&\quad\quad a, b, x
\end{aligned}
\end{equation}
というコードでは
$r$は$5, 3, 3$となる.

\subsection{参照}\label{ssc:expr-reference}
nibkameにおいては一度束縛した変数の中身を変更することが出来ない.
しかし,実際に副作用の一切無いコードを書くのは手間なので,
参照という概念を導入し,
変数に格納されている「データが格納されている場所への参照」は変化しないが,
「データが格納されている場所への参照」をたどって,格納されているデータを得たり,変更したりすることが出来るようになっている.

例えば,
\begin{equation}
\begin{aligned}
&\mathnkop{let} p = \mathnkop{ref} 3 \\
&\mathnkop{let} q = \\
&\quad \mathnkop{let} x = p \coloneqq 7 \mathnkop{in} \\
&\quad\quad p!
\end{aligned}
\end{equation}
というコードは上から順に実行され,
最終的に\lstinline|q|は\lstinline|7|,\lstinline|p|は\lstinline|7|が格納されている場所への参照が束縛されていることなる.

\subsection{型構築子}\label{ssc:expr-constructor}
型構築子とはいわゆる代数的データ構造を作るものである.
集合で言うところの直和と等しい.

\subsection{パターンマッチング}\label{ssc:expr-match}
パターンマッチングは,条件分岐を行ったり,
型構築子によって作られた代数的データ型の値から,
中身を取り出して変数に束縛したりできる.

$\mathnkop{match}e\mathnkop{with}\mathit{clss}$
という式の場合だと,
節$p\mathnkop{with}e_g\rightarrow e_c$を先頭から順に
パターン$p$を$e$の値に当てはめることができるかどうかと,ガード$e_g$の値が$\mathnkval{true}$どうかを調べ,
両方成立する場合は式$e_c$を評価し,その値がmatch式の値となる.
ただし,ガードのない節$p\rightarrow e_c$は$p\mathnkop{with}\mathnkval{true}\rightarrow e_c$と等価である.

\paragraph{定数}
定数パターンは定数にのみマッチする.たとえば,
\begin{equation}
\begin{aligned}
&\mathnkop{match}\text{\lstinline|'b'|}\mathnkop{with} \\
&\quad \clauseor \text{\lstinline|'a'|}\rightarrow 1 \\
&\quad \clauseor \text{\lstinline|'b'|}\rightarrow 2 \\
&\quad \clauseor \text{\lstinline|'c'|}\rightarrow 3 \\
&\quad \clauseor \text{\lstinline|'d'|}\rightarrow 4
\end{aligned}
\end{equation}
というコードにおいてmatch式の値は$2$となる.

\paragraph{ワイルドカード}
ワイルドカードパターンは型構築子を除く全てにマッチする.たとえば,
\begin{equation}
\begin{aligned}
&\mathnkop{match}e\mathnkop{with} \\
&\quad \clauseor \patternany \rightarrow 1
\end{aligned}
\end{equation}
というコードにおいてmatch式の値は,
変数$e$の値がなんであろうと常に$1$である.

また,パターンの評価順序より,
\begin{equation}
\begin{aligned}
&\mathnkop{match}\text{\lstinline|'d'|}\mathnkop{with} \\
&\quad \clauseor \text{\lstinline|'a'|}\rightarrow 1 \\
&\quad \clauseor \text{\lstinline|'b'|}\rightarrow 2 \\
&\quad \clauseor \patternany \rightarrow 3 \\
&\quad \clauseor \text{\lstinline|'d'|}\rightarrow 4
\end{aligned}
\end{equation}
というコードにおいてmatch式の値は,
ワイルドカードパターンにマッチするため$3$となる.

\paragraph{変数}
変数パターンはワイルドカードパターンとほぼ同じであり,型構築子を除く全てにマッチする.
ワイルドカードパターンとの違いは,
変数パターン$x$に値$e$が当てはまったとき,
変数$x$に値$e$が束縛された状態で
節のガード$e_g$と式$e_c$が評価されることである.

たとえば,
\begin{equation}
\begin{aligned}
&\mathnkop{match}\text{\lstinline|'a'|}\mathnkop{with} \\
&\quad \clauseor x\rightarrow 1,x \\
\end{aligned}
\end{equation}
というコードにおいてmatch式の値は,
$1,\text{\lstinline|a|}$となる.

\paragraph{型構築子・型構築子適用}
型構築子パターンと型構築子適用パターンは代数的データ型に対して適用される.
型構築子の型によってどちらが使われなくてはいけないかが決まり,
引数を取らない型構築子に対しては型構築子パターン,
引数を取る型構築子に対しては型構築子適用パターンが使われる.

たとえば,
\begin{equation}
\begin{aligned}
\mathnkvariant{Left} : \typename{int}\rightarrow \typename{intEither}\\
\mathnkvariant{Right} : \typename{int}\rightarrow \typename{intEither}
\end{aligned}
\end{equation}
という型構築子があった場合,
\begin{equation}
\begin{aligned}
&\mathnkop{match}\mathnkvariant{Left}\ 2\mathnkop{with} \\
&\quad \clauseor \mathnkvariant{Left}\ x\rightarrow 1,x \\
&\quad \clauseor \mathnkvariant{Left}\ 2\rightarrow 2,x \\
&\quad \clauseor \mathnkvariant{Right}\ x\rightarrow 3,x
\end{aligned}
\end{equation}
というコードにおいてmatch式の値は$2,2$となる.

また,
\begin{equation}
\begin{aligned}
\mathnkvariant{MyTrue} : \typename{mybool}\\
\mathnkvariant{MyFalse} : \typename{mybool}
\end{aligned}
\end{equation}
という型構築子があった場合,
\begin{equation}
\begin{aligned}
&\mathnkop{match}\mathnkvariant{MyTrue}\mathnkop{with} \\
&\quad \clauseor \mathnkvariant{MyTrue}\rightarrow 1 \\
&\quad \clauseor \mathnkvariant{MyFalse}\rightarrow 0 \\
\end{aligned}
\end{equation}
というコードにおいてmatch式の値は$1$となる.

\paragraph{組}
組パターンは組に対して適用される.
組の数は一致している必要がある.

たとえば,
\begin{equation}
\begin{aligned}
&\mathnkop{match}1,2,\mathnkval{false}\mathnkop{with} \\
&\quad \clauseor x,y,\mathnkval{true}\rightarrow x,y \\
&\quad \clauseor x,y,\mathnkval{false}\rightarrow y,x \\
\end{aligned}
\end{equation}
というコードにおいてmatch式の値は$2,1$となる.

\paragraph{連言}
連言パターン$p_1\patternand \cdots\patternand p_n$は
全てのパターン$p_i$がマッチしたときのみマッチする.
ただし,各パターンに出現する変数パターンの名前は,他のパターンに出現してはいけない.

たとえば,
\begin{equation}
\begin{aligned}
&\mathnkop{match}1,(2,\mathnkval{false})\mathnkop{with} \\
&\quad \clauseor x,((1, z)\patternand(y,(b\patternand\mathnkval{false})))\rightarrow x,b,z \\
&\quad \clauseor x,((2, c)\patternand(y,\mathnkval{false}))\rightarrow x,y,c \\
\end{aligned}
\end{equation}
というコードにおいてmatch式の値は$1,2,\mathnkval{false}$となる.

\paragraph{選言}
選言パターン$p_1\patternor \cdots\patternor p_n$は
全てのパターン$p_i$がマッチしたときのみマッチする.
ただし,各パターンに出現する変数パターンの名前と型の集合
は全て同じでなければいけない.

たとえば,
\begin{equation}
\begin{aligned}
&\mathnkop{match}1,(2,\mathnkval{false})\mathnkop{with} \\
&\quad \clauseor x,((1, z)\patternor(\patternany,(z\patternand\mathnkval{true})))\rightarrow z,x \\
&\quad \clauseor x,((1, z)\patternor(\patternany,(z\patternand\mathnkval{false})))\rightarrow x,z \\
\end{aligned}
\end{equation}
というコードにおいてmatch式の値は$1,\mathnkval{false}$となる.

\paragraph{否定}
否定パターン$\lnot p$はパターン$p$がその位置でマッチしなかった時に,マッチする.
ただし,パターン$p$の中に変数パターンを含んではならない.

たとえば,
\begin{equation}
\begin{aligned}
&\mathnkop{match}1\mathnkop{with} \\
&\quad \clauseor \lnot 1\rightarrow \mathnkval{true} \\
&\quad \clauseor \lnot 2\rightarrow \mathnkval{false} \\
\end{aligned}
\end{equation}
というコードにおいてmatch式の値は$1,\mathnkval{false}$となる.
また,否定パターンは連言・選言パターンとともに使用すると便利であり,
\begin{equation}
\begin{aligned}
&\mathnkop{match}1,(2,\mathnkval{false})\mathnkop{with} \\
&\quad \clauseor x,((\lnot 1, z)\patternand(y,(b\patternand\mathnkval{false})))\rightarrow x,b,z \\
&\quad \clauseor x,((2, c)\patternand(y,\mathnkval{false}))\rightarrow x,y,c \\
\end{aligned}
\end{equation}
というコードにおいてmatch式の値は$1,\mathnkval{false},2$となる.

\subsection{関数}\label{ssc:expr-function}
関数はある値をとってある値を返す値である.
nibkameにおいて関数は第一級のオブジェクト(first-class object)なので,
変数に束縛したり,関数の返り値として返したりすることができる.

\subsection{不動点演算子}\label{ssc:expr-fix}
nibkameコア言語において,
$\mathnkop{let} x = e_1 \mathnkop{in} e_2$という束縛の式において
変数$x$の定義$e_1$の中で$x$を使用することができない.
よって,再帰関数を定義するために不動点演算子(不動点コンビネータとも)を用いる.
不動点演算子$\mathnkop{fix}$は以下の特徴を持つ.
\begin{equation}
\mathnkop{fix}f= f(\mathnkop{fix}f)
\end{equation}

\subsection{適用}\label{ssc:expr-apply}
関数に値を与え値を得ることである.
nibkameにおいて評価戦略(evaluation strategy)は
値渡し(call-by-value)である.
そのため,関数に渡される引数は事前に評価されてから渡されることになる.


\section{�^}\label{sc:type}
nibkame�ł͌^�V�X�e���Ƃ��āCHindley-Milner�^�V�X�e�����̗p�����D

Hindley-Milner�^�V�X�e���Ƃ�Haskell��ML�̂悤�ȐÓI�^�t���֐��v���O���~���O����ɂ悭�p������^�V�X�e���ł���C
�u���ׂĂ̎��Ɉ�ӂȍŔ�(most general)�^�C������^(principal type)��t���邱�Ƃ��ł���v�Ƃ����d�v�Ȑ��������D

��\�I�ȃA���S���Y���́u\algorithmW�v\cite{Milner1978348}\cite{Damas:1982:PTF:582153.582176}�Ƃ��Ēm���Ă�����̂ł���D
���̃A���S���Y����Hindley-Milner�A���S���Y���Ƃ�Damas-Milner�A���S���Y���Ƃ��Ă΂��D
�^�ϐ��ւ̑���ɂ��^�̓��ꉻ(unification)����{����Ƃ��Ă���D

���̃A���S���Y���̋N���́CHaskell Curry��Robert Feys��1958�N�ɒP���^�t�����_�v�Z�̌^���_�̂��߂ɍl�Ă������̂ł���C
1969�N��J. Roger Hindley��������g�����C�܂��C���ꂪ��ɍŔĂȌ^�ɐ��_���邱�Ƃ��ؖ������D
1978�N��Robin Milner��Hindley�Ƃ͓Ɨ���\algorithmW\cite{Milner1978348}�𓱂��o���C
1982�N�CLuis Damas���ŏI�I��Milner�̃A���S���Y���̊��S�����ؖ�����\cite{Damas:1982:PTF:582153.582176}�D

�����݁C
Hindley-Milner�^�V�X�e�����̗p���Ă���nibkame�ł͌^�͈ȉ��̗l�ɒ�`�����D
\begin{equation}\label{eq:type} 
\begin{array}{rcll}
    \tau    &\Coloneqq& \iota                       & \text{�^�萔 (\refSsc{ssc:type-constant})} \\
            &\mid&  \alpha                          & \text{�^�ϐ� (\refSsc{ssc:type-variable})} \\
            &\mid&  \tau \rightarrow \tau           & \text{�֐��^ (\refSsc{ssc:type-function})} \\
            &\mid&  \tau \times \cdots \times \tau  & \text{���ό^ (\refSsc{ssc:type-tuple})} \\
            &\mid&  \tau\ \textsf{ref}              & \text{�Q�ƌ^ (\refSsc{ssc:type-reference})} \\
            &\mid&  \tau\ \tau                      & \text{�^�K�p (\refSsc{ssc:type-apply})} \\
    \mathset{Type}&\triangleq& \varrange{\tau}            & \text{�^} \\
    \sigma  &\Coloneqq &\tau                            & \text{�^} \\
            &\mid&  \foralldot{\vec{\alpha}}{\tau}      & \text{�S�̗ʉ�}\\
    \mathset{TypeScheme}&\triangleq& \varrange{\sigma}            & \text{�^�X�L�[��}\\
    \mathset{TypeVar}&\triangleq& \varrange{\alpha}            & 
\end{array}
\end{equation}

\subsection{�^�萔}\label{ssc:type-constant}
nibkame�ɂ����Č^�萔�͎��̗l�ɒ�`�����D
\begin{equation}\label{eq:type-constant} 
\begin{array}{rcll}
    \iota   &\Coloneqq& \typename{int}                  & \text{�����^} \\
            &\mid&  \typename{float}                & \text{�����^} \\
            &\mid&  \typename{char}                 & \text{char�^} \\
            &\mid&  \typename{list}                 & \text{���X�g�^} \\
            &\mid&  \typename{array}                & \text{�z��^} \\
            &\mid&  \mathit{variant}                & \text{���@���A���g�^} \\
    \mathset{TypeConstant}&\triangleq& \varrange{\iota}            & \text{�^�萔}\\
    \mathset{TypeVariant}&\triangleq& \varrange{\mathit{variant}}  & \text{���@���A���g�^}
\end{array}
\end{equation}

�ȏ�̂����C\typename{int}��\typename{float}�͐��l��\���^�ł���C
�����I�ɂ͕v�X�}�V���ɂ���������Ƃ����R�Ȑ�����\���\��\jpdash �Ⴆ�΃��W�X�^�ɂ�鐮���\��\jpdash �ƁC
�����Ƃ����R�ȕ��������_����\���\��\jpdash �Ⴆ��IEEE�̕��������_��\jpdash ���p������D
\typename{char}��1�o�C�g�ɂ���ĕ\���ł��镄���Ȃ���
\jpdash �T�^�I�ɂ�$0$����$2^8-1$\jpdash 
��\���^�ł���D

$\typename{list}$��$\typename{array}$�C���@���A���g�^���^�萔�ƂȂ��Ă���̂Ɉ�a�����o������������邾�낤�����R��
\refSsc{ssc:type-apply}�ŏڐ�����D

\subsection{�^�ϐ�}\label{ssc:type-variable}
�^�ϐ��Ƃ͌^�𓮂��ϐ��ł���D

\subsection{�֐��^}\label{ssc:type-function}
�֐��^�Ƃ͂��̖��̒ʂ�֐������Œ^�ł���C
$s$�^���Ƃ���$t$�^��Ԃ��֐��̌^��$s\rightarrow t$�ƂȂ�D

\subsection{���ό^}\label{ssc:type-tuple}
���ό^�͑g�̌^�ł���C�g�̕v�X�̗v�f�̌^�̑g�ƂȂ��Ă���D

\subsection{�Q�ƌ^}\label{ssc:type-reference}
�Q�Ƃ����Œ^�ł���C�Ⴆ�΁C\lstinline|int|�^�ւ̎Q�Ƃł���΁C���̌^��
\lstinline|int ref|�ƂȂ�D

\subsection{�^�K�p}\label{ssc:type-apply}
ML�n����̌^�ɂ́C
$\typename{int}\ \typename{list}$��C
$(\typename{int}\times\typename{char}\ \typename{array})\ \typename{list}$
���̌^�����݂���
\footnote{
    Haskell�CMiranda�ł͂��ꂼ��
    $[\typename{Int}]$�C
    $[(\typename{Int}, \typename{Array}\ \typename{Char})]$
    �ƂȂ�D
    }
�D
����͕v�X�����^�̃��X�g�C������(char�^�̔z��)�̑g�̃��X�g�Ƃ����Ӗ��ł���D

��L�̗���킩��Ƃ���C\typename{list}��\typename{array}�Ȃǂ͌^�������Ƃ��Ď��^�ł���D
������������邽�߂ɁC�������^������Č^��Ԃ��֐��̗l�Ɉ����D
�A���CML�n�֐��̋L�@�ƍ��킹�邽�߂ɁC�֐��ƈ����̕��т��ʏ�Ƌt�ɂȂ�D

\subsection{���@���A���g�^}\label{ssc:type-variant}
�㐔�I�f�[�^�^�Ƃ������D����̓��[�U���V������`�ł���^�ł���D���@���A
���g�͒�`�̂��тɂ܂������Ɨ������V�����^�ƂȂ�D
���@���A���g�^�̒l�͌^�\�z�q�ɂ���Đ��������D

���Ƃ��΁C
\begin{lstlisting}
type season = Spring | Summer | Fall | Winter
\end{lstlisting}
�Ƃ����
\lstinline|Spring|�C\lstinline|Summer|�C\lstinline|Fall|�C\lstinline|Winter|
���^�\�z�q�Ƃ��郔�@���A���g�^\lstinline|season|����`�����D

�܂��e�^�\�z�q�͒l��������邱�Ƃ��ł��C���@���A���g�^�͌^��������邱��
���ł���D�����p����
\begin{lstlisting}
type 'a tree = Node of 'a tree * 'a * 'a tree | Leaf of 'a
\end{lstlisting}
�̂悤�ɓ񕪖؂Ȃǂ��`���邱�Ƃ��ł���D

\subsection{����}\label{ssc:example}
1��3.14�C'c'�͂��ꂼ��萔�^
\typename{int}�C\typename{float}�C\typename{char}�ł���D

������1�Ž��C�㑱�̐�����Ԃ��֐�add1���l����D����Ƃ��̌^��
\lstinline|int -> int|�ƂȂ�D

�C�ӂ̗v�f��2�‚̑g�������Ɏ��C1�‚߂̒l��Ԃ��֐�fst���l����D������
�͔C�ӂ̌^�����邩��C\lstinline|(int, char) -> int|�C
\lstinline|(float array, int) -> float array|�Ȃǂ��l������D�����Ƃ�
��ʓI�Ȍ^��\lstinline|('a, 'b) -> 'a|�ł���D


\subsection{�^�ϐ��ւ̑��}\label{ssc:type-order}
$\tau$�Ɋ܂܂�邢���‚��̌^�ϐ��ɑ΂��Ēu��$\substType$��K�p���邱�ƂŁC
$\Sigma\tau=\tau'$�ƂȂ�Ƃ��C
$\tau\succ\tau'$�Ƃ����D
����$\succ$�͔������ƂȂ�D
$\tau\succ\tau'$����$\tau'\succ\tau$�̂Ƃ��^$\tau$�C$\tau'$��$\alpha$���l�Ƃ����C
$\tau\mathrel{\overset{\alpha}{=}}\tau'$�ƕ\���D





\section{意味論}\label{sc:semantics}
まず,変数や型変数,型構築子の型の環境は次の様に定義される.
\begin{equation}\label{eq:env} 
\begin{array}{rcll}
    \envExpr  &\subset& \mathset{Var}\times\mathset{TypeScheme} & \text{変数の型環境} \\
    \envType  &\subset& \mathset{TypeVar}\times\mathset{TypeScheme} & \text{型変数の型環境} \\
    \envVariant  &\subset& \mathset{TypeVariant}\times\powerset(\mathset{Constructor}\times\mathset{TypeScheme}) & \text{ヴァリアントの型環境} \\
\end{array}
\end{equation}
%置換の集合は次のように定義される.
%\begin{equation}\label{eq:subst} 
%\begin{array}{rcll}
%    \substExpr  &\subset& \mathset{Var}\times\mathset{Expr} & \text{変数の置換} \\
%    \substType  &\subset& \mathset{TypeVar}\times\mathset{Type} & \text{型変数の置換} \\
%\end{array}
%\end{equation}
%置換の集合については特別に$\{(v,e),\cdots\}$を$[e/v,\cdots]$とも書く.
式$e$,型$\tau$の中に現れる自由変数の集合は夫々$\freevars{e}$,$\freetypevars{\tau}$と書く.

%式に対する置換の適用$\applysubst{\Sigma}{x}$は次のように定義される.
%\begin{equation}
%\begin{aligned}
%    \applysubst{\substExpr}{c} &= c \\
%    \applysubst{\substExpr}{s:\tau} &= s:\tau \\
%    \applysubst{\substExpr}{x} &=\begin{cases}
%        e & \text{if $(v,e)\in\substExpr$.} \\
%        x & \text{otherwise.} \end{cases} \\
%    \applysubst{\substExpr}{\fundot{x}{e}} &= \applysubst{\substExpr'}{e} &
%            \text{which $\substExpr' = \domainrestrict{\substExpr}{(\domain\substExpr\setminus\{x\})}$.}\\
%\end{aligned}
%\end{equation}

\subsection{�^�t��}\label{sc:typing}
�萔�̌^��Ԃ��֐�$\mathfunc{ConstType}:\mathset{ExprConst}\rightarrow\mathset{TypeConst}$�͎��̗l�ɒ�`�����D
\begin{equation}\label{eq:func-const-type} 
    \mathfunc{ConstType}(c) = \begin{cases}
        \typename{unit} & \text{if $c$ is $\mathnkop{()}$.} \\
        \typename{bool} & \text{if $c$ is $\mathnkop{true}$ or $\mathnkop{false}$.} \\
        \typename{int} & \text{if $c$ is integer.} \\
        \typename{float} & \text{if $c$ is float.} \\
        \typename{char} & \text{if $c$ is char.} \\
        \alpha\ \typename{list} & \text{if $c$ is $\mathnkop{Nil}$. And $\alpha$ is fresh.}
        \end{cases}
\end{equation}

�p�^�[���ɂ�鑩���̌^�‹�$\patternandenv{p:\tau}{\envPattern}$�͈ȉ��̗l�ɒ�`�����D
\begin{subequations}
\begin{equation}
    \infer[\rulename{P-All}]
        {\envExpr\vdash \patternandenv{\_:\tau}{\emptyset}}
        {}\\
\end{equation}
\begin{equation}
    \infer[\rulename{P-Not}]
        {\envExpr\vdash \patternandenv{\lnot p:\tau}{\emptyset}}
        {\envExpr\vdash \patternandenv{p:\tau}{\emptyset}}\\
\end{equation}
\begin{equation}
    \infer[\rulename{P-Const}]
        {\envExpr\vdash \patternandenv{c:\tau}{\emptyset}}
        {\mathfunc{ConstType}(c) \succ \tau} \\
\end{equation}
\begin{equation}
    \infer[\rulename{P-Var}]
        {\envExpr\vdash \patternandenv{x:\tau}{\{x:\tau\}}}
        {x\in\mathset{ExprVar}}\\
\end{equation}
\begin{equation}
    \infer[\rulename{P-Constructor}]
        {\envExpr\vdash \patternandenv{g:\tau}{\emptyset}}
        {g\in\mathset{Constructor} & (g,\tau)\in\envExpr & \tau_1\rightarrow\tau_2\not\succ\tau}\\
\end{equation}
\begin{equation}
    \infer[\rulename{P-Apply}]
        {\envExpr\vdash \patternandenv{g\ p:\tau}{\envPattern}}
        {g\in\mathset{Constructor} & (g,\tau'\rightarrow\tau )\in\envExpr & \envExpr\vdash \patternandenv{p:\tau'}{\envPattern} }\\
\end{equation}
\begin{equation}
    \infer[\rulename{P-Tuple}]
        {\envExpr\vdash \patternandenv{e_0,\cdots,e_n:\tau_0\times\cdots\times\tau_n}{\bigcup_i \envPattern_i}}
        {\envExpr\vdash\patternandenv{p_0:\tau_0}{\envPattern_0} &
        \cdots &
        \envExpr\vdash\patternandenv{p_n:\tau_n}{\envPattern_n} &
        \foralldot{i,j}{i\not= j\limpl (\domain\envPattern_i \cap \domain\envPattern_j = \emptyset)}}\\
\end{equation}
\begin{equation}
    \infer[\rulename{P-And}]
        {\envExpr\vdash \patternandenv{p_1 \patternand p_2:\tau}{\envPattern_1\cap\envPattern_2}}
        {\envExpr\vdash \patternandenv{p_1:\tau}{\envPattern_1} &
        \envExpr\vdash \patternandenv{p_2:\tau}{\envPattern_2} &
        \domain\envPattern_1 \cap \domain\envPattern_2 = \emptyset
        }\\
\end{equation}
\begin{equation}
    \infer[\rulename{P-Or}]
        {\envExpr\vdash \patternandenv{p_1 \patternor p_2:\tau}{\envPattern}}
        {\envExpr\vdash \patternandenv{p_1:\tau}{\envPattern} &
        \envExpr\vdash \patternandenv{p_2:\tau}{\envPattern}}
\end{equation}
\end{subequations}
���̎��Œ^�͈ȉ��̗l�ɒ�`�����D
�A���C�f��̖�������
%$i,n\in\N$�C
$c\in\mathset{ExprConst}$, 
$x,v\in\mathset{ExprVar}$�C
$e\in\mathset{Expr}$
�Ƃ���D
\begin{subequations}
\begin{equation}
    \infer[\rulename{E-Const}]
        {\envExpr\vdash c:\tau}
        {\mathfunc{ConstType}(c) \succ \tau} \\
\end{equation}
\begin{equation}
    \infer[\rulename{E-Var}]
        {\envExpr\vdash v:\tau}
        {(v,\tau')\in\envExpr & \tau' \succ \tau}\\
\end{equation}
\begin{equation}
    \infer[\rulename{E-Fun}]
        {\envExpr\vdash \fundot{x}{e}:\tau_1\rightarrow\tau_2}
        {\envExpr\uplus(x,\tau_1)\vdash e:\tau_2}\\
\end{equation}
\begin{equation}
    \infer[\rulename{E-Tuple}]
        {\envExpr\vdash e_0,\cdots,e_n:\tau_0\times\cdots\times\tau_n}
        {\envExpr\vdash e_0:\tau_0 &
        \cdots &
        \envExpr\vdash e_n:\tau_n}\\
\end{equation}
\begin{equation}
    \infer[\rulename{E-MakeRef}]
        {\envExpr\vdash \mathnkop{ref}e:\tau\ \mathsf{ref}}
        {\envExpr\vdash e:\tau}\\
\end{equation}
\begin{equation}
    \infer[\rulename{E-Ref}]
        {\envExpr\vdash e!:\tau}
        {\envExpr\vdash e:\tau\ \mathsf{ref}}\\
\end{equation}
\begin{equation}
    \infer[\rulename{E-SetRef}]
        {\envExpr\vdash e_1\coloneqq e_2:\typename{unit}}
        {\envExpr\vdash e_1:\tau\ \mathsf{ref} &
        \envExpr\vdash e_2:\tau}\\
\end{equation}
\begin{equation}
    \infer[\rulename{E-Apply}]
        {\envExpr\vdash e_1\ e_2:\tau}
        {\envExpr\vdash e_1:\tau'\rightarrow\tau &
        \envExpr\vdash e_2:\tau'} \\
\end{equation}
\begin{equation}
    \infer[\rulename{E-Let}]
        {\envExpr\vdash \mathnkop{let}x=e_1\mathnkop{in}e_2:\tau}
        {\envExpr\vdash e_1:\tau' &
        \envExpr\uplus(x,\tau')\vdash e_2:\tau}\\
\end{equation}
%\begin{equation}
%    \infer[\rulename{E-If}]
%        {\envExpr\vdash \mathnkop{if}e_1\mathnkop{then}e_2\mathnkop{else}e_3:\tau}
%        {\envExpr\vdash e_1:\typename{bool} &
%        \envExpr\vdash e_2:\tau &
%        \envExpr\vdash e_3:\tau}\\
%\end{equation}
\begin{equation}
    \infer[\rulename{E-Fix}]
        {\envExpr\vdash \fixdot{f}e:\tau}
        {\envExpr\uplus(f,\tau)\vdash e:\tau}\\
\end{equation}
\begin{equation}
    \infer[\rulename{E-Match}]
        {\envExpr\vdash \mathnkop{match}e\mathnkop{with}\mathit{clss}:\tau}
        {\envExpr\vdash e:\tau' &
        \envExpr\vdash \mathit{clss}:\tau'\rightarrow\tau}
\end{equation}
\end{subequations}
�߂̌^�͈ȉ��̗l�ɒ�`�����D
\begin{subequations}
\begin{equation}
    \infer[\rulename{C-Always}]
        {\envExpr\vdash p\rightarrow e:\tau'\rightarrow\tau}
        {\envExpr\vdash \patternandenv{p:\tau'}{\envPattern} &
        \envExpr\uplus\envPattern\vdash e:\tau}\\
\end{equation}
\begin{equation}
    \infer[\rulename{C-Guarded}]
        {\envExpr\vdash p\mathnkop{with}e_g\rightarrow e:\tau'\rightarrow\tau}
        {\envExpr\vdash \patternandenv{p:\tau'}{\envPattern} &
        \envExpr\uplus\envPattern\vdash e_g:\typename{bool} &
        \envExpr\uplus\envPattern\vdash e:\tau}\\
\end{equation}
\begin{equation}
    \infer[\rulename{C-Seq}]
        {\envExpr\vdash \mathit{cls}_0\clauseor\cdots\clauseor\mathit{cls}_n:\tau}
        {\envExpr\vdash cls_0:\tau &
        \cdots &
        \envExpr\vdash cls_n:\tau}
\end{equation}
\end{subequations}





\section{nibkame追加構文}\label{sc;additional-syntax}
以上で定義したnibkameコア言語に
if文などの糖衣構文(syntax sugar)を追加し,nibkame言語とする.

構文上の変換規則$\mathcal{A}$は以下のように定義される.
\begin{equation}
\begin{aligned}
    \semanticA{\mathnkop{if}e_c\mathnkop{then}e_\top\mathnkop{else}e_\bot} &=
        \mathnkop{match}\semanticA{e_c}\mathnkop{with}
            \mathnkval{true}\rightarrow \semanticA{e_\top}
            \clauseor \mathnkval{false}\rightarrow \semanticA{e_\bot} \\
    \semanticA{e} &= e \qquad\text{if $e$ is not matched in the definitions above.}
\end{aligned}
\end{equation}


\section{�W�����C�u����}
nibkame����ŕW���Ŏg�p�ł���֐��̎d�l�ɂ‚��ďq�ׂ�D

�����ŏq�ׂ�֐��̓\�[�X�R�[�h�̓ǂݍ��݈ȑO�����`����Ă���C�����`
�����֐��Ɠ����悤�Ɏg�p���邱�Ƃ��ł���D

\subsection{�Z�p���Z}
�g�p�ł���Z�p���Z�q�Ƃ��̌^�����Ɏ����D���ʂň͂܂ꂽ���ʎq�͒��u���Z�q
�ł��邱�Ƃ������D���ꂼ��̗D�揇�ʂ͕��i�̌v�Z�Ɠ����ł���D
\begin{lstlisting}
 (+) : int -> int -> int
 (-) : int -> int -> int
 ( * ) : int -> int -> int
 (/) : int -> int -> int
 (+.) : float -> float -> float
 (-.) : float -> float -> float
 ( *. ) : float -> float -> float
 (/.) : float -> float -> float
\end{lstlisting}

\subsection{��r���Z�q}
�g�p�ł����r���Z�q�Ƃ��̌^�����Ɏ����D���ꂼ��̉��Z�q�͎Z�p���Z�q���
�D�揇�ʂ͒Ⴂ�D
\begin{lstlisting}
 (=) : 'a -> 'a -> bool
 (<>) : 'a -> 'a -> bool
 (<=) : 'a -> 'a -> bool
 (<) : 'a -> 'a -> bool
 (>) : 'a -> 'a -> bool
 (>=) : 'a -> 'a -> bool
\end{lstlisting}
������r�͐����ƕ��������_���C�����^�݂̂ɒ�`����C�^�v����z��ɑ΂���
�͖���`�ł���D

\subsection{����p�������֐�}
�������v�Z�ɗp���Ȃ����Ƃ𖾎��I�Ɏ������߁C���̊֐�����`����Ă���D
\begin{lstlisting}
 ignore : 'a -> unit
\end{lstlisting}

\subsection{�W���o�͊֐�}
���̊֐��́C���ꂼ��̈������\����͊�֓��͉”\�Ȍ`���ň󎚂���֐��ł�
��D
\begin{lstlisting}
 print_int : int -> unit
 print_float : float -> unit
 print_char : char -> unit
 print_string : char array -> unit
\end{lstlisting}

\subsection{���X�g�����֐�}
���X�g�^�̒l����������֐������Ɏ����D
\begin{lstlisting}
 hd : 'a list -> 'a
 tl : 'a list -> 'a list
 null : 'a list -> bool
 map : ('a -> 'b) -> 'a list -> 'b list
 length : 'a list -> int
\end{lstlisting}
\lstinline|hd|��\lstinline|tl|�́C���X�g����擪�̗v�f���㑱�̃�
�X�g�����o���֐��ł���D��̃��X�g��K�p�����ꍇ�̓���͖���`�ł���D
\lstinline|null|�̓��X�g�̖����ł��邩���肷��֐��ł���D
\lstinline|map|�́C��2�����̃��X�g�̊e�v�f�ɑ�1�����̊֐���K�p���C����
�����X�g�ɂ��ĕԂ��֐��ł���D
\lstinline|length|�̓��X�g�̒��������߂�֐��ł���D

\subsection{�z�񏈗��֐�}
�z��𑀍삷��֐������Ɏ����D�z��̍X�V�͔j��I�ł��邱�Ƃɒ��ӂ���K�v
������D
\begin{lstlisting}
 array-create : 'a -> int -> 'a array
 array-set : 'a array -> 'a -> int -> unit
 array-ref : 'a array -> int -> 'a
 array-from-list : 'a list -> 'a array
 array-from-list-with-length : int -> 'a list -> 'a array
\end{lstlisting}
\lstinline|array-create|�͑�1�����������l�Ƃ����C��������2���������̔z��
��V���ɐ�������֐��ł���D
\lstinline|array-set|�͔z��̎w�肵���v�f��j��I�ɍX�V����֐��ł���D
\lstinline|array-ref|�͔z��̎w�肵���v�f�����o���֐��ł���D
\lstinline|array-from-list|�̓��X�g�Ɠ��������œ����v�f�����”z��𐶐�
����֐��ł���D
\lstinline|array-from-list-with-length|�̓��X�g�̓��C�w�肳�ꂽ�����̐�
���v�f�����”z��𐶐�����D���X�g��蒷���l���w�肳�ꂽ�Ƃ��C����ȍ~��% �{���ɁH
�l�͖���`�ł���D




%\chapter*{謝辞}
\addcontentsline{toc}{chapter}{謝辞}
\section*{謝辞}
樋口先生にはプロジェクト計画についてご意見をいただきました.大墳先生には発表
資料の作成や発表についてご指導いただきました.牛田先生には予稿などの書法につ
いて助言をいただきました.深謝の意を表します.

\bibliographystyle{jplain}
\bibliography{reference}

\backmatter
\appendix

\end{document}
