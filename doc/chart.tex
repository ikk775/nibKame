\documentclass[a4paper, 8pt, landscape]{jsarticle}
\usepackage{ascmac}
\usepackage[osf]{mathpazo}
\usepackage{pst-gantt}
\pagestyle{myheadings}
\setlength{\evensidemargin}{0.7in}
\setlength{\oddsidemargin}{0.7in}

\begin{document}
\newpsstyle{Important}{fillstyle=solid,fillcolor=red}
\newpsstyle{NotImportant}{fillstyle=vlines}
\begin{PstGanttChart}[unit=1,yunit=2,ChartUnitIntervalName=Month,ChartUnitBasicIntervalName=Day,TaskUnitIntervalValue=30,TaskUnitType=Month,ChartShowIntervals]{14}{5}
\PstGanttTask[TaskUnitType=Day,TaskInsideLabel={小},TaskOutsideLabel={型推論部の作成}]{16}{14}
\PstGanttTask[TaskUnitType=Day,TaskInsideLabel={酒},TaskOutsideLabel={中間言語への変換部の作成}]{16}{7}
%\PstGanttTask[TaskUnitType=Day,TaskStyle=Important,TaskOutsideLabel={Task 3},
%TaskInsideLabel={\Large\textcolor{white}{\textbf{Important}}}]{2}{5}
%\PstGanttTask[TaskUnitType=Day,TaskStyle=NotImportant,TaskOutsideLabel={Task 4}]{4}{2}
\PstGanttTask[TaskUnitType=Day,TaskInsideLabel={酒},TaskOutsideLabel={正規化部の作成}]{23}{14}
\PstGanttTask[TaskUnitType=Day,TaskInsideLabel={酒},TaskOutsideLabel={最適化部の作成}]{37}{14}
\PstGanttTask[TaskUnitType=Day,TaskInsideLabel={酒},TaskOutsideLabel={仮想機械語への変換部の作成}]{51}{7}
\PstGanttTask[TaskUnitType=Day,TaskInsideLabel={小},TaskOutsideLabel={仮想機械語最適化部の作成}]{51}{7}
\PstGanttTask[TaskUnitType=Day,TaskInsideLabel={酒},TaskOutsideLabel={機械語への変換部の作成}]{58}{7}
\PstGanttTask[TaskUnitType=Day,TaskInsideLabel={小・酒},TaskStyle=Important,TaskOutsideLabel={基本仕様のデバッグ}]{65}{14}
\PstGanttTask[TaskUnitType=Day,TaskInsideLabel={小・酒},TaskOutsideLabel={標準ライブラリの作成}]{30}{21}
\PstGanttTask[TaskUnitType=Day,TaskInsideLabel={小},TaskOutsideLabel={FFIの作成}]{58}{14}
\PstGanttTask[TaskUnitType=Day,TaskInsideLabel={酒},TaskOutsideLabel={構文木変換部の作成}]{72}{7}
\PstGanttTask[TaskUnitType=Day,TaskInsideLabel={小},TaskOutsideLabel={字句・構文解析部の作成}]{79}{7}
\PstGanttTask[TaskUnitType=Day,TaskInsideLabel={小・酒},TaskStyle=NotImportant,TaskOutsideLabel={GCの作成}]{86}{30}
\PstGanttTask[TaskUnitType=Day,TaskInsideLabel={小・酒},TaskStyle=Important,TaskOutsideLabel={全体的なデバッグ}]{86}{30}

\end{PstGanttChart}
\end{document}

