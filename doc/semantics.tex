\section{意味論}\label{sc:semantics}
まず,変数や型変数,型構築子の型の環境は次の様に定義される.
\begin{equation}\label{eq:env} 
\begin{array}{rcll}
    \envExpr  &\subset& \mathset{Var}\times\mathset{TypeScheme} & \text{変数の型環境} \\
    \envType  &\subset& \mathset{TypeVar}\times\mathset{TypeScheme} & \text{型変数の型環境} \\
    \envVariant  &\subset& \mathset{TypeVariant}\times\powerset(\mathset{Constructor}\times\mathset{TypeScheme}) & \text{ヴァリアントの型環境} \\
\end{array}
\end{equation}
%置換の集合は次のように定義される.
%\begin{equation}\label{eq:subst} 
%\begin{array}{rcll}
%    \substExpr  &\subset& \mathset{Var}\times\mathset{Expr} & \text{変数の置換} \\
%    \substType  &\subset& \mathset{TypeVar}\times\mathset{Type} & \text{型変数の置換} \\
%\end{array}
%\end{equation}
%置換の集合については特別に$\{(v,e),\cdots\}$を$[e/v,\cdots]$とも書く.
式$e$,型$\tau$の中に現れる自由変数の集合は夫々$\freevars{e}$,$\freetypevars{\tau}$と書く.

%式に対する置換の適用$\applysubst{\Sigma}{x}$は次のように定義される.
%\begin{equation}
%\begin{aligned}
%    \applysubst{\substExpr}{c} &= c \\
%    \applysubst{\substExpr}{s:\tau} &= s:\tau \\
%    \applysubst{\substExpr}{x} &=\begin{cases}
%        e & \text{if $(v,e)\in\substExpr$.} \\
%        x & \text{otherwise.} \end{cases} \\
%    \applysubst{\substExpr}{\fundot{x}{e}} &= \applysubst{\substExpr'}{e} &
%            \text{which $\substExpr' = \domainrestrict{\substExpr}{(\domain\substExpr\setminus\{x\})}$.}\\
%\end{aligned}
%\end{equation}

\subsection{�^�t��}\label{sc:typing}
�萔�̌^��Ԃ��֐�$\mathfunc{ConstType}:\mathset{ExprConst}\rightarrow\mathset{TypeConst}$�͎��̗l�ɒ�`�����D
\begin{equation}\label{eq:func-const-type} 
    \mathfunc{ConstType}(c) = \begin{cases}
        \typename{unit} & \text{if $c$ is $\mathnkop{()}$.} \\
        \typename{bool} & \text{if $c$ is $\mathnkop{true}$ or $\mathnkop{false}$.} \\
        \typename{int} & \text{if $c$ is integer.} \\
        \typename{float} & \text{if $c$ is float.} \\
        \typename{char} & \text{if $c$ is char.} \\
        \alpha\ \typename{list} & \text{if $c$ is $\mathnkop{Nil}$. And $\alpha$ is fresh.}
        \end{cases}
\end{equation}

�p�^�[���ɂ�鑩���̌^�‹�$\patternandenv{p:\tau}{\envPattern}$�͈ȉ��̗l�ɒ�`�����D
\begin{subequations}
\begin{equation}
    \infer[\rulename{P-All}]
        {\envExpr\vdash \patternandenv{\_:\tau}{\emptyset}}
        {}\\
\end{equation}
\begin{equation}
    \infer[\rulename{P-Not}]
        {\envExpr\vdash \patternandenv{\lnot p:\tau}{\emptyset}}
        {\envExpr\vdash \patternandenv{p:\tau}{\emptyset}}\\
\end{equation}
\begin{equation}
    \infer[\rulename{P-Const}]
        {\envExpr\vdash \patternandenv{c:\tau}{\emptyset}}
        {\mathfunc{ConstType}(c) \succ \tau} \\
\end{equation}
\begin{equation}
    \infer[\rulename{P-Var}]
        {\envExpr\vdash \patternandenv{x:\tau}{\{x:\tau\}}}
        {x\in\mathset{ExprVar}}\\
\end{equation}
\begin{equation}
    \infer[\rulename{P-Constructor}]
        {\envExpr\vdash \patternandenv{g:\tau}{\emptyset}}
        {g\in\mathset{Constructor} & (g,\tau)\in\envExpr & \tau_1\rightarrow\tau_2\not\succ\tau}\\
\end{equation}
\begin{equation}
    \infer[\rulename{P-Apply}]
        {\envExpr\vdash \patternandenv{g\ p:\tau}{\envPattern}}
        {g\in\mathset{Constructor} & (g,\tau'\rightarrow\tau )\in\envExpr & \envExpr\vdash \patternandenv{p:\tau'}{\envPattern} }\\
\end{equation}
\begin{equation}
    \infer[\rulename{P-Tuple}]
        {\envExpr\vdash \patternandenv{e_0,\cdots,e_n:\tau_0\times\cdots\times\tau_n}{\bigcup_i \envPattern_i}}
        {\envExpr\vdash\patternandenv{p_0:\tau_0}{\envPattern_0} &
        \cdots &
        \envExpr\vdash\patternandenv{p_n:\tau_n}{\envPattern_n} &
        \foralldot{i,j}{i\not= j\limpl (\domain\envPattern_i \cap \domain\envPattern_j = \emptyset)}}\\
\end{equation}
\begin{equation}
    \infer[\rulename{P-And}]
        {\envExpr\vdash \patternandenv{p_1 \patternand p_2:\tau}{\envPattern_1\cap\envPattern_2}}
        {\envExpr\vdash \patternandenv{p_1:\tau}{\envPattern_1} &
        \envExpr\vdash \patternandenv{p_2:\tau}{\envPattern_2} &
        \domain\envPattern_1 \cap \domain\envPattern_2 = \emptyset
        }\\
\end{equation}
\begin{equation}
    \infer[\rulename{P-Or}]
        {\envExpr\vdash \patternandenv{p_1 \patternor p_2:\tau}{\envPattern}}
        {\envExpr\vdash \patternandenv{p_1:\tau}{\envPattern} &
        \envExpr\vdash \patternandenv{p_2:\tau}{\envPattern}}
\end{equation}
\end{subequations}
���̎��Œ^�͈ȉ��̗l�ɒ�`�����D
�A���C�f��̖�������
%$i,n\in\N$�C
$c\in\mathset{ExprConst}$, 
$x,v\in\mathset{ExprVar}$�C
$e\in\mathset{Expr}$
�Ƃ���D
\begin{subequations}
\begin{equation}
    \infer[\rulename{E-Const}]
        {\envExpr\vdash c:\tau}
        {\mathfunc{ConstType}(c) \succ \tau} \\
\end{equation}
\begin{equation}
    \infer[\rulename{E-Var}]
        {\envExpr\vdash v:\tau}
        {(v,\tau')\in\envExpr & \tau' \succ \tau}\\
\end{equation}
\begin{equation}
    \infer[\rulename{E-Fun}]
        {\envExpr\vdash \fundot{x}{e}:\tau_1\rightarrow\tau_2}
        {\envExpr\uplus(x,\tau_1)\vdash e:\tau_2}\\
\end{equation}
\begin{equation}
    \infer[\rulename{E-Tuple}]
        {\envExpr\vdash e_0,\cdots,e_n:\tau_0\times\cdots\times\tau_n}
        {\envExpr\vdash e_0:\tau_0 &
        \cdots &
        \envExpr\vdash e_n:\tau_n}\\
\end{equation}
\begin{equation}
    \infer[\rulename{E-MakeRef}]
        {\envExpr\vdash \mathnkop{ref}e:\tau\ \mathsf{ref}}
        {\envExpr\vdash e:\tau}\\
\end{equation}
\begin{equation}
    \infer[\rulename{E-Ref}]
        {\envExpr\vdash e!:\tau}
        {\envExpr\vdash e:\tau\ \mathsf{ref}}\\
\end{equation}
\begin{equation}
    \infer[\rulename{E-SetRef}]
        {\envExpr\vdash e_1\coloneqq e_2:\typename{unit}}
        {\envExpr\vdash e_1:\tau\ \mathsf{ref} &
        \envExpr\vdash e_2:\tau}\\
\end{equation}
\begin{equation}
    \infer[\rulename{E-Apply}]
        {\envExpr\vdash e_1\ e_2:\tau}
        {\envExpr\vdash e_1:\tau'\rightarrow\tau &
        \envExpr\vdash e_2:\tau'} \\
\end{equation}
\begin{equation}
    \infer[\rulename{E-Let}]
        {\envExpr\vdash \mathnkop{let}x=e_1\mathnkop{in}e_2:\tau}
        {\envExpr\vdash e_1:\tau' &
        \envExpr\uplus(x,\tau')\vdash e_2:\tau}\\
\end{equation}
%\begin{equation}
%    \infer[\rulename{E-If}]
%        {\envExpr\vdash \mathnkop{if}e_1\mathnkop{then}e_2\mathnkop{else}e_3:\tau}
%        {\envExpr\vdash e_1:\typename{bool} &
%        \envExpr\vdash e_2:\tau &
%        \envExpr\vdash e_3:\tau}\\
%\end{equation}
\begin{equation}
    \infer[\rulename{E-Fix}]
        {\envExpr\vdash \fixdot{f}e:\tau}
        {\envExpr\uplus(f,\tau)\vdash e:\tau}\\
\end{equation}
\begin{equation}
    \infer[\rulename{E-Match}]
        {\envExpr\vdash \mathnkop{match}e\mathnkop{with}\mathit{clss}:\tau}
        {\envExpr\vdash e:\tau' &
        \envExpr\vdash \mathit{clss}:\tau'\rightarrow\tau}
\end{equation}
\end{subequations}
�߂̌^�͈ȉ��̗l�ɒ�`�����D
\begin{subequations}
\begin{equation}
    \infer[\rulename{C-Always}]
        {\envExpr\vdash p\rightarrow e:\tau'\rightarrow\tau}
        {\envExpr\vdash \patternandenv{p:\tau'}{\envPattern} &
        \envExpr\uplus\envPattern\vdash e:\tau}\\
\end{equation}
\begin{equation}
    \infer[\rulename{C-Guarded}]
        {\envExpr\vdash p\mathnkop{with}e_g\rightarrow e:\tau'\rightarrow\tau}
        {\envExpr\vdash \patternandenv{p:\tau'}{\envPattern} &
        \envExpr\uplus\envPattern\vdash e_g:\typename{bool} &
        \envExpr\uplus\envPattern\vdash e:\tau}\\
\end{equation}
\begin{equation}
    \infer[\rulename{C-Seq}]
        {\envExpr\vdash \mathit{cls}_0\clauseor\cdots\clauseor\mathit{cls}_n:\tau}
        {\envExpr\vdash cls_0:\tau &
        \cdots &
        \envExpr\vdash cls_n:\tau}
\end{equation}
\end{subequations}




\subsection{付値}\label{sc:evaluation}
以下の操作的意味論は\cite{r6rs:jfp}を参考にした.

値$\upsilon$は
\begin{equation}\label{eq:eval-value} 
\begin{array}{rcll}
\upsilon&\Coloneqq& c                          & \text{定数} \\
 &\mid&  s:\tau                     & \text{外部定数} \\
 &\mid&  v                          & \text{変数} \\
 &\mid&  \mypair{g}{\upsilon}       & \text{型構築子} \\
 &\mid&  \lambdadot{x}{e}           & \text{関数} \\
 &\mid&  e, \cdots, e               & \text{組} \\
 &\mid&  \mathnkval{undefined}      & \text{未定義}\\
\mathset{EvalVar}&\triangleq& \varrange{v} \\
\mathset{EvalData}&\triangleq& \mathset{Constructor}\times\varrange{\upsilon}
\end{array}
\end{equation}
と定義される.

環境
\begin{align}\label{eq:eval-func} 
\nu&\msubset\mathit{ExprVar}\times\mathit{EvalVar} \\
\rho&\msubset\mathit{EvalVar}\times\upsilon
\end{align}

以下の定義に登場する$\xi$は以前に使用されていない名前の変数を意味する.
\begin{subequations}
\begin{align}
    \semanticS{c} &= \lambdadot{\nu\rho}{\valueandenv{c}{\rho}}\\
    \semanticS{v} &= \lambdadot{\nu\rho}{\valueandenv{\nu(v)}{\rho}}\\
    \semanticS{\mathnkop{if}e_1\mathnkop{then}e_2\mathnkop{else}e_3} &=
        \lambdadot{\nu\rho}{\lambdaIf (\semanticSVal(\semanticS{e_1}\nu\rho) = \mathnkval{true}) \nonumber\\
            &\lambdaThen \semanticS{e_2}{\nu(\semanticSVEnv(\semanticS{e_1}\nu\rho))} \nonumber\\
            &\lambdaElse \semanticS{e_3}{\nu(\semanticSVEnv(\semanticS{e_1}\nu\rho))}} \\
    \semanticS{g} &= \lambdadot{\nu\rho}{\valueandenv{\mypair{g}{\mathnkval{undefined}}}{\rho}}\nonumber\\
    \semanticS{e_1\ e_2} &= \lambdadot{\nu\rho}{ \nonumber\\
        &\lambdaLet \omega_1 = \semanticS{e_1}\nu\rho \lambdaIn \nonumber\\
        &\lambdaIf \semanticSVal\omega_1 \in \mathset{EvalData} \nonumber\\
        &\lambdaThen \mypair{\mypair{\pi_0(\semanticSVal\omega_1)}{\semanticSVal(\semanticS{e_2}\nu(\semanticSVEnv\omega_1))}}{\semanticSVEnv(\semanticS{e_2}\nu(\semanticSVEnv\omega_1))} \nonumber\\
        &\lambdaElse \mypair{(\semanticSVal\omega_1)\ \semanticSVal(\semanticS{e_2}\nu(\semanticSVEnv\omega_1))}{\semanticSVEnv(\semanticS{e_2}\nu(\semanticSVEnv\omega_1))}
    }\nonumber\\
    \semanticS{\mathnkop{ref}e} &= \lambdadot{\nu\rho}{\valueandenv{\xi}{\semanticSVEnv(\semanticS{e}\nu\rho)\cup\mypair{\xi}{\semanticSVal(\semanticS{e}\nu\rho)}}}\\
    \semanticS{e!} &= \lambdadot{\nu\rho}{\valueandenv{\rho(\semanticSVal(\semanticS{e}\nu\rho))}{\semanticSVEnv(\semanticS{e}\nu\rho)}}\\
    \semanticS{e_1\coloneqq e_2} &= \lambdadot{\nu\rho}{ \nonumber\\
        &\lambdaLet \omega_1 = \semanticS{e_1}\nu\rho \lambdaIn \nonumber\\
        &\lambdaLet \omega_2 = \semanticS{e_2}\nu(\semanticSVEnv\omega_1) \lambdaIn \nonumber\\
        &\mypair{\text{\lstinline|()|}}{\semanticSVEnv(\omega_2)+\mypair{\semanticSVal\omega_1}{\semanticSVal\omega_2}}}\\
\end{align}
\end{subequations}


