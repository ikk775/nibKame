\subsection{付値}\label{sc:evaluation}
以下の操作的意味論は\cite{r6rs:jfp}を参考にした.

値$\upsilon$は
\begin{equation}\label{eq:eval-value} 
\begin{array}{rcll}
\upsilon&\Coloneqq& c                          & \text{定数} \\
 &\mid&  s:\tau                     & \text{外部定数} \\
 &\mid&  v                          & \text{変数} \\
 &\mid&  \mypair{g}{\upsilon}       & \text{型構築子} \\
 &\mid&  \lambdadot{x}{e}           & \text{関数} \\
 &\mid&  e, \cdots, e               & \text{組} \\
 &\mid&  \mathnkval{undefined}      & \text{未定義}\\
\mathset{EvalVar}&\triangleq& \varrange{v} \\
\mathset{EvalData}&\triangleq& \mathset{Constructor}\times\varrange{\upsilon}
\end{array}
\end{equation}
と定義される.

環境
\begin{align}\label{eq:eval-func} 
\nu&\msubset\mathit{ExprVar}\times\mathit{EvalVar} \\
\rho&\msubset\mathit{EvalVar}\times\upsilon
\end{align}

以下の定義に登場する$\xi$は以前に使用されていない名前の変数を意味する.
\begin{subequations}
\begin{align}
    \semanticS{c} &= \lambdadot{\nu\rho}{\valueandenv{c}{\rho}}\\
    \semanticS{v} &= \lambdadot{\nu\rho}{\valueandenv{\nu(v)}{\rho}}\\
    \semanticS{\mathnkop{if}e_1\mathnkop{then}e_2\mathnkop{else}e_3} &=
        \lambdadot{\nu\rho}{\lambdaIf (\semanticSVal(\semanticS{e_1}\nu\rho) = \mathnkval{true}) \nonumber\\
            &\lambdaThen \semanticS{e_2}{\nu(\semanticSVEnv(\semanticS{e_1}\nu\rho))} \nonumber\\
            &\lambdaElse \semanticS{e_3}{\nu(\semanticSVEnv(\semanticS{e_1}\nu\rho))}} \\
    \semanticS{g} &= \lambdadot{\nu\rho}{\valueandenv{\mypair{g}{\mathnkval{undefined}}}{\rho}}\nonumber\\
    \semanticS{e_1\ e_2} &= \lambdadot{\nu\rho}{ \nonumber\\
        &\lambdaLet \omega_1 = \semanticS{e_1}\nu\rho \lambdaIn \nonumber\\
        &\lambdaIf \semanticSVal\omega_1 \in \mathset{EvalData} \nonumber\\
        &\lambdaThen \mypair{\mypair{\pi_0(\semanticSVal\omega_1)}{\semanticSVal(\semanticS{e_2}\nu(\semanticSVEnv\omega_1))}}{\semanticSVEnv(\semanticS{e_2}\nu(\semanticSVEnv\omega_1))} \nonumber\\
        &\lambdaElse \mypair{(\semanticSVal\omega_1)\ \semanticSVal(\semanticS{e_2}\nu(\semanticSVEnv\omega_1))}{\semanticSVEnv(\semanticS{e_2}\nu(\semanticSVEnv\omega_1))}
    }\nonumber\\
    \semanticS{\mathnkop{ref}e} &= \lambdadot{\nu\rho}{\valueandenv{\xi}{\semanticSVEnv(\semanticS{e}\nu\rho)\cup\mypair{\xi}{\semanticSVal(\semanticS{e}\nu\rho)}}}\\
    \semanticS{e!} &= \lambdadot{\nu\rho}{\valueandenv{\rho(\semanticSVal(\semanticS{e}\nu\rho))}{\semanticSVEnv(\semanticS{e}\nu\rho)}}\\
    \semanticS{e_1\coloneqq e_2} &= \lambdadot{\nu\rho}{ \nonumber\\
        &\lambdaLet \omega_1 = \semanticS{e_1}\nu\rho \lambdaIn \nonumber\\
        &\lambdaLet \omega_2 = \semanticS{e_2}\nu(\semanticSVEnv\omega_1) \lambdaIn \nonumber\\
        &\mypair{\text{\lstinline|()|}}{\semanticSVEnv(\omega_2)+\mypair{\semanticSVal\omega_1}{\semanticSVal\omega_2}}}\\
\end{align}
\end{subequations}
