\documentclass[a4paper,12pt]{jarticle}
\usepackage{mathpazo}
\title{\LARGE 関数型言語の設計と製作}
\date{\relax}

\makeatletter
\newcounter{ienum}
%\renewcommand{\theienum}{\@arabic\c@ienum}
%\renewcommand{\theienum}{\@alph\c@ienum}
%\renewcommand{\theienum}{\@roman\c@ienum}
%\renewcommand{\theienum}{\@Alph\c@ienum}
\renewcommand{\theienum}{\@alph\c@ienum}
%\newcommand{\labelienum}{\theienum.}
%\newcommand{\labelienum}{(\theienum)}
%\newcommand{\labelienum}{\theienum.}
\newcommand{\labelienum}{\theienum.}
\let\@one=1
\newcommand{\inlineitem}[1][\relax]{%
\hspace{1ex plus 1.5ex}%
\linebreak[1]%
\addtocounter{ienum}{1}%
\textbf{\labelienum}%
\hspace{1ex plus 0.5ex}%
\ifx#1\relax\relax\else\textbf{#1}\hspace{1ex plus 1ex}\fi%
\nolinebreak}
\newenvironment{inlineenummerate}
  {\begingroup
   \setcounter{ienum}{0}%
   \let\item=\inlineitem%
   \hspace{1ex}%
   }{\endgroup\hspace{2ex plus 5ex minus 0.2ex}\linebreak[3]}%
\makeatother
\topmargin=-1in
\advance\topmargin -10mm
\advance\textheight 2in
\advance\oddsidemargin -0.5in
\advance\evensidemargin -0.5in
\advance\textwidth 1in
\begin{document}
\maketitle
%\section{はじめに}
我々は\emph{nibKame}という名の関数型言語を設計し、そのコンパイラを製作した。
nibKameとは、ML系の関数型言語である\emph{MinCaml}という言語を拡張したものである。
そもそもMinCamlはコンパイラの教育用に設計された言語であるため機能が少なく、
実用的なML系の言語に存在する
\begin{inlineenummerate}
\item[多相関数]
\item[関数の部分適用]
\item[代数型データ型]
\item[モジュール]
\item[パターンマッチング]
\item[GC]
\end{inlineenummerate}
等の機能が欠けており、nibKameにおいては
多相関数、
関数の部分適用、
代数的データ型
を実装した。

%\section{nibKameの言語仕様について}
nibKameは以下の特徴を持つ。
\begin{inlineenummerate}
\item[値呼び]
\item[暗な多層型]
\item[非純粋な関数型言語]
\item[Hindley/Milner-like type system]
\item[モジュール]
\end{inlineenummerate}

%\section{nibKameのコンパイラの仕様について}
コンパイラは以下のフェーズに分かれている。
\begin{inlineenummerate}
\item[構文解析] ソースコードを読み込み抽象構文木に変換。
\item[抽象構文木の読み込み] 
\item[型推論] 
\item[パターンマッチの展開] 
\item[K正規形に変換] 
\item[K正規形に対する最適化] 
\item[クロージャ変換] 
\item[仮想機械語に変換] 
\item[直線変換] 木構造となっている仮想機械語の命令を線形リストに変換する。
\item[機械語に変換] レジスタが無限にあるとして機械語に変換。
\item[レジスタ割り当て] 実際にレジスタを割り当てる。
\item[アセンブリで出力] 
\end{inlineenummerate}
また、その後GNUアセンブラによって出力したアセンブリをアセンブルし、
GNUリンカで事前に用意したランタイムとリンクする。

現状では、構文解析部はScheme (Gauche)で、それ以外はOCaml (OCaml)で書かれている。
ソースソードの言語別行数内訳は以下のようになっている。
ここで、OCamlにおいては\verb/*.ml/のファイルが実装を定義し、
\verb/*.mli/のファイルがインターフェースを定義する。
実装: 
\begin{inlineenummerate}
\item Scheme \verb/*.scm/ : 337行
\item OCaml \verb/*.ml/ : 3305行
\item OCaml \verb/*.mli/ : 646行
\end{inlineenummerate}
テスト: 
\begin{inlineenummerate}
\item OCaml \verb/*.ml/ : 1058行
\end{inlineenummerate}
\end{document}
