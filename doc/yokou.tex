\documentclass[twocolumn,a4paper,10pt]{jsarticle}
\usepackage{mathpazo}
\usepackage{url}
%\usepackage[deluxe]{otf}

\pagestyle{empty}

\title{\LARGE 関数型言語の設計と製作}
\author{L班 \quad 07317 小堀 育男 \quad 07322 酒本 典明}
\date{\relax}
\makeatletter
\newcounter{ienum}
%\renewcommand{\theienum}{\@arabic\c@ienum}
%\renewcommand{\theienum}{\@alph\c@ienum}
%\renewcommand{\theienum}{\@roman\c@ienum}
%\renewcommand{\theienum}{\@Alph\c@ienum}
\renewcommand{\theienum}{\@alph\c@ienum}
%\newcommand{\labelienum}{\theienum.}
%\newcommand{\labelienum}{(\theienum)}
%\newcommand{\labelienum}{\theienum.}
\newcommand{\labelienum}{\theienum.}
\let\@one=1
\newcommand{\inlineitem}[1][\relax]{%
\hspace{1ex plus 1.5ex}%
\linebreak[1]%
\addtocounter{ienum}{1}%
\textbf{\labelienum}%
\hspace{1ex plus 0.5ex}%
\ifx#1\relax\relax\else\textbf{#1}\hspace{1ex plus 1ex}\fi%
\nolinebreak}
\newenvironment{inlineenummerate}
  {\begingroup
   \setcounter{ienum}{0}%
   \let\item=\inlineitem%
   \hspace{1ex}%
   }{\endgroup\hspace{2ex plus 5ex minus 0.2ex}\linebreak[3]}%
\makeatother
\topmargin=-0.5in
\advance\topmargin -10mm
\advance\textheight 1in
%\advance\oddsidemargin -0.5in
%\advance\evensidemargin -0.5in
%\advance\textwidth 1in

\renewcommand{\emph}[1]{\textit{\textgt{#1}}}
\newcommand{\texthline}{\par\vskip 2pt\rule{0.4\textwidth}{0.5pt}\vskip 2pt\par}

\begin{document}
\maketitle
\section{はじめに}
教育目的のコンパイラであるMinCaml\cite{住井英二郎:2008-04-24}
というものがあり、
簡潔な仕様で非常に高い性能を実現している。
しかし、MinCamlの言語仕様にはML系言語にある
機能が複雑であるという理由で省略されている機能がある。

我々はプログラムが大規模化した際に生産性に大きな差がみられる
以下の機能を追加することとした。
\begin{inlineenummerate}
\item[\emph{多相関数}]
%\item[関数の\emph{部分適用}]
\item[\emph{代数型データ型}]
\item[\emph{モジュール}]
\item[\emph{パターンマッチング}]
%\item[\emph{Garbage Collection} (GC)]
\end{inlineenummerate}
以上の機能はMinCaml自体のコードでも多用されている。

また、MinCamlの出力する機械語はSPARCとPowerPCのものであり、
x86の機械語は対象としていないため、


\section{nibKameの言語仕様}
追加した機能について概説する
\paragraph{多相関数}
\paragraph{代数型データ型}
\paragraph{モジュール}
\paragraph{パターンマッチング}
%nibKameは以下の特徴を持つ. 
%\begin{inlineenummerate}
%\item[\emph{値呼び}]
%\item[暗な多相型]
%\item[\emph{非純粋な}関数型言語]
%\item[\emph{Hindley-Milner Type System}]
%\item[モジュール]
%\end{inlineenummerate}

\section{nibKameのコンパイラの仕様}
コンパイラはいくつかのフェーズに分かれている. 
以下, 特に改良を加えたところのみを
解説する. 
\begin{description}
\item[構文解析]
ソースコードを読み込み\emph{抽象構文木}(\emph{S式})に変換する. 
\emph{解析表現文法}(\emph{Parsing Expression Grammar}, PEG)で解析する. 
\item[抽象構文木の読み込み] 
抽象構文木(S式)を読み込み, 
%中間言語(\texttt{Syntax.t})に変換する. 
\item[\emph{カリー化}] 
複数引数の関数をカリー化する. 
%中間言語(\texttt{Syntax.t})を中間言語(\texttt{TypingExpr.t})に変換する. 
\item[\emph{型推論}] 
式に現れる変数全てに正しい型をつける. 
%中間言語(\texttt{TypingExpr.t})を中間言語(\texttt{Typing.t})に変換する. 
\item[パターンマッチの展開]
パターンマッチの網羅性を検査し, 
複雑なパターンマッチをより簡単にする. 
%Wright\footnote{Andrew K. Wright \url{wright@research.nj.nec.com}}
%の書いたScheme向けのパターンマッチングライブラリの機能を参考にした. 
%\item[K正規形に変換] 
%\item[K正規形に対する最適化] 
%\item[クロージャ変換] 
\item[仮想機械語に変換] 
\item[直線変換]
木構造となっている仮想機械語の命令を線形リストに変換する. 
\item[機械語に変換]
レジスタが無限にあるとして機械語に変換. 
\item[レジスタ割り当て]
実際に使用するレジスタを割り当てる. 
\item[アセンブリで出力] 
x86アセンブリで出力する. 
\end{description}
また, その後\emph{GNUアセンブラ}と
\emph{GNUリンカ}を用いて
実行ファイルを生成する。

%\section{新規性}
%構文木としてS式を採用し、その仕様が決められていることより、
%言語仕様としてマクロを導入しやすいこと。

\section{実装}
構文解析部は\emph{Lisp}系言語の\emph{Scheme}の処理系である
\emph{Gauche}用に書かれており, 
それ以外は\emph{Objective Caml} (O'Caml)で書かれている. 
2011年1月12日時点でのソースコードの全行数は6320行であり、
ソースソードの目的別言語別行数内訳は以下のようになっている. 

\begin{itemize}
    \item 本体
    \begin{description}
    \item[O'Caml 実装] 4175行
    \item[O'Caml インターフェース\footnote{C言語で言うところのヘッダファイルに相当}] 760行 %手動footnoteする!
    \item[Scheme] 337行
    \end{description}

    \item 単体テスト
    \begin{description}
    \item[O'Caml 実装部分] 1058行
    \end{description}
\end{itemize}

%\texthline
%キーワード:
%Objective Caml, MinCaml, 解析表現文法, パターンマッチング, 型理論.

\bibliographystyle{jplain}
\bibliography{reference}
\end{document}
