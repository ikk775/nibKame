\documentclass[a4paper,12pt]{jsarticle}
\usepackage{mathpazo}
\usepackage{url}
\usepackage[deluxe]{otf}

\pagestyle{empty}

\title{\LARGE 関数型言語の設計と製作}
\author{L班 \quad 07317 小堀 育男 \quad 07322 酒本 典明}
\date{\relax}
\makeatletter
\newcounter{ienum}
%\renewcommand{\theienum}{\@arabic\c@ienum}
%\renewcommand{\theienum}{\@alph\c@ienum}
%\renewcommand{\theienum}{\@roman\c@ienum}
%\renewcommand{\theienum}{\@Alph\c@ienum}
\renewcommand{\theienum}{\@alph\c@ienum}
%\newcommand{\labelienum}{\theienum.}
%\newcommand{\labelienum}{(\theienum)}
%\newcommand{\labelienum}{\theienum.}
\newcommand{\labelienum}{\theienum.}
\let\@one=1
\newcommand{\inlineitem}[1][\relax]{%
\hspace{1ex plus 1.5ex}%
\linebreak[1]%
\addtocounter{ienum}{1}%
\textbf{\labelienum}%
\hspace{1ex plus 0.5ex}%
\ifx#1\relax\relax\else\textbf{#1}\hspace{1ex plus 1ex}\fi%
\nolinebreak}
\newenvironment{inlineenummerate}
  {\begingroup
   \setcounter{ienum}{0}%
   \let\item=\inlineitem%
   \hspace{1ex}%
   }{\endgroup\hspace{2ex plus 5ex minus 0.2ex}\linebreak[3]}%
\makeatother
\topmargin=-1in
\advance\topmargin -10mm
\advance\textheight 2in
%\advance\oddsidemargin -0.5in
%\advance\evensidemargin -0.5in
%\advance\textwidth 1in

\renewcommand{\emph}[1]{\textit{\textgt{#1}}}
\newcommand{\texthline}{\par\vskip 2pt\rule{0.4\textwidth}{0.5pt}\vskip 2pt\par}

\begin{document}
\maketitle
%\section{はじめに}
我々は\emph{nibKame}という名の
\emph{関数型言語}を設計し, 
そのコンパイラを製作した. 
nibKameとは, 
\emph{ML}系の関数型言語である
\emph{MinCaml}という言語を拡張したものである. 
そもそもMinCamlはコンパイラの教育用に設計された言語であるため機能が少なく, 
実用的なML系の言語に存在する
\begin{inlineenummerate}
\item[\emph{多相関数}]
\item[関数の\emph{部分適用}]
\item[\emph{代数型データ型}]
\item[\emph{モジュール}]
\item[\emph{パターンマッチング}]
\item[\emph{Garbage Collection} (GC)]
\end{inlineenummerate}
等の機能が欠けており, nibKameにおいては
多相関数, 
関数の部分適用, 
代数的データ型
を実装した. 

%\section{nibKameの言語仕様について}
nibKameは以下の特徴を持つ. 
\begin{inlineenummerate}
\item[\emph{値呼び}]
\item[暗な多相型]
\item[\emph{非純粋な}関数型言語]
\item[\emph{Hindley-Milner Type System}]
\item[モジュール]
\end{inlineenummerate}

%\section{nibKameのコンパイラの仕様について}
コンパイラはいくつかのフェーズに分かれている. 
以下, 特に改良を加えたところのみを
詳しく解説する. 
\begin{inlineenummerate}
\item[構文解析]
ソースコードを読み込み\emph{抽象構文木}(\emph{S式})に変換. 
\emph{解析表現文法}(\emph{Parsing Expression Grammar}, PEG)で解析する. 
\item[抽象構文木の読み込み] 
抽象構文木(S式)を読み込み, 
中間言語(\texttt{Syntax.t})に変換する. 
\item[\emph{カリー化}] 
複数引数の関数をカリー化する. 
中間言語(\texttt{Syntax.t})を中間言語(\texttt{TypingExpr.t})に変換する. 
\item[\emph{型推論}] 
式に現れる変数全てに正しい型をつける. 
中間言語(\texttt{TypingExpr.t})を中間言語(\texttt{Typing.t})に変換する. 
\item[パターンマッチの展開]
パターンマッチの網羅性を検査し, 
複雑なパターンマッチをより簡単にする. 
%Wright\footnote{Andrew K. Wright \url{wright@research.nj.nec.com}}
%の書いたScheme向けのパターンマッチングライブラリの機能を参考にした. 
\item[K正規形に変換] 
\item[K正規形に対する最適化] 
\item[クロージャ変換] 
\item[仮想機械語に変換] 
\item[直線変換]
木構造となっている仮想機械語の命令を線形リストに変換する. 
\item[機械語に変換]
レジスタが無限にあるとして機械語に変換. 
\item[レジスタ割り当て]
実際に使用するレジスタを割り当てる. 
\item[アセンブリで出力] 
x86アセンブリで出力する. 
\end{inlineenummerate}
また, その後\emph{GNUアセンブラ}によって出力したアセンブリをアセンブルし, 
\emph{GNUリンカ}で事前に用意したランタイムとリンクする. 

現状では, 
構文解析部は\emph{Lisp}系言語の\emph{Scheme}の処理系である
\emph{Gauche}用に書かれており, 
それ以外は\emph{Objective Caml} (O'Caml)で書かれている. 
2011年1月12日の
ソースソードの目的別言語別行数内訳は以下のようになっている. 
ここで, 
O'Camlにおいては
\verb/*.ml/のファイルが実装
(C言語で言うところの\verb/.c/ファイル)
を定義し, 
\verb/*.mli/のファイルがインターフェース
(C言語で言うところのヘッダファイル)
を定義する. 

本体部分:
\begin{inlineenummerate}
\item O'Caml \verb/*.ml/ : 4175行
\item O'Caml \verb/*.mli/ : 760行
\item Scheme \verb/*.scm/ : 337行
\end{inlineenummerate}

テストケース:
\begin{inlineenummerate}
\item O'Caml \verb/*.ml/ : 1058行
\end{inlineenummerate}
\texthline
キーワード:
Objective Caml, ML, MinCaml, x86, 解析表現文法, Pakrat parsing, パターンマッチング,
カリー化, 型理論, Scheme, Gauche.
\end{document}
