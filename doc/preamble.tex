% 定理類似環境:
\newtheorem{theo}{定理}
\newtheorem{defi}[theo]{定義}
\newtheorem{lemm}[theo]{補題}
\newtheorem{prop}[theo]{命題}
\renewcommand{\proofname}{\bf 証明}
\pagestyle{myheadings}

% 人名
\newcommand{\sakamoto}{酒本 典明}
\newcommand{\kobori}{小堀 育男}

% 見出し:
\renewcommand{\lstlistlistingname}{プログラム一覧}
\newsavebox{\articleauthor}
\newcommand{\responsibility}[1]{\nopagebreak[2]\begin{flushright}[文責: #1]\end{flushright}}
\newcommand{\defresponsible}[2]{\newenvironment{#1}[3][\relax]{%
\ifx##1\relax#2[##2 (##3)]{##2}%
\else#2[##1 (##3)]{##2}\fi%
\sbox{\articleauthor}{##3}}%
{\responsibility{\usebox{\articleauthor}}}}
\defresponsible{resbonsiblechapter}{\chapter}
\defresponsible{resbonsiblesection}{\section}

% 略語:
\newcommand{\algorithmW}{algorithm $\mathscr{W}$}
\newcommand{\algorithmU}{algorithm $\mathscr{U}$}

% 記号:
\DeclareSymbolFont{symbolsC}{U}{txsyc}{m}{n}
\DeclareMathSymbol{\strictif}{\mathrel}{symbolsC}{74}
\DeclareMathSymbol{\boxright}{\mathrel}{symbolsC}{128}
\newcommand{\R}{\mathbb{R}}
\newcommand{\Z}{\mathbb{Z}}
\newcommand{\N}{\mathbb{N}}
\newcommand{\msubset}{\subseteq}
\newcommand{\mnsubset}{\nsubseteq}
\newcommand{\mpsubset}{\subset}
\newcommand{\mnpsubset}{\nsubset}
\newcommand{\msupset}{\supseteq}
\newcommand{\mnsupset}{\nsupseteq}
\newcommand{\mpsupset}{\supset}
\newcommand{\mnpsupset}{\nsupset}
\newcommand{\limpl}{\supset}
\newcommand{\gtrdotrel}{\mathrel{\gtrdot}}
\newcommand{\Rrel}{\mathrel{R}}
\newcommand{\curlyveeord}{\mathord{\curlyvee}}
\newcommand{\curlywedgeord}{\mathord{\curlywedge}}

\newcommand{\powerset}{\mathfrak{P}}
\newcommand{\domain}{\mathop{\mathfrak{Dom}}}
\newcommand{\codomain}{\mathop{\mathfrak{Cod}}}

\newcommand{\mathnkop}[1]{\mathop{\mathsf{#1}}}
\newcommand{\mathnkenv}[1]{\mathcal{#1}}
\newcommand{\mathnkval}[1]{\mathsf{#1}}
\newcommand{\mathset}[1]{\mathit{#1}}
\newcommand{\mathfunc}[1]{\mathit{#1}}
\newcommand{\newmonadicopdot}[2]{\newcommand{#1}[2]{#2##1.\,##2}}
\newmonadicopdot{\foralldot}{\forall}
\newmonadicopdot{\existsdot}{\exists}
\newmonadicopdot{\lambdadot}{\lambda}
\newcommand{\fixdot}[2]{\mathnkop{fix}#1\;#2}
\newcommand{\fundot}[2]{\mathnkop{fun}#1\rightarrow#2}
\newcommand{\semanticS}[1]{\mathcal{S}\left\llbracket#1\right\rrbracket}
\newcommand{\semanticM}[1]{\mathcal{M}\left\llbracket#1\right\rrbracket}
\newcommand{\mypair}[2]{\left(#1,#2\right)}

\newcommand{\lambdaIf}{\mathop{\mathrm{if}}}
\newcommand{\lambdaThen}{\mathop{\mathrm{then}}}
\newcommand{\lambdaElse}{\mathop{\mathrm{else}}}
\newcommand{\lambdaLet}{\mathop{\mathrm{let}}}
\newcommand{\lambdaIn}{\mathop{\mathrm{in}}}

\newcommand{\semanticSVal}{\mathop{\pi_\mathrm{val}}}
\newcommand{\semanticSVEnv}{\mathop{\pi_\rho}}

\newcommand{\ltrue}{\top}
\newcommand{\lfalse}{\bot}
\newcommand{\defeq}{\triangleq} % 定義時に使う等号

\newcommand{\envExpr}{\mathnkenv{E}}
\newcommand{\envType}{\mathnkenv{T}}
\newcommand{\envVariant}{\mathnkenv{V}}
\newcommand{\envPattern}{{\mathnkenv{E}_p}}
\newcommand{\substExpr}{{\Sigma_\mathnkenv{E}}}
\newcommand{\substType}{{\Sigma_\mathnkenv{T}}}
\newcommand{\substVariant}{{\Sigma_\mathnkenv{V}}}
\newcommand{\applysubst}[2]{\mathfrak{S}\ #1\ #2}
\newcommand{\removeassoc}[2]{\mathfrak{Rm}\ #1\ #2}
\newcommand{\patternandenv}[2]{#1\Rsh#2}
\newcommand{\valueandenv}[2]{\mypair{#1}{#2}}
\newcommand{\domainrestrict}[2]{#1\mathbin{\upharpoonright}#2}
\newcommand{\freevars}[1]{\mathit{FV}(#1)}
\newcommand{\freetypevars}[1]{\mathit{FTV}(#1)}
\newcommand{\boundvars}[1]{\mathit{BV}(#1)}
\newcommand{\clauseor}{\mathbin{|}}
\newcommand{\patternor}{\mathbin{|}}
\newcommand{\patternand}{\mathbin{@}}
\newcommand{\patternany}{\_}

% 日本語特有の記号
\newcommand{\jpdash}{―\nobreak\hspace{-0.5zw}\nobreak―\nobreak\hspace{-0.5zw}\nobreak―}

\newcommand{\rulename}[1]{\text{\bfseries\scshape #1}}
\newcommand{\typename}[1]{\text{\sffamily\bfseries #1}}
\newcommand{\widevec}[1]{\overrightarrow{#1}}
\newcommand{\varrange}[1]{\boldsymbol{\widetilde{#1}}} %\widetriangle \widering \wideparen \widetilde \widehat のいづれかが良いと思う
\newcommand{\refEq}[1]{式\ref{#1}}
\newcommand{\refTbl}[1]{表\ref{#1}}
\newcommand{\refFig}[1]{図\ref{#1}}
\newcommand{\refCh}[1]{\ref{#1}章}
\newcommand{\refSc}[1]{\ref{#1}節}
\newcommand{\refSsc}[1]{\ref{#1}項}
\newcommand{\refSssc}[1]{\ref{#1}目}

% 罫線に太線を追加
\makeatletter
\def\@arrayrule{\@addtopreamble{%
\vrule \@width \arrayrulewidth}}
\def\Hline{\noalign{\hrule height 3\arrayrulewidth}}
%\def\LTHline{\noalign{\hrule height 3\arrayrulewidth}}
\def\Vline{\vrule width 3\arrayrulewidth}
%\def\LTHline{\LT@Hline}
\def\LT@Hline{%
  \noalign{\ifnum0=`}\fi
    \penalty\@M
    \futurelet\@let@token\LT@@Hline}
\def\LT@@Hline{%
  \ifx\@let@token\Hline
    \global\let\@gtempa\@gobble
    \gdef\LT@sep{\penalty-\@medpenalty\vskip\doublerulesep}%
  \else
    \global\let\@gtempa\@empty
    \gdef\LT@sep{\penalty-\@lowpenalty\vskip-\arrayrulewidth}%
  \fi
  \ifnum0=`{\fi}%
  \multispan\LT@cols
     \unskip\leaders\hrule\@height3\arrayrulewidth\hfill\cr
  \noalign{\LT@sep}%
  \multispan\LT@cols
     \unskip\leaders\hrule\@height3\arrayrulewidth\hfill\cr
  \noalign{\penalty\@M}%
  \@gtempa}
\def\LT@makecaption#1#2#3{%
  \LT@mcol\LT@cols c{\hbox to\z@%
  {\hss\parbox[t]\LTcapwidth{%
    \sbox\@tempboxa{#1{#2\hskip1zw\relax}#3}%
    \ifdim\wd\@tempboxa>\hsize
      #1{#2\hskip1zw\relax}#3%
    \else%
      \hbox to\hsize{\hfil\box\@tempboxa\hfil}%
    \fi%
    \endgraf\vskip\baselineskip}%
  \hss}}}%
\def\LT@array[#1]#2{%
  \refstepcounter{table}\stepcounter{LT@tables}%
  \if l#1%
    \LTleft\z@ \LTright\fill
  \else\if r#1%
    \LTleft\fill \LTright\z@
  \else\if c#1%
    \LTleft\fill \LTright\fill
  \fi\fi\fi
  \let\LT@mcol\multicolumn
  \let\LT@@tabarray\@tabarray
  \let\LT@@hl\hline
  \def\@tabarray{%
    \let\hline\LT@@hl
    \LT@@tabarray}%
  \let\\\LT@tabularcr\let\tabularnewline\\%
  \def\newpage{\noalign{\break}}%
  \def\pagebreak{\noalign{\ifnum`}=0\fi\@testopt{\LT@no@pgbk-}4}%
  \def\nopagebreak{\noalign{\ifnum`}=0\fi\@testopt\LT@no@pgbk4}%
  \let\hline\LT@hline \let\kill\LT@kill\let\caption\LT@caption
  \let\Hline\LT@Hline
% \@tempdima\ht\strutbox%                                       変更1
  \iftdir\@tempdima\ht\tstrutbox\else\@tempdima\ht\strutbox\fi%  <-
  \let\@endpbox\LT@endpbox
  \ifx\extrarowheight\@undefined
    \let\@acol\@tabacol
    \let\@classz\@tabclassz \let\@classiv\@tabclassiv
    \def\@startpbox{\vtop\LT@startpbox}%
    \let\@@startpbox\@startpbox
    \let\@@endpbox\@endpbox
    \let\LT@LL@FM@cr\@tabularcr
  \else
    \advance\@tempdima\extrarowheight
    \col@sep\tabcolsep
    \let\@startpbox\LT@startpbox\let\LT@LL@FM@cr\@arraycr
  \fi
  \setbox\@arstrutbox\hbox{\vrule
    \@height \arraystretch \@tempdima
%   \@depth \arraystretch \dp \strutbox%          変更2
    \iftdir\@depth \arraystretch \dp \tstrutbox%   <-
    \else\@depth \arraystretch \dp \strutbox\fi%   <-
    \@width \z@}%
  \let\@sharp##\let\protect\relax
   \begingroup
    \@mkpream{#2}%
    \xdef\LT@bchunk{%
       \global\advance\c@LT@chunks\@ne
       \global\LT@rows\z@\setbox\z@\vbox\bgroup
       \LT@setprevdepth
       \tabskip\LTleft \noexpand\halign to\hsize\bgroup
      \tabskip\z@ \@arstrut \@preamble \tabskip\LTright \cr}%
  \endgroup
  \expandafter\LT@nofcols\LT@bchunk&\LT@nofcols
  \LT@make@row
% \m@th\let\par\@empty%                               変更3
  \iftdir\m@th\let\par\@@par%                          <-
  \else\m@th\let\par\@empty\fi%                        <-
  \everycr{}\lineskip\z@\baselineskip\z@
  \LT@bchunk}

\makeatother

% listings用設定
\lstdefinelanguage{nibkame}{%
   morekeywords={%
        let,letrec,in,if,match,with,type,fun,+,-,*,/,+.,-.,*.,\.,(),~-,;,
        and,or,not,unit,Nil%
   },%
%   morekeywords={%
%        ignore
%   },%
%   morekeywords={%
%        print_int,print_float,print_char,print_string
%   },%
%   morekeywords={%
%        hd,tl,null,map,length
%   },%
%   morekeywords={%
%        array-create,array-set,array-ref,array-from-list,array-from-list-with-length
%   },%
   sensitive,% ???
   alsodigit=->,%
   morecomment=[l];,%
   morecomment=[s]{\#|}{|\#},%
   morestring=[b]",%"
   literate=%
       {[|}{[\hskip -1pt$|$}2%
       {|]}{$|$\hskip -1pt]}2%
%      {[]}{\ensuremath{[\hskip -0.1em]}}2%
       {->}{\ensuremath{\rightarrow}~}2%
       {::}{\ensuremath{:\hskip -0.1em:}~}2%
}[keywords,comments,strings]

\lstdefinelanguage[Objective display]{Caml}[Objective]{Caml}{%
    morestring=[d]',%'
%    classoffset=2,
%    morekeywords={int,float,char,string},keywordstyle=\color{red},
%    classoffset=3,
%    morekeywords={list,array},keywordstyle=\color{blue},
%    classoffset=0,
    literate=%
        {[|}{[\hskip -1pt$|$}2%
        {|]}{$|$\hskip -1pt]}2%
%       {[]}{\ensuremath{[\hskip -0.1em]}}2%
        {->}{\ensuremath{\rightarrow}~}2%
        {::}{\ensuremath{:\hskip -0.1em:}~}2%
}
\lstset{%
  %language=[Objective display]Caml,%
  language=nibkame,%
  basicstyle={\normalfont\normalsize\sffamily},%
  commentstyle={\small\ttfamily\itshape\bfseries\upshape},%
  classoffset=1,%
  keywordstyle={\bfseries},%
  %frame=,%
  %framesep=0pt,%
  showstringspaces=false,%
  %numbers=left,%
  %numberstyle={\scriptsize},%
  %stepnumber=1,%
  tabsize=8,%
  lineskip=-0.5ex,%
%
  breaklines=true,%
  linewidth=\the\textwidth,
%  columns=[l]flexible%
  columns=flexible%
}
